\section{Aeran Style Guide}
OVERVIEW:


PARTS:

    * ENGINES (propulsion)
          o Engines (Strike)
          o Engines (Shuttle)
          o Engines (Subcapital)
          o Engines (Freight)
          o Engines (Capital)
          o Overdrive


Aeran engines tend to be centrally mounted along the forward-rear axis, except in some smaller craft.
Only the exhaust vents tend to be visible, the rest of the engine being internal
The exhaust vent calderas can be very large, with several actual exhaust exit points within one caldera.
Overdrive units exist only for strike craft. These modify the engine to handle a greater quantity of exhaust at the cost of decreased fuel efficiency and increased engine wear.


    * THRUSTERS (maneuvering and attitude control)
          o Thrusters (Strike)
          o Thrusters (Shuttle)
          o Thrusters (Subcapital)
          o Thrusters (Freight)
          o Thrusters (Capital)


Aeran thrusters protrude only slightly above the surface of the ship.
They are compact, and roughly cylindrical, with a wider exhaust head than body.
Larger classes of thruster emplacement do not scale beyond Subcapital. Beyond subcapital, a thruster emplacement simply has MORE thruster heads, and a ship has more thruster emplacements.
They look similar to the below:



    * REACTORS
          o Reactors (Strike)
          o Reactors (Shuttle)
          o Reactors (Subcapital)
          o Reactors (Capital)
          o Reactor (Starbase)


Aeran reactors are fusion reactors.
Their building blocks are toroidal (donut) in shape. Standard design features a stack of two alternating radii of toroids.
Larger reactors within a subclass{strike, shuttle, etc.} have more toroids.
Larger subclasses have larger toroids.
Larger ships have several reactors of possibly varying size.
Shuttle reactors are lower performance, but lower maintenance.
Strike craft will have either 1 or 2 reactors.
Reactors are covered with coolant plumbing and electromagnetic plasma guides/stabilizers
The following is a (very) rough sketch of the basic toroids:


    * SENSORS
          o Sensors (Active)
            "Gravitic" - should look somehow related to shields. Probably spiny.
            LIDAR/RADAR - Spinning parabolic dishes, radar domes, or fixed arrays of other similar emitters.

          o Sensors (Passive)
            Radar domes
            Optical pickups - flat, dark hexagonal patches with darker spherical protrusion.
            Radio receivers (looks like antennae -- mix of 2-D embedded, like in back of car window, and spiny transceiver style radio tower)
            "Gravitic pickups" - should look related to shields. Oddly shaped.


    * TURRETS
          o Turrets (Strike)
          o Turrets (Subcapital)
          o Turrets (Capital)
          o Turrets (Starbase)
          o Turrets (Point-defense)

            All Aeran Point-Defense turrets are laser turrets.
            Most Capital turrets have either a Sarissa or a Xiphos.
            No strike craft have strike turrets. The smallest Aeran craft with (anti-)strike turrets is a corvette.
            The Dory is the most common weapon in a sub-capital turret.
            The largest Aeran military starbases may mount unique turreted laser weapons. Others just mount modified versions of capital turrets.
            Aeran turrets are heavily armored, even to the point of increasing tracking time.
            Larger turrets will be exclusively specialized for either energy weapons (massive cooling) or projectile weapons (massive ammo feed/supply). Smaller turrets are generalists, mounting fixed standard weapons and then pivoting them.


    * SHIELDS
          o GEM Shield (Strike/Shuttle)
          o GEM Shield (Subcapital/Freight)
          o GEM Shield (Capital)
          o GEM Shield (Starbase)


Look vaguely like sowbugs.

    * CARGO/SUPPLY PODS/TANKS
          o Internal Cargo Pods
          o External Cargo Pods
          o Fuel Tanks (Internal)
          o Fuel Tanks (External)
          o Coolant tanks (internal)
          o Coolant tanks (external)

To be added

    * WEAPON MOUNTS
          o Internal Hardpoints
            Weapons slot into hexagonal openings. Weapon bay heavily armored.
            Expected weapon sizes are long and skinny. Slightly skinnier and slightly longer compared to human equivalents.

          o External Hardpoints

Are commonly positioned on the outer face of a rose-thorn projection like below:

          o Ammo feeders
            To be added
          o Ammunition reserves
            To be added

          o Missile Bays (Rocket/Missile/Bomb)
            Are specifically purposed for missiles, but can be upgraded to hold different types.
            Missile packs tend to be tightly packed groups of hexagonal tubes.
          o Missile Bays (Torpedo)
            Are usually intrinsic either to the chassis or to the torpedo turret.
          o Missile Bays (Capital)
            Are so large that they are always part of the main chassis design.
          o Spinal Weaponry (strike)
            Are rather variable.
          o Spinal Weaponry (Subcapital)
            As the center is occupied by the engine, Aeran spinals tend to be radially arranged around the main armored tube, or are built into the hammerheads.
          o Spinal Weaponry (Capital)
            As the center is occupied by the engine, Capital spinal mounts are radially arranged around the main armored tube.
          o Intrinsic Weaponry (Starbase)
            High-end Aeran military starbases may have massive internal turret-mounted laser generators. The laser generator itself is safe and unseen within the bowels of the station, but the fixed emitters can still be destroyed. The truly lucky may fire down the ruins of an emitter tube to damage the actual laser.



    * CAPACITORS
          o Capacitors (Weapons)
          o Capacitors (FTL)


Are rarely visible from the outside of a vessel. Similar physics and function makes them very similar to human constructed banks of capacitors.

    * FTL
          o In-system FTL

Almost never seen from the outside of any vessel, especially an Aeran one. Internal system, connected internally to shield emitters. Most species FTL drives look a bit like a cross between a metal and plastic bird's nest and what an automobile looks like when it's been through a scrapyard car-crusher. While this is true for the Aera as well, their device looks cleaner and more elegant than human models.
Upgraded FTL systems look like they've had new parts glued on.

          o Jump Drive

Looks very similar to the in-system drive, but is larger, and has 2 distinct nodes at front and rear. It is likewise internal, and not directly seen.
Jump drive models scale very little among ship sizes.
Externally visible due to additional short, stubby projections with miniature shield-emitters on them being placed at the fore and rear of a vessel.
Larger vessels will have more stubby projections, and these projections will be larger (but still quite small in proportion to the vessel)


    * SURFACE SYSTEMS
          o Aesthetic fluff (Decals/Paintjob/etc.)
          o Radiators
            Are usually present on hammerheads and rosethorn projections on larger ships
            Are large and mostly flat.
            Radiators are present both as intrinsic (rosethorn/hammerhead) and distinct, modular components for strike-craft.
            Radiators will glow a dull red when active. Augmented reality may depict them more strongly, as below:

          o Armor


    * MINOR SYSTEMS / UPGRADES
          o Damage Control (Strike/Shuttle)
          o Damage Control (Subcapital)
          o Damage Control (Freight)
          o Damage Control (Capital)
          o Damage Control (Starbase)
          o Passenger Cabins
          o Life Support
          o Escape Pods

            To be added



    * ELECTRONICS/ELECTRONIC WARFARE
          o ECM
          o ECCM
          o HUD Visualizations

            To be added


    * STANDARD WEAPONS
          o Projectile
                + Makhaira
                  Are artillery pieces that fire a warhead at moderate velocity. The primary damage vector is the detonation of the warhead on contact with an enemy shield.
                  The barrels are fairly modest, <10 meters long, as they do not rely a kinetic damage vector. They are very sturdily constructed, and most of their innards are obscured behind an armored covering.
                  The Makhaira is not turreted on strike craft.
                  On larger craft, it is used as a turreted anti-strike weapon. In these emplacements, the armor shroud may be elided, with the armor of the turret instead providing protection.
                  They look much like the below:

                + Xiphos
                  Capital scaled version of the Makhaira. At this size, they begin to bear some resemblance to the guns on a WWII battleship. The Xiphos is almost always seen turreted.
                + Kopis
                  Capital scale linear accelerator, usually > 1 Km long.
                  See below:

          o Laser
                + Hoplon
                  Point defense laser, always turreted. Red-laser with ball-mirror turret in hexagonal socket. Fires very short bursts.
                  The Thermopylae system uses several hoplon in a tight cluster.
                + Dory
                  Light assault laser primarily found in Areus and corvette class vessels.
                  Usually deployed for maximum efficiency sans turret, i.e. aiming the ship, not the laser.
                + Sarissa
                  Capital scale lasers. Usually mounted at end of rosethorn tips. Lighter models use a multi-phase turreting system with shallow angles of deflection over several points and multiple fixed lenses.
                  The highest end Sarissas, (UV and X) are either turetted to move the entire laser assembly, or mounted in a fixed position, depending on the size of the vessel in question.
                  Some of the high-end UV Sarissa mounts are internal turrets - where the laser generator moves within a sphere in the center of the ship, firing down a wide, empty path to one of several semi-fixed lenses (these were only built during the Aeran-Rlaan war, however).

          o Missile
                + Missile
                + Torpedo
                + Capital Missile
                + Capital Kinetic

    * UNIQUE WEAPONS
          o Oxybeles
            An enormously upscaled Kopis, unique to the Leonidas class. Approximately 7Km in length, running the length of the entire vessel.


    * COMMON SHIP FEATURES

          o Teardrop hammerhead (Aeran)
            (see center)
            c

          o Rosethorn spire (Aeran)
            (see upper right)

          o Central armored tube (Aeran)
            (see middle and bottom)

          o Molded cosmetic shells (Aeran)

                Aeran hulls often appear to be constructed from a single piece as if they were carved from a large block of alloy into their final shape.  In reality this is not true but they do like to hide this fact.  Aeran hulls have a layered effect.

          o Track lighting (Aeran)

                Aeran craft feature strips of cyan-colored track lighting to not only show off their craftsmanship but also to light logical pathways to service hatches and such.
                (See below for shell and lighting)               

          o Hierarchical hexagonal prism cargo pod rings (Aeran Merchant)
            Aeran cargo containers are hexagonal prisms.

          o Access bay for reactor torus-stack (Aeran)
            See section above on reactors.

          o Docking spar (Aeran)

                Docking spars allow vessels to transfer cargo, munitions, equipment, supplies, and personnel between each other or space stations while maintaining a safe standoff distance to avoid collisions.


\wfigure[Aeran_style_guide1.pdf,Aeran Style Guide - page 1,guide-Aeran,Aeran Style Guide - 1]
\wfigure[Aeran_style_guide2.pdf,Aeran Style Guide - page 2,guide-Aeran,Aeran Style Guide - 2]
\wfigure[Aeran_style_guide3.pdf,Aeran Style Guide - page 3,guide-Aeran,Aeran Style Guide - 3]
\wfigure[Aeran_style_guide4.pdf,Aeran Style Guide - page 4,guide-Aeran,Aeran Style Guide - 4]
\wfigure[Aeran_style_guide5.pdf,Aeran Style Guide - page 5,guide-Aeran,Aeran Style Guide - 5]
\wfigure[Aeran_style_guide6.pdf,Aeran Style Guide - page 6,guide-Aeran,Aeran Style Guide - 6]
\wfigure[Aeran_style_guide7.pdf,Aeran Style Guide - page 7,guide-Aeran,Aeran Style Guide - 7]

