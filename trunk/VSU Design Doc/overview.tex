\label{chapt:overview}
This chapter, along with Chapter~\ref{chapt:VSUreality} and
Chapter~\ref{chapt:livinginVSU}, provides an overview of some of the
high-level aspects of the Vega Strike Universe and the design
philosophies at play. They indicate with broad strokes what forms of
content and approaches to topic matter will be appropriate for the
Vega Strike Universe. This chapter primarily focuses on the history of
the VSU's design, the forces and influences at play in its genesis,
and outlines some of the broader design philosophies and directions
for both the Vega Strike Universe and the games it is designed to
support. More detailed discussions of the nature of the VSU in terms
of physics and social realities are left for later chapters.

\ignore{
The rest of this document is organized as
follows. Chapter~\ref{chapt:timelines} provides an outline of the
entire scope of VSU history from an omniscient viewpoint, and delves
into details of particular historically significant
events. Chapter~\ref{chapt:PCplots} introduces Deucalion, the player
character for {\it Upon the Coldest Sea}, and discusses the
personal-scale plots that he may be come involved in during the UtCS
time period. Chapter~\ref{chapt:portfolios} provides one-stop shopping
for those, such as artists, looking to find all relevant information
concerning a particular group, as well as more sections discussing
particular types of objects, such as vessels, and how they should be
depicted to be in-line with the VSU
canon. Chapter~\ref{chapt:uncategorized} contains a potpourri of
information yet to be re-located, or lacking a more appropriate place
in the document. Appendices~\ref{appendix:Species}, \ref{appendix:Factions}, and \ref{appendix:trailer} are then presented, followed by the~\ref{References:References}, \ref{Glossary:Glossary}, and \ref{Index:Index}.
}

\section{The nature of the Vega Strike Universe}
\label{sec:VSUflavor}
Everyone, it seems, has their own, often deeply held, views as to what
the future will be like. The Vega Strike Universe, as a whole, doesn't
happen to correspond to any of the developers' ideas of what the future
will actually be like. Let's pull no punches - real space travel and
conflict, should it ever occur, will almost certainly be arduously
long, heavily automated, and completely boring to look at without a
lot of magnification and false-color filters. FTL? Probably not
possible. Manned airplane-like fighter-craft? Unlikely - missiles and
probes do most of the potential jobs a lot better, can withstand much
larger material stresses, don't die from radiation sickness, or run
out of food and water if it takes weeks to get to the target - which,
if there's no FTL, it probably will. As one can imagine, the list goes
on.

\subsection{Requirements of the VSU}
What, one may well then ask, does the Vega Strike Universe correspond
to? The genesis of the VSU started with a set of game premises - there
was to be an Elite~\cite{Elite} style game heavily influenced by
Privateer~\cite{Privateer} featuring humans and at least two other
prominent alien species. There was to be an ongoing interstellar war
between the humans and at least one of the prominent alien
groups. Scales of celestial objects and distances were to be
comparable to the known scales of celestial objects. Physics was to be
respected where possible. Complicating things greatly was a desire for
the game to be multi-player, thus ensuring that arbitrary dynamic time
dilation would not be a viable game-play mechanic - much of the game
must happen in something approximating ``real-time''.

%%%%%%%%%%%%
%%%%%%%%%%%%


\subsubsection{The VS game specification}
\label{Danny's stuff goes here}



%%%%%%%%%%%%
%%%%%%%%%%%%


\subsection{Genesis of the VSU: dependencies among game, universe, and game-engine}
The VSU, therefore, has to provide an environment in which all of
these prerequisite sets of features can coexist. Serving a particular
type of game and style of game-play, it is constrained in ways that
lead to compromise, both with reality as we know it or expect it to
be, and with the aesthetic and artistic predilections of those looking
to design the universe as they would prefer it to be. Speaking as the
chief architect of the VSU, I can vouchsafe both that the VSU as it
stands diverges noticeably from the sort of fictional universe I would
create absent the particular game constraints, and that I sleep
perfectly well at night having made that compromise. The task I was
given was to oversee the creation of a universe that would allow a
particular type of game powered by the VS engine, an entirely
different beast than the VS Universe, to have, for a target audience
much larger than the project leads or the developers and contributors
themselves, an engaging and entertaining setting. There is no sin in
seeking to entertain, and neither is there shame in targeting an
audience centered far from where one stands oneself. There is a
deliberate scent of popcorn to the requirements the VSU must meet, and
that necessarily colors the universe that serves it, but that in no
way keeps us from aspiring to quality, even if we must resign
ourselves at times to painting soup cans instead of canvas.

\ignore{ Contact with alien species implies FTL. For there to be
something recognizable as an ``ongoing war'' on an interstellar
scale, neither the attacker (borders completely permeable to primary
FTL mechanism) nor the defender (fixed points of FTL transit) could be
unduly advantaged - hence two different mechanisms for FTL. Early
attempts at a unified FTL mechanism (point-point) that differed for
in-system and out-system use only in terms of some sort of presumption
of scalability were eventually retconned due both to difficulties in
reconciling the ability to jump from one side of a system to another
with the necessary drawbacks for interstellar travel that this same
system must still possess. While the early point-point mechanism was
widely criticized for being too fast, its replacement (SPEC) was
widely criticized for being too slow, except by those who continued to
argue that hours, rather than minutes, should be the bare minimum for
crossing a star system.  }


\section{Philosophy of scope and goals for plot and protagonist}
\label{sec:plottingphilosophy}

\subsection{Small protagonists in a big universe}
Many people enjoy pretending to be heroes (or villains), but I believe
that there exists a great and pervasive confusion that conflates
heroism with actions on a scale of epic myth. There are a number of
classical archetypes for heroes and their journeys. One can wander
down the Freudian/Jungian pages of Campbell's {\em Hero with a 1000
Faces}~\cite{herowith1000faces} and see everyone from Siddhartha to
Luke Skywalker.  That's not quite the sort of protagonist we're
actually going to be looking for in most interactive content set in
the VSU, at least in terms of player characters. This is because,
certainly for UtCS and likewise for most other planned tales, we're
not looking to craft a mythic journey centered on the player
character. A mythic aspect would be much more appropriate for the {\it
Rimward of Eden} setting, but even there, restraint is warranted. We
wish to avoid having the player character become the, witting or
unwitting, linchpin of all further history. We aim for epic scope in
setting, not in the degree of impact that the player characters
actions will have upon future generations to come. We wish to allow
opportunities for heroism, but on a personal scale, not a galactic or
cosmic one.

We wish the setting to be epic, and the player sized on a human,
rather than comic-book superhero scale. Compelling examples of the
personal heroic within an epic setting are not hard to come by in
either reality or fiction. One can think of Sgt. Alvin York's
exploits~\cite{SergeantYork} in the context of the first world war, or
any number of characters from various world war two epics. One can
play a very active role, both helping to effect and being affected by
the changes occurring in a turbulent period of history, without being
the primary determinant of outcomes. A VSU protagonist should not be a
Gordon Freeman~\cite{Half-Life-GordonFreeman} stand-in. They will not
be, from the perspective of the history books, the most important
person of the era. They will, however, be among the most important
people in their own lives and in the lives of those close to them.

We should not avoid big stories, epic stories of sweeping scope. Merely,
there appears to be frequent confusion that, in order to tell the tale
of \emph{Moby Dick}~\cite{MobyDick} the player somehow must have to be either
Ahab or the whale. This is why, for the purposes of story development
for VS, I want to distinguish between the big plot and the little
plots. The big plot is an epic canvas, a great, boldly painted
backdrop with thick lines and firm colors against which and within
which smaller events are set, contrasted, and constrained. The big
plot should concern itself with what would be written in history texts
after its conclusion, whereas the little plot should concern itself
with what its surviving characters would tell their grandchildren
about what they were doing during some chapter in the aforementioned
history text. The epic actions should remain mostly constrained to
actions of sufficiently sized groups, those who command such groups,
intelligible chance happening, and other such mechanisms as are
required to create the epic sweep of time and space beyond the ken of
small beings. This is not to say that players in the little plots
cannot affect the events of the big plot, it is rather they need not
do so, and are likely to only have extraordinary effect when they have
performed extraordinary action. The converse, of course, is not true -
actions that occur in the big plot can clearly have profound and
immediate or delayed and subtle effects upon the player - the fall of
a government will obviously change a player's experience if they were
to go visit that region, and the stresses of a war economy should
become apparent in various price levels, patrol levels, availability of
side tasks, etc. To rein things back a bit from the general to the
specific: UtCS (the first undertaking set in the VSU) will, at root, be the story of one man's life during
wartime, seen through his eyes,

\subsection{Appropriate player-level plots}

Cosmic significance is not necessary for compelling drama or, more
generally, good stories, drama or otherwise. In
\emph{Catch-22}~\cite{Catch22}, Yossarian never strikes a deathblow against
the Nazis. In \emph{All Quiet on the Western
Front}~\cite{AllQuietOnTheWesternFront} the protagonist accomplishes
little except surviving only to die, and if that's not a compelling
story for being too far from the SF vein, then ponder \emph{Blade
Runner}~\cite{BladeRunner} - the world is just as screwed up in the end
as in the beginning, but one is consumed by the story of even the few
people (and Replicants - not all interesting characters need be truly
human) involved. One can credit much of the success of the first
season, and the relative decline of subsequent seasons, of the new
Battlestar Galactica series~\cite{NewBattleStar} to the strength that
can come from focusing on human interactions and frailties even in a
dire and cosmically important situation. In short, it is sufficient
for the story the protagonist is directly entangled in to be important
to the protagonist and those around him/her, even if it is not direly
important to most of the other beings in existence.

When it comes to specific plots, there are too many reasonable ones to
list, although something that meshes well with the themes discussed in
Section~\ref{sec:VSthemes} would be preferential. There are, however,
a number of approaches that one is explicitly dissuaded from. 
\begin{itemize}

\item Morality plays. 

I'm not interested in the crafting of thinly disguised morality plays,
as many Star Trek episodes often amounted to. A game set in the VSU
probably isn't the right platform for beating the audience over the
head with the fact that having a war because one group is black on the
right side and white on the left and the other group is white on the
right side and black on the left is both a demonstrably unjustifiable
cause and a forced allegory of 1960s American race relations.

\item Juvenile empowerment fantasies

I'd prefer to limit the degree to which things devolve into an
expression of teenage fantasies of godlike empowerment - Dragonball Z~\cite{DBZ}
and it's level-up obsessed ilk are not a good starting point for much of
anything, even ignoring the sequel strangling power growth whereby
planets are being absentmindedly smashed. If players want to blow up
planets, let them hack the files themselves. There's no need for us to
hand them a chest-thumping anthem of non-constructive violence on a
platter. Likewise, there's no need for us to make it easy, nor even
possible, for the player to have ``the best'' attainable possession as it presumes both
that such a monotonic ordering over all objects exist and that the
purpose of the game is somehow to reward them with increasing levels
of destructive capacity or shiny whatnots to sate a lust for unending
one-upsmanship. Neither is the case.

\item Standard RPG tropes

The standard RPG premise wherein the young weakling levels up
repeatedly until he becomes a veritable force of nature that the
course of all history depends upon is not a desirable goal. Even the
more subdued Freelancer~\cite{Freelancer} variant thereof is something
to be avoided - especially the standard "enemies keep getting more
powerful just to match the player's progress despite the fact that
this means that later stage enemies could slaughter entire
civilizations from earlier in the game" and "fate of humanity rests
upon the player" problems. Privateer~\cite{Privateer} levels of player
significance are probably as far as we should consider treading, and,
where possible for VS, I'd like to move the bulk of the significance
to before the player takes control of the character. I find it
preferable to have a character intrinsically with some significance
and then let the player do with that what they will rather than to
force a player down a path that will cause them to become
significant. I think it's much more interesting to have ( as a purely
hypothetical and unrelated-to-VS example) a scenario where a powerful
entity decides to spend its time surfing, gambling and sowing its wild
oats instead of saving the human race because it just didn't feel the
motivation or didn't want to risk it than to yank the player
incongruously along every time they start to wander down a path that
doesn't take them toward becoming some "powerful entity". Cosmic
significance for an individual is really difficult to not screw up in
SF. Perhaps it may be less so in a fantasy setting where there are
easier outlets for suspension of disbelief compatible with the base
universe. However, the VSU strongly attempts to not be just such a
universe.

\item Excessive romanticizing

This category has several subcategories with similar root short-comings:
\begin{itemize}

\item Pretending that human expansion into space is some previous time period

Those who cannot distinguish between drawing parallels and forcing
them are doomed to repeat the past in ways extremely implausible to
occur again. Even if certain periods, most notably, \emph{Icarus
descended} have notable similarities with the expansion of the western
American frontier, this isn't a reason to bring back the ten-gallon
hat and spurs. Flogging the midshipman is not going to come back into
style. Likewise, feudal empires and other such historical arrangements
of governance and/or society (at least as we know them in humans) tend
to arise out of particular circumstances that may be difficult to
reconcile with the current state of a space-faring civilization.

\item Excessive romanticizing of Neo-Feudalism

There's a reason why we most of us aren't ruled by monarchies
anymore. Correction - there are a staggering multitude of reasons most
of us aren't ruled by monarchies anymore.

\item Excessive romanticizing of anarchism

No matter how much you may rage against paying taxes, governments
aren't going away any time soon. Anyone who believes without any doubt
that we'd be better off without any government at all has probably
never read Niven's ~\emph{Cloak of
Anarchy}~\cite{PurpleRobeofAnarchy}, and should.

\item Returns to a golden age

There never was a golden age when everything was better, so it's hard
to return to it. The quest for utopia is an unending struggle for the
unattainable, not a nostalgic search for that presumed lost.

\item Noble savages

No. Really, just no. Not only is this attempted Rousseauism almost
certainly inappropriate, it's also a misconception of Rousseauism (the
phrase is actually from Dryden's 1672 work, \emph{The Conquest of
Granada}).

\end{itemize}

\end{itemize}

\subsection{Moral certainty, prerequisite idiocy, and other things worth avoiding} 

As for characters themselves, let the fools be fools and the sages,
sages. However, no matter how cynical one may be, constructing plots
that require the vast majority of entities involved to be either
blithering idiots, willfully ignorant, or (even worse)
schizophrenically swinging between brilliance and incompetence, is
poor craftsmanship. Hinging a plot on some important thing not being
known isn't a bad thing, but if, absent intentional dramatic irony,
the readers are all able to discern what is occurring several chapters
ahead of any of the characters, the author is serving the reader
poorly. Although there may be something to be said for wandering down
a cliche path and then twisting it at the end to inform the audience
that you were aware of the cliches - Alan Moore's {\emph
Watchmen}~\cite{Moore-Watchmen} has some good examples of such twists - 
you need to really know what you're doing to avoid botching such an
attempt at deconstruction in an irredeemable fashion. The odds are that
the people working on VS stories will be capable, but not masters of
their field, and should thus likely limit their ambitions for literary
cleverness accordingly. Equally important to avoiding unnecessary
idiocy on the part of the characters is to avoid ``genius'' or
``flashes of genius'' as the chronic deus ex machina for plot
resolution( Stargate Atlantis~\cite{StargateAtlantis} I'm talking to
you here). The Manhattan project wasn't finished in a day, or a week,
or a month, and involved large numbers of painfully brilliant people
who expanded on existing bodies of work - the atomic bomb was not
summoned whole from the intellectual aether. As one of multiple people
on the VS design team who has worked in actual research labs, the
``genius model'' for problem resolution as portrayed in many forms of
media is somewhat insulting to the actual amount of effort and
infrastructure it tends to take to actually accomplish interesting
research. 

Hinging a plot on an otherwise perfect plan crafted by keenly
intelligent planners having a "fatal flaw" that allows "good" to
triumph over "evil" is just plain old fashioned bad storytelling, and
magical thinking at that. Let me be clear on this - there is nothing
good or evil in the VSU, but the thinking of some group therein terms
it so (my apologies to the Bard).  \emph{Good} and \emph{Evil} of the
capitalized variety are good fodder for terrorizing children and
certain underachieving elected officials, but, at the risk of
alienating the vast potential audience of the nuance-averse, to the
degree possible and appropriate, material set in the VSU should deal
with the more complicated and more interesting problems of
adults. There are settings where this will not be entirely the case
(stories in the Rimward of Eden setting, for instance would be
reasonable places for "coming of age" themes, even if it's an Aera
coming of age) but, while some have said that the golden age of SF was
not the late 1930's to the 1950's but was instead 14, there is nothing that
prevents us from telling realistically complicated stories in an SF
setting. Everyone who's been a teenager already did the teen angst
thing and can perhaps thus relate to it somehow, but there's no reason
to keep reveling in it, even if many of us can't always keep from
doing so. If we manage to get ourselves accused of moral ambiguity, of
an uncomfortable gap between actions noble and actions necessary,
we're probably on the right track. There is a very telling scene in
\emph{The Third Man}~\cite{ThirdMan} wherein it is revealed that the
protagonist, Martins, echoes the works of Zane Grey~\cite{ZaneGrey} in
his own pulp western writings. Showing that he still actually believes
in the black hat and white hat archetypes of simplified good and evil
present in such works, Martins then proceeds by failing to adequately
capture the nuances and complexities of the people and situations
around him, generally worsening the outcomes for all involved. In so
doing, he shows himself, and his views, to be childlike amidst those
who, for better or worse, have moved on to more adult world views.

\subsection{Blank slate vs. pre-determined protagonists}

There is a lot of room for character development between the extremes
of the stock amnesiac character who will come to be defined solely by
player actions and the traditional JRPG characters who remain
completely as pre-defined no matter what actions the player may take
or wish to take. Moreover, there are different types of games that are
best suited to different points in this spectrum. If one wishes to
tell a somewhat particular story, it is useful to have an anchor, but
if one is primarily offering a sandbox, then there is no need for such
a restriction. The latter is appropriate for multiplayer and
``free-play'' modes, which games set in the VSU should support
readily, as the setting contains stories independent of the
single-player player character. The former, anchored approach,
however, is what we will focus on for our first major game attempt in
UtCS. There is a tension between defining the character enough to
allow the telling of pre-crafted stories in a sensible fashion, and
allowing enough flexibility and growth potential that the player feels
that they have control over an avatar of their own will, and are not
merely being asked to bang on buttons until the plot moves forward
linearly. For our first effort, UtCS, we will take the approach of
joining a character with a well defined past, Deucalion, at a moment
of considerable unrest in both his personal life and the universe
around him. Thus, he is both defined and pliable. How he has been
defined makes the degrees of pliability finite, and not all equally
likely, but it will still allow each player to develop Deucalion in
the fashion they wish him to be played.

Let us consider this example in more detail. At the beginning of UtCS,
Deucalion is a bit of an emotional mess, but this is to be
expected. He's just gone through a traumatic event that nearly killed
him and did kill his best friend and (for simplification)
brother-in-law. He's rather a bit shaken, his life plans have just
been rudely interrupted, and, to add insult to injury, the Aera have
just started invading Forsaken space while Deucalion was recovering
from his injuries. It's a situation that is hobbled with guilt and the
impotence of any individual against the blind and uncaring motions of
the universe. It's a situation ripe for catharsis and, just as
importantly, for life altering change - it is therefore a good
starting point for a player to take over a character who already has a
defined past. They can take the character in rather different
directions without it breaking all suspension of disbelief, as "he
cracked" and "you're not the same since XXX" are easily intelligible
human reactions to tragedy. At the same time, they can use the
existing past and the benefits and limitations it has bestowed upon
the character as either guide or leverage - his past gives him skills
that make him useful, hence a target for interaction with other
entities, and it limits his direct involvement with governance or
military forces. As he's already left the Protectorate navy, he can
much more believably stray further, or wander back - the choice is
his, and thus, the choice is the player's.

There is always the issue in an open-ended game of when things end -
where is the resolution? For UtCS, the big plot will advance whether
or not the player does too much. For pragmatic purposes corresponding
to player learning curves, we may decide to put the galactic events of
the Big Plot on some degree of pause during the prologue or first
chapter, but beyond that, events will play out as scripted whether the
player pushes them along or not. If one only wants to see what happens
in the big picture, one need only play a bit, and then play a
(safety-seeking) waiting game - in doing so, one has played out a
rather boring life, but it's a player's choice. If they aren't
actively playing, then they won't get to see the effects.  For UtCS, I
think having a number of soft-endings for a set of little plots
(hence, the plurality) is the best choice. Profession/lifestyle
specific plots can have an apex or plateau if not a conclusion.

\section{Philosophical assumptions}

In this section, we briefly note some of the underlying axiomatic
assumptions at play in our crafting of the Vega Strike Universe. We
don't ask that contributers necessarily agree with these assumptions,
but we do ask that they be respected when working within the confines
of the VSU. Whether or not one believes they apply to the real world,
within the VSU, the following should be held to be true, if sometimes
not self-evident.

\begin{itemize}

\item There is a measure of absurdity to existence

The VSU does not make accessible to its inhabitants objective morals,
valuations, and innate purposes, although many of its inhabitants do
not choose to view it in this fashion. No matter how much a given
group within the VSU would benefit from validation, we, as the
creators of this universe are under no obligations to provide it, and,
indeed, should not. Doing so would make us (the authors)
interventionist deities, and the actions of such will not be seen in
the VSU (See also: Section ~\ref{sec:thingsnotinVSU}).

\item Some problems cannot be solved

One of the defining characteristics of Golden Age American science
fiction is a reliance on ingenious (often technological) solutions to
every problem that presented itself. This was an expression of
unbridled optimism, a romanticizing of ``good old American
know-how'' and ``pioneer values,'' and the firmly held belief that man
is an unlimited creature. In these works, not only were all problems
capable of being solved, but, specifically, a hero would always solve
them when the need arose. These sentiments would not be well-echoed in
post-WWII science fiction authors, some of whom presented futures so
pessimistic as to paint humanity's struggles as innately
futile. Alexandr Kramer's 1979 short story
\emph{That Invincible Human Spirit, or, The Golden
Ships}~\cite{GoldenShips} is a good example of a work that works
directly against many of the presumptions of the American Golden
age. Likewise, later American authors, especially those influenced by
such missteps as the American role in Vietnam, would come to present
futures that, if not necessarily bleak, were as likely to be damaged
by our actions and innovations as saved by them.

The VSU is not so pessimistic and futile a future as those such as
Kramer imagined, but it does deny some of the premises of the earlier,
optimistic ages. Some problems are inherently difficult, others
actually without solution, and some have solutions, but only
unpalatable ones. The universe is limited, and we with it likewise
so. Perhaps most important to the scope of this document and the
construction of narratives, there are nearly innumerable problems
that, while solvable, have solutions that cannot be reasonably
effected by single individuals, such as the player character.

When it comes to solutions to problems, it is also very important to
consider limitations of foresight. The locally optimal solution may
not be the globally optimal solution. Poor choices can be made, not
because the decision making process was flawed, but because the
information needed to determine that another choice would be
preferable was unknown or even unknowable at the time.

\item Perfection is usually either unattainable, or subjective

Most valuations that ascribe perfection to something are subjective in
nature or arbitrarily defined. In particular, this topic is raised so
as to make the following point explicit: Humanity is not
perfectible. Moreover, introducing perfect objects, concepts,
etc. should be generally avoided and approached with caution, as they
are probably implausible.

\item Good will not always triumph over evil

Even assuming that both good and evil are well defined from the
perspective of the narrative in question, there is no requirement that
one should defeat the other, nor even that conflict between them is in
some way inevitable, save as conflicts between any two distinct world
views are in some way inevitable. The VSU is not defined by conflicts
among innately opposed absolute moralities, but rather by conflicts
among a myriad of overlapping arbitrary ideologies.

\end{itemize}


\section{Themes in Vega Strike}
\label{sec:VSthemes}
There are certain themes which are either already introduced directly
via the existing VSU history and heavily featured, or that we feel
would be particularly wells suited to explore in the VSU setting.

\subsection{Generational conflict and impositions}
One recurring theme that is present in the overarching VS universe
story (although echoes of it will be seen in the player stories as
well) is that of intergenerational relations, of the dynamics among parents and their
children. In this case, the figurative species-children of various
groups and the effects that the actions of those who came before them
have on their own development. Our parents are our first gods, and
even if they succeed as parents, they will always fail us as gods. I
thought it would be particularly interesting then, to start the VSU
cosmology off with a parental group, the TWHON, that really was, in
many ways, godlike in the scope of its powers - but still failed, and
in their greatness, had their failure cast a commensurately large
continuing shadow over those who came after them.

The history of the VSU, however, as much as the TWHON may impact the
events that unfold, should not be seen as a story about them. Their
day, and indeed, their greatness, and even their existence as anything
but vague and warped shadow of what they once were, has long passed by
the time we examine the VSU. While it is worthwhile to understand
their place in the story, I would caution giving them too lavish a
portion of our attentions - gods may seem interesting, but stories
about people are much more involving in the long term, if for no other
reason than we're not particularly good at portraying things beyond
ourselves in any closeup detail. It's easier to believe it's not a man
in a rubber suit if the behemoth is only seen at distance, from the
corner of the eye, or, as I would prefer it for most portions of the
VS time-line, evidenced primarily by what they have long since
wrought. This likewise removes many burdens of contemplating what
godlike beings would do interacting with the likes of us, or even our
post-human successors -- with the action in the past, we can leave
many motives enigmatic, and focus on how characters more like
ourselves must deal with a reality of consequences, and not the
possibilities of dead gods.

\subsubsection{Gods and Titans}

The Greek myths concerning the conflict between the Gods and Titans
are a good example of intergenerational conflicts, and there are some
easy parallels with certain events in the VSU's history. The titan
Cronus fears being supplanted by his children and attempts to destroy
them. In the end, his children do supplant him, just as he had
castrated his own father and supplanted him. There is room for
sympathies with both sides: an older generation who lingers too long
can stifle their erstwhile successors, and a younger generation too
impatient in their ambitions can deny the previous generation the
fullness of their potential existence.

The Ancient/TWHON conflict draws to some degree from this mythical
conflict, but also draws heavily from a reading of the story of
Abraham and Isaac that posits Abraham as being necessarily conflicted
by the order to sacrifice his child.

\subsubsection{Abraham and Isaac}

The conflict between the Ancients and the TWHON draws somewhat from
the biblical story of Abraham and Isaac and later analysis thereof,
such as Kierkegaard's \emph{Fear and
Trembling}~\cite{KierkegaardFearandTrembling}. Here, however, in an
extremely important divergence, the TWHON are both Abraham and
god. Wherein profound human indecision might result in grinding of
teeth or pulling of hair, the mental division within the singular
TWHON mind over how to proceed with those Ancient groups which had
come to resemble more their children than their experiments had
consequences proportional to the scale of the TWHON.

In the VSU, this tale relates less in terms of faith, as the binding
of Isaac is usually viewed, than in terms of the context of
intergenerational power and trust. It shows, among other things, how
guardianship can readily be abused by those willing to sacrifice their
own children, or, more figuratively, their own children's future. This
is not only seen in the Ancient/TWHON conflict, but is also repeated
time and time again when short-sighted actions are taken that benefit
the current generation at the expense of those to come.

\subsection{Trans-humanism}

The people of the future will, at some point, cease to be people as we
know them. While we tend to think of the {\em uncanny valley} in terms
of approximations of humanity as they approach us but are not us, I am
confident that a similar reaction will occur as we gaze upon people
who once were as ourselves, but no longer are. It is no accident that
those groups in the VSU capable of making viable forms of governance and economics that have been very problematic here on earth (e.g. the
Andolians and Shapers) are no longer fully human, or not human at all
(e.g. various alien groups).

\subsection{Man (and Alien) against the Environment}

Space is an intrinsically unpleasant environment for terrestrial
lifeforms. This may seem in some ways too similar a theme to {\em Man
against circumstance}, as represented by the inter-generational
baggage of the previous residents of the galaxy, and thus not a good
choice for focus, but the two are actually distinct. This theme also
encompases the often unwittingly futile struggle against the laws
defining the VSU (i.e. physics and such) that impose their own
limitations on the desires of the VSU inhabitants. The non-futile
struggles will be those wherin it is realized that a given path has
been denied by what exists and can exist, but that another avenue
remains possible.

\section{Things we can learn from other games/universes}

\begin{itemize}
\item Fallout~\cite{FalloutRPG}:

Putting an armed grenade in a child's inventory and then watching them
go off to play with their friends and exploding is a vile
act. Shooting unarmed townspeople is not a friendly thing to
do. However, allowing the character to perform fairly vile actions if
it's reasonable to assume that they could perform such actions and the
game-play mechanisms exist to perform such actions is good game
design. The constraints on interaction with others should be limited
to those provided by the game engine, not those provided by any moral
compass other than the player's.

\item Brotherhood of Steel:

The only people who think that excessive swearing makes a game mature
are juvenile. It just makes it vulgar. This is not to say that the
language in a VSU game should be squeaky clean, just context
appropriate. If a character just got shot in the foot, he or she may
have some choice comments to make. Depending on the particular game
version, these may be bleeped out, but the obvious content should be
implied. If, however, a character is negotiating a business deal and
can't go half a sentence without veering Anglo-Saxon, that's probably
just crude, and we should be classier than that.

\item Oblivion:

If you're not in a sandbox game, you shouldn't be forgetting the main
plot-line(s).  Oblivion is a very interesting game in an expansive
world with many, many side quests. However, since the side quests tend
to be completely orthogonal to what's going on in the rest of the plot
progression, it's a little too easy to not be able to see the forest
for the trees.

\item Shadows of Amn (Baldur's Gate 2), Hordes of the Underdark (Neverwinter Nights expansion)

If you have a well defined past, starting at something beyond level
one makes a lot of sense.

\item Mass Effect

\item Freelancer

\item Privateer

\end{itemize}


% LocalWords:  Kardashev UTCS psionics FTL technobabble Skywalker Yossarian Uln
% LocalWords:  Rimward Aera Andolians Deucalion TWHON Mechanists Klk'k statist
% LocalWords:  Andolian Purth LIHW Aerans Aeran Rlaan Shmrn Dgn Saahasayaay VSU
% LocalWords:  Alphans Betans UtCS RPG Replicants Battlestar Moby retconned
% LocalWords:  Galactica wisenheimers anthrophilic PAIs VSU's Dragonball JRPG
% LocalWords:  multiplayer
