\label{chapt:overview}
This chapter, along with Chapter~\ref{chapt:VSUreality} and
Chapter~\ref{chapt:livinginVSU} provide an overview of some of the
high-level aspects of the Vega Strike Universe and the design
philosophies at play. They indicate with broad strokes what forms of
content and approaches to topic matter will be appropriate for the
Vega Strike Universe. This chapter primarily focuses on the history of
the VSU's design, the forces and influences at play in its genesis,
and outlines some of the broader design philosophies and directions
for both the Vega Strike Universe and the games it is designed to
support. More detailed discussions of the nature of the VSU in terms
of physics and social realities are left for later chapters.

\ignore{
The rest of this document is organized as
follows. Chapter~\ref{chapt:timelines} provides an outline of the
entire scope of VSU history from an omniscient viewpoint, and delves
into details of particular historically significant
events. Chapter~\ref{chapt:PCplots} introduces Deucalion, the player
character for {\it Upon the Coldest Sea}, and discusses the
personal-scale plots that he may be come involved in during the UtCS
time period. Chapter~\ref{chapt:portfolios} provides one-stop shopping
for those, such as artists, looking to find all relevant information
concerning a particular group, as well as more sections discussing
particular types of objects, such as vessels, and how they should be
depicted to be in-line with the VSU
canon. Chapter~\ref{chapt:uncategorized} contains a potpourri of
information yet to be re-located, or lacking a more appropriate place
in the document. Appendices~\ref{appendix:Species}, \ref{appendix:Factions}, and \ref{appendix:trailer} are then presented, followed by the~\ref{References:References}, \ref{Glossary:Glossary}, and \ref{Index:Index}.
}

\section{The nature of the Vega Strike Universe}
\label{sec:VSUflavor}
Everyone, it seems, has their own, often deeply held, views as to what
the future will be like. The Vega Strike Universe, as a whole, doesn't
happen to correspond to any of the developers' ideas of what the future
will actually be like. Let's pull no punches - real space travel and
conflict, should it ever occur, will almost certainly be arduously
long, heavily automated, and completely boring to look at without a
lot of magnification and false-color filters. FTL? Probably not
possible. Manned airplane-like fighter-craft? Unlikely - missiles and
probes do most of the potential jobs a lot better, can withstand much
larger material stresses, don't die from radiation sickness, or run
out of food and water if it takes weeks to get to the target - which,
if there's no FTL, it probably will. As one can imagine, the list goes
on.

\subsection{Requirements of the VSU}
What, one may well then ask, does the Vega Strike Universe correspond
to? The genesis of the VSU started with a set of game premises - there
was to be an Elite~\cite{Elite} style game heavily influenced by
Privateer~\cite{Privateer} featuring humans and at least two other
prominent alien species. There was to be an ongoing interstellar war
between the humans and at least one of the prominent alien
groups. Scales of celestial objects and distances were to be
comparable to the known scales of celestial objects. Physics was to be
respected where possible Complicating things greatly was a desire for
the game to be multi-player, thus ensuring that arbitrary dynamic time
dilation would not be a viable game-play mechanic - much of the game
must happen in something approximating ``real-time''.

%%%%%%%%%%%%
%%%%%%%%%%%%



\label{Danny's stuff goes here}



%%%%%%%%%%%%
%%%%%%%%%%%%


\subsection{Genesis of the VSU: dependencies among game, universe, and game-engine}
The VSU, therefore, has to provide an environment in which all of
these prerequisite sets of features can coexist. Serving a particular
type of game and style of game-play, it is constrained in ways that
lead to compromise, both with reality as we know it or expect it to
be, and with the aesthetic and artistic predilections of those looking
to design the universe as they would prefer it to be. Speaking as the
chief architect of the VSU, I can vouchsafe both that the VSU as it
stands diverges noticeably from the sort of fictional universe I would
create absent the particular game constraints, and that I sleep
perfectly well at night having made that compromise. The task I was
given was to oversee the creation of a universe that would allow a
particular type of game powered by the VS engine, an entirely
different beast than the VS Universe, to have, for a target audience
much larger than the project leads or the developers and contributors
themselves, an engaging and entertaining setting. There is no sin in
seeking to entertain, and neither is there shame in targeting an
audience centered far from where one stands oneself. There is a
deliberate scent of popcorn to the requirements the VSU must meet, and
that necessarily colors the universe that serves it, but that in no
way keeps us from aspiring to quality, even if we must resign
ourselves at times to painting soup cans instead of canvas.

\ignore{ Contact with alien species implies FTL. For there to be
something recognizable as an ``ongoing war'' on an interstellar
scale, neither the attacker (borders completely permeable to primary
FTL mechanism) nor the defender (fixed points of FTL transit) could be
unduly advantaged - hence two different mechanisms for FTL. Early
attempts at a unified FTL mechanism (point-point) that differed for
in-system and out-system use only in terms of some sort of presumption
of scalability were eventually retconned due both to difficulties in
reconciling the ability to jump from one side of a system to another
with the necessary drawbacks for interstellar travel that this same
system must still possess. While the early point-point mechanism was
widely criticized for being too fast, its replacement (SPEC) was
widely criticized for being too slow, except by those who continued to
argue that hours, rather than minutes, should be the bare minimum for
crossing a star system.  }


\section{Philosophy of scope and goals for plot and protagonist}
\label{sec:plottingphilosophy}

\subsection{Small protagonists in a big universe}
Many people enjoy pretending to be heros (or villains), but I believe
that there exists a great and pervasive confusion that conflates
heroism with actions on a scale of epic myth. There are a number of
classical archetypes for heroes and their journeys. One can wander
down the Freudian/Jungian pages of Campbell's {\em Hero with a 1000
Faces}~\cite{herowith1000faces} and see everyone from Siddhartha to
Luke Skywalker.  That's not quite the sort of protagonist we're
actually going to be looking for in most interactive content set in
the VSU, at least in terms of player characters. This is because,
certainly for UtCS and likewise for most other planned tales, we're
not looking to craft a mythic journey centered on the player
character. A mythic aspect would be much more appropriate for the {\it
Rimward of Eden} setting, but even there, restraint is warranted. We
wish to avoid having the player character become the, witting or
unwitting, linchpin of all further history. We aim for epic scope in
setting, not in the degree of impact that the player characters
actions will have upon future generations to come. We wish to allow
opportunities for heroism, but on a personal scale, not a galactic or
cosmic one.

We wish the setting to be epic, and the player sized on a human,
rather than comic-book superhero scale. Compelling examples of the
personal heroic within an epic setting are not hard to come by in
either reality or fiction. One can think of Sgt. Alvin York's
exploits~\cite{SergeantYork} in the context of the first world war, or
any number of characters from various world war two epics. One can
play a very active role, both helping to effect and being affected by
the changes occurring in a turbulent period of history, without being
the primary determinant of outcomes. A VSU protagonist should not be a
Gordon Freeman~\cite{Half-Life-GordonFreeman} stand-in. They will not
be, from the perspective of the history books, the most important
person of the era. They will, however, be among the most important
people in their own lives and in the lives of those close to them.

We should avoid big stories, epic stories of sweeping scope. Merely,
there appears to be frequent confusion that, in order to tell the tale
of Moby Dick~\cite{MobyDick} the player somehow must have to be either
Ahab or the whale. This is why, for the purposes of story development
for VS, I want to distinguish between the big plot and the little
plots. The big plot is an epic canvas, a great, boldly painted
backdrop with thick lines and firm colors against which and within
which smaller events are set, contrasted, and constrained. The big
plot should concern itself with what would be written in history texts
after its conclusion, whereas the little plot should concern itself
with what its surviving characters would tell their grandchildren
about what they were doing during some chapter in the aforementioned
history text. The epic actions should remain mostly constrained to
actions of sufficiently sized groups, those who command such groups,
intelligible chance happening, and other such mechanisms as are
required to create the epic sweep of time and space beyond the ken of
small beings. This is not to say that players in the little plots
cannot affect the events of the big plot, it is rather they need not
do so, and are likely to only have extraordinary effect when they have
performed extraordinary action. The converse, of course, is not true -
actions that occur in the big plot can clearly have profound and
immediate or delayed and subtle effects upon the player - the fall of
a government will obviously change a player's experience if they were
to go visit that region, and the stresses of a war economy should
become apparent in various pricings, patrol levels, availability of
side tasks, etc. To rein things back a bit from the general to the
specific, UTCS will, at root, be the story of one man's life during
wartime, seen through his eyes,

\subsection{Appropriate player-level plots}

When it comes to specific plots, I'm not interested in crafting a
thinly disguised morality play, as many Star Trek episodes often
amounted to. I'd prefer to limit the degree to which things devolve
into an expression of teenage fantasies of godlike empowerment -
Dragonball Z is NOT what I would consider a good starting point for
much of anything, even ignoring the sequel strangling power growth
whereby planets are being absentmindedly smashed. Likewise, the
standard Square RPG wherein the young weakling levels up repeatedly
until he becomes a veritable force of nature that the course of all
history depends upon is NOT a desirable goal. Even the more subdued
Freelancer variant thereof is something to be avoided - especially the
standard "enemies keep getting more powerful just to match the
player's progress despite the fact that this means that later stage
enemies could slaughter entire civilizations from earlier in the game"
and "fate of humanity rests upon the player" problems. Privateer
levels of player significance are probably as far as we should
consider treading, and, where possible for VS, I'd like to move the
bulk of the significance to before the player takes control of the
character. I find it preferable to have a character intrinsically with
some significance and then let the player do with that what they will
rather than to force a player down a path that will cause them to
become significant - I think it's much more interesting to have, as a
purely hypothetical and unrelated-to-VS example, a scenario where a
powerful entity decides to spend its time surfing, gambling and sowing
its wild oats instead of saving the human race because it just didn't
feel the motivation or didn't want to risk it than to yank the player
incongruously along every time they start to wander down a path that
doesn't take them toward becoming some "powerful entity". Cosmic
significance for an individual is really difficult to not screw up in
SF, less so perhaps in fantasy where there are easier outlets for
suspension of disbelief compatible with the base universe. Even if the
rest of this paragraph falls out of your heads from my rambling text,
here's the point I want to make clearest - cosmic significance is not
necessary for compelling drama or even more generally, good stories,
drama or otherwise. In Catch-22, Yossarian never strikes a deathblow
against the Nazis. In All Quiet on the Western Front the protagonist
accomplishes little except surviving only to die, and if that's not a
compelling story for being too far from the SF vein then ponder Blade
Runner - the world is just as screwed up in the end as in the
beginning, but one is consumed by the story of even the few people
(and Replicants - not all interesting characters need be truly human)
involved. In a more recent pop setting, Law and Order doesn't feature
gods and monsters, but has audiences so hooked they keep spinning off
companion series. Even looking at the new Battlestar Galactica series
(the first season, at least ;-) )shows the strength that can come from
focusing on human interactions and frailties even in a dire and
cosmically important situation. In short, it is sufficient for the
story the protagonist is directly entangled in to be important to the
protagonist and those around him/her.

\subsection{Moral certainty, prerequisite idiocy, and other things to avoid} 

As for characters themselves, let the fools be fools and the sages,
sages. However, no matter how cynical one may be, constructing plots
that require the vast majority of entities involved to be either
blithering idiots, willfully ignorant, or (even worse)
schizophrenically swinging between brilliance and incompetence, is
poor craftsmanship - although there is something to be said for
wandering down a cliche path and then twisting it at the end to inform
the audience that you were aware of the cliches - Alan Moore's
{\emph Watchmen}~\cite{Moore-Watchmen} has some good examples of such twists. Hinging a plot on
some important thing not being known isn't a bad thing, but hinging a
plot on an otherwise perfect plan crafted by keenly intelligent
planners having a "fatal flaw" that allows "good" to triumph over
"evil" is just plain old fashioned bad storytelling and magical
thinking at that. Let me be clear on this - there is nothing good or
evil in VS, but the thinking of some group therein terms it so (my
apologies to the Bard).  Good and Evil is good fodder for children,
but, at the risk of alienating the vast potential audience of the
unnuanced, to the degree possible and appropriate (stories in the
Rimward of Eden setting, for instance would be reasonable places for
"coming of age" themes, even if it's an Aera coming of age), VS should
deal with the more complicated, and frankly more interesting, problems
of adults (to be honest, everyone who's been a teenager already did
the teen angst thing -there's no reason to keep reveling in it, even
if many of us can't always keep from doing so). If we manage to get
ourselves accused of moral ambiguity, of an uncomfortable gap between
actions noble and actions necessary, we're probably on the right track
(consider, for instance, the Andolians - their work with the AI Quorum
in creating the Grandchildren will help ensure the continued existence
(and local importance) of humanity, but they're also responsible for
billions of cold, calculated human deaths during the Fraternal war,
etc.).

\subsection{Blank slate vs. pre-determined protagonists}

The VS universe has strong existentialist influences - such is (my)
life and art does tend to imitate life J. However, that is not to say
that the goal is for VS to become an emo angst-fest of ennui. At the
beginning of UTCS, Deucalion is a bit of an emotional mess, but this
is to be expected. He's just gone through a traumatic event that
nearly killed him and did kill his best friend and (for
simplification) brother-in-law. He's rather a bit shaken, his life
plans have just been rudely interrupted, and, to add insult to injury,
the Aera have just started invading Forsaken space while Deucalion was
recovering from his injuries. It's a situation that is hobbled with
guilt and the impotence of any individual against the uncaring and
unnoticing motions of the universe. It's a situation ripe for
catharsis and, just as importantly, for life altering change - it is
therefore a good starting point for a player to take over a character
who already has a defined past. They can take the character in rather
different directions without it breaking all suspension of disbelief,
as "he cracked" and "you're not the same since XXX" are easily
intelligible human reactions to tragedy. At the same time, they can
use the existing past and the benefits and limitations it has bestowed
upon the character as either guide or leverage - his past gives him
skills that make him useful, hence a target for interaction with other
entities, and it limits his direct involvement with governance or
military forces. As he's already left the Protectorate navy, he can
much more believably stray further, or wander back - the choice is
his, and thus, the choice is the player's.

There is always the issue in an open-ended game of when things end -
where is the resolution? For UTCS, the big plot will advance whether
or not the player does too much. If one only wants to see what happens
in the big picture, one need only play a bit, and then play a
(safety-seeking) waiting game - in doing so, one has played out a
rather boring life, but it's a player's choice, and, if they aren't
actively playing, then they won't get to see the effects.  For UTCS, I
think having a number of soft-endings for a set of little plots
(hence, the plurality) is the best choice. Profession/lifestyle
specific plots can have an apex or plateau if not a conclusion. The
closest game play that comes to mind is the guild-leveling in
TES:Oblivion, but the pace of advancement in the Mage's guild seemed
less than proportionate to the tasks at hand, very gamish, but I
digress.

\section{Themes in Vega Strike}
\label{sec:VSthemes}
There are certain themes which are either introduced directly via the
existing VSU history and heavily featured, or that we feel would
be particularly poignant to explore in the VSU setting.

\subsection{Generational conflict and impositions}
One recurring theme that is present in the overarching VS universe
story (although echoes of it will be seen in the player stories as
well) is that of intergenerational relations, of parents and their
children, in this case, the figurative species-children of various
groups and the effects that the actions of those who came before them
have on their own development. Our parents are our first gods, and
even if they succeed as parents, they will always fail us as gods. I
thought it would be particularly interesting then, to start the story
off with a parental group that really was, in many ways, godlike - but
still failed, and in their greatness, had their failure cast a
commensurately large continuing shadow over those who came after them.

VS, however, as much as the TWHON may impact the events that unfold,
should not be seen as a story about them. Their day, and indeed, their
greatness, and even their existence as anything but vague and warped
shadow of what they once were, has long passed. While it is worthwhile
to understand their place in the story, I would caution giving them
too lavish a portion of our attentions - gods may seem interesting,
but stories about people are much involving in the long term, if for
no other reason than we're not particularly good at portraying things
beyond ourselves in any closeup detail. It's easier to believe it's
not a man in a rubber suit if the behemoth is only seen at distance,
from the corner of the eye, or, as I would prefer it for most portions
of the VS time-line, evidenced primarily by what they have long since
wrought. This likewise removes many burdens of contemplating what
godlike beings would do interacting with the likes of us, or even our
post-human successors -- with the action in the past, we can leave
many motives mysterious, and focus on how characters more like
ourselves must deal with a reality of consequences, and not the
possibilities of dead gods.

\subsection{Transhumanism}

The people of the future will, at some point, cease to be people as we
know them. While we tend to think of the {\em uncanny valley} in terms
of approximations of humanity as they approach us but are not us, I am
confident that a similar reaction will occur as we gaze upon people
who once were as ourselves, but no longer are.

\subsection{Man (and Alien) against the Environment}

Space is an intrinsically unpleasant environment for terrestrial
lifeforms. This may seem in some ways too similar a theme to {\em Man
against circumstance}, as represented by the inter-generational
baggage of the previous residents of the galaxy, and thus not a good
choice for focus, but the two are actually distinct.

% LocalWords:  Kardashev UTCS psionics FTL technobabble Skywalker Yossarian Uln
% LocalWords:  Rimward Aera Andolians Deucalion TWHON Mechanists Klk'k statist
% LocalWords:  Andolian Purth LIHW Aerans Aeran Rlaan Shmrn Dgn Saahasayaay VSU
% LocalWords:  Alphans Betans UtCS RPG Replicants Battlestar Moby retconned
% LocalWords:  Galactica wisenheimers anthrophilic PAIs
