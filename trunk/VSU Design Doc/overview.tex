\label{chapt:overview}
This chapter provides an overview of some of the high-level aspects of
the Vega Strike Universe and the design philosophies at play. It
indicates with broad strokes what forms of content and approaches to
topic matter will be appropriate for the Vega Strike Universe. Section
~\ref{sec:thingsseeninVSU} introduces a number of things which one
should expect to see in any finalized (though not necessarily interim)
depiction of the VSU, while Section~\ref{sec:thingsnotinVSU} notes
categories explicitly {\bf not} suitable for being present in the VS
universe. Section~\ref{sec:VSphysics} outlines the key assumptions of
VSU physics, notes divergences from likely physical realities, and
briefly delves into some of the ways in which VSU physics will
manifest in gameplay. Section~\ref{sec:plottingphilosophy} discusses
our position on the role of the player with respect to the progress of
galactic events, and the resultant bifurcation of plot into the
galactic-scale plot (as per the timelines in
Chapter~\ref{chapt:timelines}) and the personal plot segments that
will directly involve the player character. Section~\ref{sec:VSthemes}
introduces common themes that will be appearing in the VS Universe and
galactic-scale time/plot-lines, and may likewise be echoed by the
smaller player-scale plots. Section~\ref{sec:groupintuitions}
concludes the overview chapter with some advice geared toward
internalizing intuitions for various human and alien groups present in
the VSU during the UtCS period.

\section{Things that should (eventually) be seen in the VS universe}
\label{sec:thingsseeninVSU}
\begin{itemize}
\item FIXME
\end{itemize}

\section{Things that should not / will not be seen in the VS universe}
\label{sec:thingsnotinVSU}
\begin{itemize}

\item Interventionist deities or agents thereof

Whether or not a divine being or beings exist in the VS universe is
irrelevant. Neither evidence nor action by any such entities occur in
the VS universe.

\item Interventionist deities or agents thereof (pretending to be aliens)

The term ``god-like'' could easily be applied to any type II or type
III civilizations (Kardashev scale~\cite{Kardashev}), or even members
thereof. This does not make them actual gods. Nor should they be used
as stand-ins for traditional human deity-figures, redemption figures,
moral archetypes and so forth. That they are profoundly powerful does
not make them well intentioned, infallible, or even necessarily
wise. That they are ``god-like'' in power merely denotes them as
powerful enough that they possess a potential for actions at greater
scales. It is worth noting, if nothing else, the number of powers
previously reserved for gods that even our own civilization (not yet
even a type I civilization) has already secured for itself without attaining
any shred of divinity.

\item Absolute Morality, embodiments thereof, and anthro-exclusive destiny

There is nothing intrinsically {\em good} or {\em evil} in the VS
universe, be it an action or an entity. More importantly, there is no
external guidance toward ``the good'' or away from ``the evil'' that
acts as some imperative societal force, nor a presumption that what we
consider elements of ``the good'' will prove to yield higher
survivability than aspects our modern societies do not deign to
associate with goodness. While it is certainly more difficult to see
some actions as beneficial than others, as there is no presumption of
an external calibrating entity in the VS universe, such judgments are
societal, and their merit only capable of being judged properly long
after any associated actions have been taken. Additionally, the
continued existence of humanity should not be blithely assumed to be a
universal ``good''. While humanity and their descendants will play an
important role for a certain time-period in the VS universe, this is
happenstance, and not providence. Indeed, much of what happens after
UTCS is in many ways a decline of humanity and the supplanting of
humanity by its children.

\item Magic (as magic)

The following, among other things not listed, do not exist (at least
as people currently describe them) in the VS universe: ESP, psionics,
talking to the dead, spiritual possession, auras, telepathy,
telekinesis, purity of essence, and clairvoyance. Events in the VS
universe will not be determined, nor even affected, by `fate', gods,
prophecy, or other elements of the supernatural.

\item Magic (pretending to be technology)

While Clarke's law posits that any technology, sufficiently advanced,
is indistinguishable from magic, the converse is decidedly {\bf not}
true in the VS universe. Just because we can think of some magical
means for doing something does not mean that any technological
implement, no matter how advanced, can ever achieve the same
effect. This is not to be confused with things such as fusion reactors
which we have firm understandings for how, in general principle, one
might work, but cannot fathom how to build a practical one. Allowing
leeway for overcoming implementation difficulties, or even profound
knowledge gaps is a different beast altogether than positing something
to exist which we can already know to be in violation of numerous
aspects of our model of reality. This means no perpetual motion
machines, no instantaneous galactic communication (although we do
still regrettably violate causality with non-instantaneous FTL), no
living-super-armor (it's just a fad fuelled by, one presumes, a poor
understanding of high-tech carbon composite materials and the sorts of
metabolisms needed to produce such on the fly.), no life-force, no
ascension, and no ``energy-beings'' without a concrete definition of
the sorts of energies involved (i.e. ``being existing as patterns in
EM waves'' is OK, but ``ascending to pure energy'' is just pandering
to a magical reality, all apologies to Stargate, Babylon 5, etc.). We
must strive to differentiate the unlikely (even perhaps sometimes
coming close to the borders of implausibility), which is fertile
ground for science fiction (e.g. room-temperature superconductors)
from the truly magical masquerading in the guise of science
(e.g. though I do enjoy the series, most Doctor Who episodes are
fantasy, not science fiction) and from fundamentally unsound
propositions passing themselves off as profound knowledge or advanced
technology (e.g. splitting the beer atom, gaining mutant powers from
gamma radiation, almost anything from the movie {\em What the \#\$*! Do
We (K)now!?}, entities that gain mass without consuming anything,
re-spinning the earth's core with a hydrogen bomb, neutrino weapons,
hacking alien mother ships with a laptop, instantaneous evolutionary
accelerators, ``polarity reversal'' as the answer to everything, and
claims that ``humans only use 3 percent of their brains'').

While the VS universe allows some liberties with known or
expected physics for the sake of plot and game-play possibilities, we
are bound by the laws of physics unless explicitly relieved from said
limitations. Deviations from our base physical reality have a nasty
habit of causing cascading implications that aren't desirable, and
should only be undertaken with both good cause and firm caution. 

We make explicit and limited exceptions for FTL, shields, and other
related space-warping technologies which we bin under
``gravitics''. Gravitics is known to be junk-science, but, as magical
additions to reality go, can be presented in a reasonably
self-contained fashion as long as nobody stares too long and hard at
it. FTL is, for better or worse, necessary, and shields are, if not
necessary, sufficiently desirable and expected that we've made room
for them.

Toward the end of limiting the sickly spread of junk science and
outright silliness into every aspect of VS, we should avoid
technobabble like the plague that it is. While it's fine, in documents
such as this, to ponder to some degree how some of the VS tech is
presumed to work in order that we can coherently and consistently
describe the properties of objects using said tech, the user should be
heavily insulated from such discussions. Firstly, what the user needs
to know are the end effects (e.g. does it pierce shields, how does the
range fall off, how much energy per second does this reactor provide,
etc.) and what the useful inferences they can make are (e.g. this is
a ``shield-type'' weapon, other ``shield-type'' weapons will have
similar properties) more so than any nitty-gritty made-up details of
how tech we don't actually know how to build works. Secondly, if we
attempt to explain how something no one knows how to build works,
we're almost certainly going to come off sounding like either outright
fools (for using real terminology incorrectly) or as purveyors of
gobbledygook (for using unreal terms randomly). Star Trek is the
perfect example of what {\bf NOT} to emulate here. There will be no
``transferring of power from the rear neutrino phase shift emitters
to synchronize our tachyon-positron field into a stable
co-phase matrix''. In the long term, it will only be held against us.

\end{itemize}

\section{Physics and Technology in the VS Universe}
\label{sec:VSphysics}
\subsection{On Weaponry, Defense, and Damage in the VS universe}

So, assuming one has the ability to diddle with the surrounding space
(leaving discussion of whether this, or any other stated principle,
was/could be a good choice for a fundamental assumption to another
time) how might one construct a shield?  Well, I thought perhaps one
could set up something based around gravitic shear forces (locally
violent, but, with opposing forces mostly canceling each other out at
greater distance due to super-linear falloff).

I then figured it would probably be worthwhile to augment such a setup
with an EM component, so as to assist against charged particles, as
charged particles are easy to accelerate, and therefore a likely
choice in assorted weapons systems. So, when descriptions (minimal as
they were) were written for shields, they were referred to as
providing a combination of gravitic and electro-magnetic protection.

Now, where did this lead me (at least as far as I saw it) - almost
everything except for something that looks like a shield should
penetrate a shield in some manner to some degree.

(a brief aside: ship collisions are somewhat outside the scope of this
post - suffice it to say that they should be much more catastrophic
than they are, but the reason is not related to shields - it's that
our damage model only works on energy right now, and doesn't look at
time related components, so if a ship smacks into something at 300m/s
and bounces off at 100m/s in the opposite direction we apply damage
due to the loss of kinetic energy, but don't currently address the
problem that, if this collision took 1/10 of a second, the ship
experienced an acceleration of 400g's, the pilot should be paste (even
assuming some (limited) means of inertial compensation as a cheap way
to warp space may be deemed to provide), and the ship should be
assorted bits of fine debris - this is a bug, a feature failure in
need of fixing. We don't have a model for acceleration tolerance,
clearly, we need one.)

Shield effects, by category:
\begin{itemize}
\item LASERS and other coherent EM radiation 

hard to get a beam of light to interact strongly with this setup at
all (unless one assumes that photons passing through the distorted
topology can be convinced to dump energy and shift down the frequency
spectrum in return for degrading the desired topology - but the more
that I've thought about that, the less it appeals to me, so let's not
spend much time there) but it might interact weakly, de-focusing the
beam. For low frequency radiation, de-focusing is going to be quite
detrimental (in terms of the likelihood of armor being capable of
dealing with incoming beam) but one imagines that xasers and grasers
are still going to be quite damaging even if the incoming beam is
distorted and defocused. Hence, at best, fair protection against low
end laser weapons, to negligible protection against high-end laser
weapons. This translucency (not transparency) has the benefit of making
it easier to explain how EM spectrum sensor data gets in, but causes
some problems with pilot-line-of-sight (upon further reflection, I've
come to the opinion that chuck raised an excellent point with respect
to his comment about the insistence of early astronauts on capsule
windows - there are only two major human groups in the VS universe
with pilots that likely wouldn't demand the same, if not windows per
se, then some semi-direct optical access (I also briefly, and not in
particular seriousness, pondered the notion of an "optical fuse" )-
but this delves into a whole other train of thought, so I'll stop it
here for now.)

\item Solid objects

should interact fairly strongly with the shear forces. Complex objects
could end up giving up non-negligible amounts of energy in undergoing
deformation or otherwise smacking into bits of themselves. However,
given high initial velocity, sizable portions of the incoming
remnants of the object will not be sufficiently diverted and will
still intercept the target. This is still a preferable scenario, as a
defocused impact of something more resembling dust and shrapnel should
be a lot easier for armor to handle than an intact shell. (Unless of
course, one doesn't have armor, in which case one may have just traded
one set of holes for many sets of holes.)

\item Particle beams

A) Charged - high velocity makes them hard to divert with the
gravitics, again just gaining a defocusing, but that's what the EM
systems are there for helping out with. Still, in the end it's just a
very good defocusing and diverting, and can't be expected to stop all
the incoming particles completely.

B) Neutralized - EM field doesn't help in any meaningful way,
defocused, more-so than a laser, but protection is pretty poor, and
it's mostly up to the armor.

\item Plasma 

A)Net-neutral, or B) net-charged clouds of high temperature ionized
particles that are likely to be fairly effectively diverted by an EM
field unless the plasma density was quite high at the time of
interaction (still efficiently diverted in such a case, but perhaps
not effectively).

\item Shields and shield-like weapons

Directly act upon the topology created by the shields, significantly
degrading them. However the directness of their interaction also means
that their effects do not penetrate the shields.

How I saw this playing out in terms of game mechanics: 

Firstly, as shields degraded (topology becoming unstructured, shear
forces going away), anything that penetrates a shield already would
penetrate more. The EM field wouldn't degrade in the same manner as
the space-warping component, but it was only useful in mitigating
charged particles anyway.

\end{itemize}

Weapons, by category:
\begin{itemize}
\item Lasers

would seem to be quite nasty beasts in that they mostly ignore
shields, especially at higher frequencies, except that lasers have
lousy energy efficiency, especially at higher frequencies, and
especially given that laser inefficiency tends to materialize as waste
heat. Thus I saw lasers as weapons with extreme cooling problems,
either resorting to open cycle cooling (venting coolant = limited
ammo, limited re-fire rate) or {\em very} slow re-fire rates (also a source
of perhaps interesting complexity if/when any form of heat modeling
gets implemented). Likewise, the higher frequency lasers would be
prohibitively expensive and potentially bulky beasts, probably not
found in small craft. Additionally, as they don't interact strongly
with shields, they wouldn't be good weapons for degrading them
rapidly. Range would be good though,

(lasers don't degrade as the inverse square, but diffract according to something along the lines of 

RT = 0.61 * D * L / RL 
where: 
RT = beam radius at target (m) 
D = distance from laser emitter to target (m) 
L = wavelength of laser beam (m) 
RL = radius of laser lens or reflector (m) 
) 

\item Solid objects 

Lower energy requirements (could also have internal energy sources, as
per rockets), easier cooling solutions, good rates of fire, degraded
by shields but degrade shields, and become increasingly effective as
the shield degrades. Limited ammunition. Can be augmented (at
increased size/cost) by addition of shielding, and/or nuclear or
anti-matter warheads. At the (expected relative) velocity these would
be impacting at, conventional explosives would not be useful
additions. Damage does not decrease with range (although for reasons
of limited processing power, a "max range" still needs to be specified
engine level).

\item Particle beams

A)Charged - low yield electron beams can already be made with very
high efficiency - but cranking up the power will drop the efficiency a
lot. More importantly, any charged particle beam suffers from severe
thermal and electro-static bloom. The constant on the super-linear (I
believe it's actually an inverse-square) decay in beam density can be
helped by using more massive particles, or accelerating to
relativistic velocities for the sake of time dilation, but at the
expense of efficiency (significant relativistic velocities are a
{\em huge} energy investment, neutrons are dead weight to an EM
accelerator, and only so many electrons can be conveniently added to
or removed from an atom). To make matters worse, one's ship will
accumulate net charge if repeatedly firing a charged beam, unless the
excess charge is bled off somehow (I've seen indications that
alternating between positive and negatively charged firings is a "bad
idea (tm)" due to creating a current loop involving the vessel). So,
to sum up, the range is pretty bad, the efficiency is questionable,
there's probably a hell of a re-fire delay as one cleans up the charge
accumulation problem, and EM fields can do a lot to defocus the
incoming beam. However, if you are close enough, and your particle
density is high enough, then what does get through would do nasty
things to armor, surface mounted electronics, and throw off lots of
secondary radiation.  

B) Neutralized - (and by neutralized I don't mean "neutron beams",
because I haven't the foggiest idea how to generate or accelerate them
effectively in anything resembling a coherent beam unless we start
talking about space-warping that is probably powerful enough that'd
we'd have to go back and revisit the whole "can't do to much to
photons" issue which I'd rather not, and besides, that would probably
mean that shields were impervious to just about anything... which is
rather much not the goal either) more specifically, a beam of
particles that has been rendered charge neutral; one in which
oppositely charged particles (likely electrons) are added back in
after acceleration (both must have been accelerated) to neutralize the
beam. This will almost certainly defocus the beam, and again almost
certainly drop efficiency even lower. However, it avoids the local
charge accumulation problem, this removes electrostatic bloom, leaving
only thermal bloom, increasing range, and it also negates the
effectiveness of EM fields to disperse the beam at the
target. However, it also negates the current and charge accumulation
effects on the target that might damage electronics. Still, plenty
unpleasant on impact, only mildly affected by shields, but range isn't
as good compared to lasers, and efficiency is only questionably
better, and could easily raise similar cooling/re-fire issues.

So, as for beams - mediocre range due to bloom effects, efficiency
questionable, neutralized beams achieve good penetration against
shields at cost of even lower efficiency, charged beams have lousy
penetration against shields, but can probably be used in efforts to
disable the target's electronics (at the least, those present on the
surface, or accessible by necessity (engine/reactor) - the core
protected elements are going to have to be in some Faraday cages with
optical links to the externals (optical links don't like shear forces
though, so they could break with some probability upon impact or
impact resembling damage). Ammunition (the particles in question)
necessary, but in sufficiently small quantities per firing that it can
either be ignored or modeled as extremely cheap, small, and
plentiful. Some noticeable degradation of shields due to some
interaction.


\item Plasma 

Last I investigated, unless there's some way to make plasma somehow
generate its own magnetic fields of exceptionally interesting (read:
somewhat absurd) strength, or one wants to accelerate the plasma to
very high velocity (which would start to look something more like a
shorter pulsed version of the the beams above), it's not going to be
an effective weapon at anything beyond the shortest of ranges, because
it expands like no one's business (our dear friend the inverse square
law, but with indications of unforgiving constants, the prevalence of
plasma weapons in many sci-fi works notwithstanding) and in every
direction. High-tech flamethrowers with interesting electrical
properties are cool, but not very effective unless one is close enough
to read the serial numbers on the target's fuzzy dice, never minding
the effects of EM fields on ions, which further limits effectiveness.

In short, one could build the bolt (short pulse) rather than beam
version of a particle beam, and it would be rather similar to the
particle beams, and not what one traditionally calls a plasma
weapon. Or, one could build a reasonably efficient plasma weapon, but
be limited by rapid falloff to the shortest of ranges. Ammo for plasma
weapons should be in the dirt cheap, small, and exceptionally
plentiful category. If you're actually close enough to get any
reasonable number of particles past the EM fields, you'll do nasty
things to the electronics, and you can probably afford to keep firing
for a while. Shield degradation can be somewhat more pronounced than
particle beams if more matter is being thrown at the target.

\item Shields-and shield based weapons

Ammo, none. Shield penetration, none. Efficiency, mediocre-poor, hence
re-fire, fair-slow. Target shield degradation better than any other
damage source. Transmitted damage after shield collapse (topology
unstructured) worse than any other damage source, but non-zero. Damage
vs. unshielded objects significant.

\item Missiles

Mostly depends on warhead type. Shielded kinetic is one option, single
shot weapons of various types also options, as are bomb pumped lasers
or simple nukes. Ultra-low-yield (0.5 - 1 ton range) fusion warheads
are presumed commonly available (preferable to chemical explosives due
to the manner of transmission of the energy, namely, high frequency
radiation and neutrons).
\end{itemize}

\section{Philosophy of scope and goals for plot and protagonist}
\label{sec:plottingphilosophy}
There are a number of classical archetypes for heroes and their
journeys. One can wander down the Freudian/Jungian pages of Campbell's
{\em Hero with a 1000 Faces} and see everyone from Siddhartha to Luke
Skywalker.  That's not the sort of hero we're looking for because that
isn't the sort of story we're looking to tell. I'm not interested in
crafting a thinly disguised morality play. I'd prefer to limit the
degree to which things devolve into an expression of teenage fantasies
of godlike empowerment - Dragonball Z is NOT what I would consider a
good starting point for much of anything, even ignoring the sequel
strangling power growth whereby planets are being absentmindedly
smashed. Likewise, the standard Square RPG wherein the young weakling
levels up repeatedly until he becomes a veritable force of nature that
the course of all history depends upon is NOT a desirable goal. Even
the more subdued Freelancer variant thereof is something to be avoided
- especially the standard "enemies keep getting more powerful just to
match the player's progress despite the fact that this means that
later stage enemies could slaughter entire civilizations from earlier
in the game" and "fate of humanity rests upon the player"
problems. Privateer levels of player significance are probably as far
as we should consider treading, and, where possible for VS, I'd like
to move the bulk of the significance to before the player takes
control of the character. I find it preferable to have a character
intrinsically with some significance and then let the player do with
that what they will rather than to force a player down a path that
will cause them to become significant - I think it's much more
interesting to have, as a purely hypothetical and unrelated-to-VS
example, a scenario where a powerful entity decides to spend its time
surfing, gambling and sowing its wild oats instead of saving the human
race because it just didn't feel the motivation or didn't want to risk
it than to yank the player incongruously along every time they start
to wander down a path that doesn't take them toward becoming some
"powerful entity". Cosmic significance for an individual is really
difficult to not screw up in SF, less so perhaps in fantasy where
there are easier outlets for suspension of disbelief compatible with
the base universe. Even if the rest of this paragraph falls out of
your heads from my rambling text, here's the point I want to make
clearest - cosmic significance is not necessary for compelling drama
or even more generally, good stories, drama or otherwise. In Catch-22,
Yossarian never strikes a deathblow against the Nazis. In All Quiet on
the Western Front the protagonist accomplishes little except surviving
only to die, and if that's not a compelling story for being too far
from the SF vein then ponder Blade Runner - the world is just as
screwed up in the end as in the beginning, but one is consumed by the
story of even the few people (and replicants - not all interesting
characters need be truly human) involved. In a more recent pop
setting, Law and Order doesn't feature gods and monsters, but has
audiences so hooked they keep spinning off companion series. Even
looking at the new Battlestar Galactica series (the first season, at
least ;-) )shows the strength that can come from focusing on human
interactions and frailties even in a dire and cosmically important
situation. In short, it is sufficient for the story the protagonist is
directly entangled in to be important to the protagonist and those
around him/her.

This is not to say that we should avoid big stories, epic stories of
sweeping scope. Merely, there appears to be frequent confusion that,
in order to tell the tale of Moby Dick the player somehow must have to
be either Ahab or the Whale! This is why, for the purposes of story
development for VS, I want to distinguish between the big plot and the
little plots. The big plot is an epic canvas, a great, boldly painted
backdrop with thick lines and firm colors against which and within
which smaller events are set, contrasted, and constrained. The big
plot should concern itself with what would be written in history texts
after its conclusion, whereas the little plot should concern itself
with what its surviving characters would tell their grandchildren
about what they were doing during some chapter in the aforementioned
history text. The epic actions should remain mostly constrained to
actions of sufficiently sized groups, those who command such groups,
intelligible chance happening, and other such mechanisms as are
required to create the epic sweep of time and space beyond the ken of
small beings. This is not to say that players in the little plots
cannot affect the events of the big plot, it is rather they need not
do so, and are likely to only have extraordinary effect when they have
performed extraordinary action. The converse, of course, is not true -
actions that occur in the big plot can clearly have profound and
immediate or delayed and subtle effects upon the player - the fall of
a government will obviously change a player's experience if they were
to go visit that region, and the stresses of a war economy should
become apparent in various pricings, patrol levels, availability of
side tasks, etc. To rein things back a bit from the general to the
specific, UTCS will, at root, be the story of one man's life during
wartime, seen through his eyes,

As for characters themselves, let the fools be fools and the sages,
sages. However, no matter how cynical one may be, constructing plots
that require the vast majority of entities involved to be either
blithering idiots, willfully ignorant, or (even worse)
schizophrenically swinging between brilliance and incompetence, is
poor craftsmanship - although there is something to be said for
wandering down a cliche path and then twisting it at the end to inform
the audience that you were aware of the cliches - Alan Moore's Watchmen
has some great examples of such twists. Hinging a plot on some
important thing not being known isn't a bad thing, but hinging a plot
on an otherwise perfect plan crafted by keenly intelligent planners
having a "fatal flaw" that allows "good" to triumph over "evil" is
just plain old fashioned bad storytelling and magical thinking at
that. Let me be clear on this - there is nothing good or evil in VS,
but the thinking of some group therein terms it so (my apologies to
the Bard).  Good and Evil is good fodder for children, but, at the
risk of alienating the vast potential audience of the unnuanced, to
the degree possible and appropriate (stories in the Rimward of Eden
setting, for instance would be reasonable places for "coming of age"
themes, even if it's an Aera coming of age), VS should deal with the
more complicated, and frankly more interesting, problems of adults (to
be honest, everyone who's been a teenager already did the teen angst
thing -there's no reason to keep reveling in it, even if many of us
can't always keep from doing so). If we manage to get ourselves
accused of moral ambiguity, of an uncomfortable gap between actions
noble and actions necessary, we're probably on the right track
(consider, for instance, the Andolians - their work with the AI Quorum
in creating the Grandchildren will help ensure the continued existence
(and local importance) of humanity, but they're also responsible for
billions of cold, calculated human deaths during the Fraternal war,
etc.).

The VS universe has strong existentialist influences - such is (my)
life and art does tend to imitate life J. However, that is not to say
that the goal is for VS to become an emo angst-fest of ennui. At the
beginning of UTCS, Deucalion is a bit of an emotional mess, but this
is to be expected. He's just gone through a traumatic event that
nearly killed him and did kill his best friend and (for
simplification) brother-in-law. He's rather a bit shaken, his life
plans have just been rudely interrupted, and, to add insult to injury,
the Aera have just started invading Forsaken space while Deucalion was
recovering from his injuries. It's a situation that is hobbled with
guilt and the impotence of any individual against the uncaring and
unnoticing motions of the universe. It's a situation ripe for
catharsis and, just as importantly, for life altering change - it is
therefore a good starting point for a player to take over a character
who already has a defined past. They can take the character in rather
different directions without it breaking all suspension of disbelief,
as "he cracked" and "you're not the same since XXX" are easily
intelligible human reactions to tragedy. At the same time, they can
use the existing past and the benefits and limitations it has bestowed
upon the character as either guide or leverage - his past gives him
skills that make him useful, hence a target for interaction with other
entities, and it limits his direct involvement with governance or
military forces. As he's already left the Protectorate navy, he can
much more believably stray further, or wander back - the choice is
his, and thus, the choice is the player's.

There is always the issue in an open-ended game of when things end -
where is the resolution? For UTCS, the big plot will advance whether
or not the player does too much. If one only wants to see what happens
in the big picture, one need only play a bit, and then play a
(safety-seeking) waiting game - in doing so, one has played out a
rather boring life, but it's a player's choice, and, if they aren't
actively playing, then they won't get to see the effects.  For UTCS, I
think having a number of soft-endings for a set of little plots
(hence, the plurality) is the best choice. Profession/lifestyle
specific plots can have an apex or plateau if not a conclusion. The
closest gameplay that comes to mind is the guild-leveling in
TES:Oblivion, but the pace of advancement in the Mage's guild seemed
less than proportionate to the tasks at hand, very gamish, but I
digress.

\section{Themes in Vega Strike}
\label{sec:VSthemes}
One recurring theme that is present in the overarching VS universe
story (although echoes of it will be seen in the player stories as
well) is that of intergenerational relations, of parents and their
children, in this case, the figurative species-children of various
groups and the effects that the actions of those who came before them
have on their own development. Our parents are our first gods, and
even if they succeed as parents, they will always fail us as gods. I
thought it would be particularly interesting then, to start the story
off with a parental group that really was, in many ways, godlike - but
still failed, and in their greatness, had their failure cast a
commensurately large continuing shadow over those who came after them.

VS, however, as much as the TWHON may impact the events that unfold,
should not be seen as a story about them. Their day, and indeed, their
greatness, and even their existence as anything but vague and warped
shadow of what they once were, has long passed. While it is worthwhile
to understand their place in the story, I would caution giving them
too lavish a portion of our attentions - gods may seem interesting,
but stories about people are much involving in the long term, if for
no other reason than we're not particularly good at portraying things
beyond ourselves in any closeup detail. It's easier to believe it's
not a man in a rubber suit if the behemoth is only seen at distance,
from the corner of the eye, or, as I would prefer it for most portions
of the VS timeline, evidenced primarily by what they have long since
wrought. This likewise removes many burdens of contemplating what
godlike beings would do interacting with the likes of us, or even our
post-human successors -- with the action in the past, we can leave
many motives mysterious, and focus on how characters more like
ourselves must deal with a reality of consequences, and not the
possibilities of dead gods.

\section{How to think about various groups}
\label{sec:groupintuitions}
This subsection isn't designed to give you all of the info on a given
group (check out one of the appendices for that). What this subsection
{\bf is} designed to do is to give an idea of what frame of mind one
should be in when considering a particular group in the VS universe
and how that group relates to others, themselves, and their
surroundings.

\subsection{Humanity}

One might think this part the least necessary group to consider, but
it's actually the most tricky. As humans, we've got a pretty good idea
of how humans operate. As we drift away from our current conceptions,
either confusion or disbelief can ensue. And we're going to have to
drift away a bit here, both for groups with trans-humanist directions
or aspirations, and simply because of the 1200 year gap between
ourselves and most of the VS cultures.

\begin{itemize}
\item {\bf Rule number one} - These are not the people around you. At least, many
of them aren't. The people of the 33rd century, by and large, bear
less resemblance to you than you do to a 10th century peasant - this
much is to be expected.

\item {\bf Rule number two} - Some of these "people" REALLY AREN'T the
people around you - at all. It's not just the cultural gap. Thinking
about the Purists as fairly normal, if scared, people, the Unadorned
as somewhat nutty religious people, the Forsaken (even more like
modern man than the Purists) as bitter people, the Highborn as
self-absorbed (and perhaps mildly self-deluding) people and the
Merchants as greedy people can lead to somewhat reasonable grips on
how these groups operate - they are, at heart, fundamentally still
human, if culturally distinct from today's climate. Even the
Mechanists can be superficially grokked by starting with a zealous
level of self-hatred directed at the limitations of their human
bodies. However, thinking about the Andolians or Shapers as just human
will delude you, and lead your conclusions astray. They are not yet
alien, but they are intensely foreign to the humanity we are familiar
with; they no longer think like us. The Andolians, collectively,
haven't forgotten anything meaningful for over 900 years. Each
generation grows up with immediate and nearly innate access to more
information than each generation before. They are connected, not just
in the simple physical sense of their link, but also in social senses
that modern man simply isn't. They don't think about self and the
other in the same way we do - they can't. The Shapers have adult minds
by the age of 7 and even their dullest healthy member surpasses most
modern humans. They are a society whose rate of idiocy, mental
defects, physical defects, malnutrition and insufficient pre-natal care
is so microscopic, their disease rate so low, that one it it suffices
to think of them as a post disease, post illness, post weakness
society. Theirs is a society of extreme individualists that runs
smoothly because they're all up on the game that's being played -
duping the Shaper electorate makes bribing the Supreme court look like
something a drooling infant could accomplish by accident. We share
more genes with the SuSims than with the Shapers. They are not gods or
demigods, or any such thing, but to think of them merely as human, is
to do them insufficient justice.

\item {\bf Rule number three} - The "Purist/Luddite" test: while you need
not agree with the eponymous groups, if you can't understand on a
permeating, gut level why these groups are so obsessed with bounding
what constitutes humanity and what it means to lead a human life, then
you don't yet understand the "humans" of the 33rd century that inhabit
the VS universe.
\end{itemize}
Key differentiable human groups include:
\begin{itemize}
\item Andolians

It would perhaps be inaccurate to say that the Andolians are actually
friendlier than the other major meme-groups. More accurate would be to
say that they are more tolerant, as much because they can afford to be
as because it aligns with their outlook. They are, however, often seen
as patronizing or even condescending in their tolerance of other
groups. This is still seen as preferable to the outright disgust,
hatred, or dismissal that can often be experienced between the
meme-groups. The Andolians often refer to each other with sibling
terminology, the Klk'k, even the non-linked, with diminutive sibling
references (bro-chan, sis-chan, etc.), non-Andolian humans in the
protectorate as "steps" or "steppers", the Purth as "little ones" (an
ironic touch, given that the Purth are extremely large), and
non-protectorate humans as "cousins". Such references, however, are
made only in casual discourse, and in generally unambiguous fashion,
with actual relations being made pointedly clear.

\item Forsaken
\item Highborn
\item LIHW
\item Luddites
\item Mechanists
\item Merchants
\item Purists
\item Shapers
\item Unadorned
\end{itemize}

\subsection{The Klk'k}

They're wisenheimers, to a degree. Their sense of humor permeates
their civilization more so than ours, making for odd juxtapositions,
such as it being entirely appropriate to be cracking jokes while
fighting, murdering, or engaging in serious policy decisions. The key
to thinking about the Klk'k is this - as much as they may seem to have
progressed along remarkably parallel lines, they're still aliens. As
SF author Gregory Benford once said, "the thing about aliens is,
they're alien." The Klk'k are enough like us, compared to all of the
other aliens, and they can work with us, that we keep wanting them to
be like us and expect them to be like us - but they aren't like us,
and it's always disconcerting when they prove it. One must imagine
asking a Klk'k why they have just done something, having them explain
in what appears to be a rational fashion, and still just being
dumbfounded as to why they did what they did - between differences in
axiomatic values and divergence in the nuances of the explanation it
just wasn't the same way of thinking about the situation, and thus
they arrived at a different outcome.

\subsection{The Aera}

The Aera got the short end of the stick - they drew the bad lot in the
running for "butt of cosmic joke" (perhaps they failed to
appropriately bribe the AUTHORS). Their planet was unpleasant, their
position in the jump network was supremely non-optimal, their timing
was poor and made even worse by the fact that they didn't know that
everyone else was going to run out of real estate soon enough
anyway. They aren't boogie-men, they aren't monsters, they aren't
ravenous alien invaders. They are an abused and shortchanged group
looking to survive in a universe that has repeatedly shown itself
uncaring to their existence. If the Klk'k disturb us when we are
reminded that they are unlike us, the Aera disturb us most when we are
forced to realize that we are not as different as we might like to
think, beneath bodies that each considers extremely ugly. Their
viewpoint tends to be colored by suspicions and certainties of
antagonism, but these are the result of a profoundly guarded outlook,
rather than the delusions of a human paranoid. Aerans are actually
quite distinct as individuals, but their fundamental pack and
abstracted pack loyalty structures allow them to operate cohesively in
groups in a manner that seems far more lockstep, frighteningly
authoritarian, and homogeneous to a human observer than it actually
is. The individual is celebrated post-facto. A life's accomplishments
cannot adequately be judged until that life is completed, from an
Aeran perspective. Don't think of the Aera as bad, as evil, or as
inherently inimical to the other races - this would be a miscarriage
of justice, and not even an oversimplification, but an
untruth. Rather, empathize with their miserable initial situation,
even if the only sane way for humanity to deal with them, alien and
resolute as they are, is to shoot back at them.

\subsection{The Rlaan}

The Rlaan are intensely alien. If the Klk'k are frustratingly alien,
and the Aera are at times painfully alien, then the Rlaan are
mind-bogglingly alien. They are, in fact, so alien that we can't
really understand how alien they are, because we can't identify what
in their behaviors is just complex and what is derived from more
fundamental differences. The scale just saturates at some
point. Neither they nor we really understand one another, and we
merely have gotten good at pretending. Take their civilian/worker -
defender split; they view any individual capable of willingly killing
a worker the way we'd view someone who liked to feast upon a raw,
unborn fetus, freshly cut out from its mother's womb, while wearing
its freshly raped infant siblings as shoes so that his feet won't get
cold while he's carving a scarf out of the mother's back and humming
along listening to the screams of the father as he slowly slides down
an impaling post. We have nothing remotely comparable to that -
nothing. They experience the world in parallel layers at a time, in
sight, in sound, in thought, decomposing their reality into fragments
and piecing it back together. They live for hundreds of years, but even
if that's actually a fairly short time for life at their temperatures,
they don't have any sense of individual urgency in their life. While
the Aera are vibrant individuals underneath the firm veneer of their
society, the Rlaan are, by and large, extremely similar creatures
underneath the cloak of chaotic motion that constitutes fair portions
of their society. Rlaan populations are large enough that, even with a
much smaller standard deviation, there are exceptional individuals,
but most Rlaan, especially the workers, are remarkably interchangeable
despite their differences - this is not because they do not
differentiate themselves significantly, but rather because they
differentiate themselves in ways that are reversible. Underneath
whatever they are currently doing and believing, Rlaan minds seem to
function in remarkably similar fashion to one another. A conversion to
a new mindset can make the average Rlaan a good stand-in for any
another.

Humans, however, do not often interact with the uninteresting Rlaan,
and it greatly colors our perceptions of them. Only those Rlaan
trusted with having inklings of how other minds functions are allowed
to be their diplomats. The anthrophilic Rlaan-Briin are vital to
increasing cultural understanding, but they're a distinct minority
among the Rlaan, and those, even of the Rlaan-Briin, who are capable
of moving toward "foreign" from "alien" are an even smaller
minority. We, on the other hand, have never moved from "alien" toward
"foreign" for them on our own. It is only as the result great
assistance and analysis from AIs and PAIs that we can now convince
ourselves that the Rlaan receive messages truly similar to what we
believe we are sending them.

\subsection{The Uln}

Boorish, feudal, and seemingly anachronisms, the Uln are alien, but
surprisingly uncomplicated to the degree that our interactions are
unsubtle. They are willing and well practiced in mimicking aspects of
the civilizations and societies of those they deal with, and, though
it masks deeper misunderstandings and differences, this allows them to
at least appear less alien than they truly are in the context of
particular dealings with them. They are, in many ways, a deeply
insecure people, given to grandiose displays of overcompensation.

\subsection{The Shmrn}

\subsection{The Dgn}

Though from the same stock as their Shmrn brethren, the Dgn have been
far more effectively subjugated by their Shaper masters. They do not
welcome their condition, but do not find it particularly irksome.

\subsection{The Saahasayaay}

The Saahasayaay thirst for violence and consumption is best described
in terms of lust. Their embrace of violent means to achieve ends may
lead one to believe them to be hedonistic sadists, but that would be
somewhat askew. They do not perceive their domain to be that of pain
or suffering, but of death. All else is incidental, except that it
reflects their belief in ultimate dominion over life. With their own
peculiar degree of immortality, they are consumed by their fascination
with the termination of existence. The Rlaan have often regretted not
leaving them to rot on their stagnant stone-aged planet.

\subsection{The Purth}

\subsection{The Alphans/Betans}

\subsection{The Ancients}

\subsection{The TWHON}



% LocalWords:  Kardashev UTCS psionics FTL technobabble Skywalker Yossarian Uln
% LocalWords:  Rimward Aera Andolians Deucalion TWHON timeline Mechanists Klk'k
% LocalWords:  Andolian Purth LIHW Aerans Aeran Rlaan Shmrn Dgn Saahasayaay
% LocalWords:  Alphans Betans
