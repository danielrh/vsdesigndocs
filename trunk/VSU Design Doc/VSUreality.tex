\label{chapt:VSUreality}
There are many things that {\it could} be in the VSU. To list them all
would be impractical. Rather, to provide a framework for insight into
the nature of reality in the VSU, we will discuss the sorts of things
which are prohibited from being in the VSU or should otherwise be
avoided wherever possible, and likewise note some of the key details
that must be present in a canonical rendering of the VSU.

\section{Things that should not / will not be seen in the VS universe}
\label{subsec:thingsnotinVSU}
This section presents, in no particular order, an indicative set of
prohibited items, categories and entities.

\begin{itemize}

\item Interventionist deities or agents thereof

Whether or not a divine being or beings exist in the VS universe is
irrelevant. Neither evidence nor action by any such entities occur in
the VS universe.

\item Interventionist deities or agents thereof (pretending to be aliens)

The term ``god-like'' could easily be applied to any type II or type
III civilizations (Kardashev scale~\cite{Kardashev}), or even members
thereof. This does not make them actual gods. Nor should they be used
as stand-ins for traditional human deity-figures, redemption figures,
moral archetypes and so forth. That they are profoundly powerful does
not make them well intentioned, infallible, or even necessarily
wise. That they are ``god-like'' in power merely denotes them as
powerful enough that they possess a potential for actions at greater
scales. It is worth noting, if nothing else, the number of powers
previously reserved for gods that even our own civilization (not yet
even a type I civilization) has already secured for itself without attaining
any shred of divinity.

\item Absolute Morality, embodiments thereof, and anthro-exclusive destiny

There is nothing intrinsically {\em good} or {\em evil} in the VS
universe, be it an action or an entity. More importantly, there is no
external guidance toward ``the good'' or away from ``the evil'' that
acts as some imperative societal force, nor a presumption that what we
consider elements of ``the good'' will prove to yield higher
survivability than aspects our modern societies do not deign to
associate with goodness. While it is certainly more difficult to see
some actions as beneficial than others, as there is no presumption of
an external calibrating entity in the VS universe, such judgments are
societal, and their merit only capable of being judged properly long
after any associated actions have been taken. Additionally, the
continued existence of humanity should not be blithely assumed to be a
universal ``good''. While humanity and their descendants will play an
important role for a certain time-period in the VS universe, this is
happenstance, and not providence. Indeed, much of what happens after
UtCS is in many ways a decline of humanity and the supplanting of
humanity by its children.

\item Magic (as magic)

The following, among other things not listed, do not exist (at least
as people currently describe them) in the VS universe: ESP, psionics,
talking to the dead, spiritual possession, auras, telepathy,
telekinesis, purity of essence, and clairvoyance. Events in the VS
universe will not be determined, nor even affected, by `fate', gods,
prophecy, or other elements of the supernatural.

\item Magic (pretending to be technology)

While Clarke's law posits that any technology, sufficiently advanced,
is indistinguishable from magic, the converse is decidedly {\bf not}
true in the VS universe. Just because we can think of some magical
means for doing something does not mean that any technological
implement, no matter how advanced, can ever achieve the same
effect. This is not to be confused with things such as fusion reactors
which we have firm understandings for how, in general principle, one
might work, but cannot fathom how to build a practical one. Allowing
leeway for overcoming implementation difficulties, or even profound
knowledge gaps is a different beast altogether than positing something
to exist which we can already know to be in violation of numerous
aspects of our model of reality. This means no perpetual motion
machines, no instantaneous galactic communication (although we do
still regrettably violate causality with non-instantaneous FTL), no
living-super-armor (it's just a fad fuelled by, one presumes, a poor
understanding of high-tech carbon composite materials and the sorts of
metabolisms needed to produce such on the fly.), no life-force, no
ascension, and no ``energy-beings'' without a concrete definition of
the sorts of energies involved (i.e. ``being existing as patterns in
EM waves'' is OK, but ``ascending to pure energy'' is just pandering
to a magical reality, all apologies to Stargate, Babylon 5, etc.). We
must strive to differentiate the unlikely (even perhaps sometimes
coming close to the borders of implausibility), which is fertile
ground for science fiction (e.g. room-temperature superconductors)
from the truly magical masquerading in the guise of science
(e.g. though I do enjoy the series, most Doctor Who episodes are
fantasy, not science fiction) and from fundamentally unsound
propositions passing themselves off as profound knowledge or advanced
technology (e.g. splitting the beer atom, gaining mutant powers from
gamma radiation, almost anything from the movie {\em What the \#\$*! Do
We (K)now!?}, entities that gain mass without consuming anything,
re-spinning the earth's core with a hydrogen bomb, neutrino weapons,
hacking alien mother ships with a laptop, instantaneous evolutionary
accelerators, ``polarity reversal'' as the answer to everything, and
claims that ``humans only use 3 percent of their brains'').

While the VS universe allows some liberties with known or
expected physics for the sake of plot and game-play possibilities, we
are bound by the laws of physics unless explicitly relieved from said
limitations. Deviations from our base physical reality have a nasty
habit of causing cascading implications that aren't desirable, and
should only be undertaken with both good cause and firm caution. 

We make explicit and limited exceptions for FTL, shields, and other
related space-warping technologies which we bin under
``gravitics''. Gravitics is known to be junk-science, but, as magical
additions to reality go, can be presented in a reasonably
self-contained fashion as long as nobody stares too long and hard at
it. FTL is, for better or worse, necessary, and shields are, if not
necessary, sufficiently desirable and expected that we've made room
for them.

Toward the end of limiting the sickly spread of junk science and
outright silliness into every aspect of VS, we should avoid
technobabble like the plague that it is. While it's fine, in documents
such as this, to ponder to some degree how some of the VS tech is
presumed to work in order that we can coherently and consistently
describe the properties of objects using said tech, the user should be
heavily insulated from such discussions. Firstly, what the user needs
to know are the end effects (e.g. does it pierce shields, how does the
range fall off, how much energy per second does this reactor provide,
etc.) and what the useful inferences they can make are (e.g. this is
a ``shield-type'' weapon, other ``shield-type'' weapons will have
similar properties) more so than any nitty-gritty made-up details of
how tech we don't actually know how to build works. Secondly, if we
attempt to explain how something no one knows how to build works,
we're almost certainly going to come off sounding like either outright
fools (for using real terminology incorrectly) or as purveyors of
gobbledygook (for using unreal terms randomly). Star Trek is the
perfect example of what {\bf NOT} to emulate here. There will be no
``transferring of power from the rear neutrino phase shift emitters
to synchronize our tachyon-positron field into a stable
co-phase matrix''. In the long term, it will only be held against us.

\end{itemize}

\section{Things that should (eventually) be seen in the VS universe}
\label{subsec:thingsseeninVSU}
It is important to decouple the VSU, as described in this document,
from our current attempts at implementing a rendering thereof
in-game. The following items may currently be unimplemented, or
implemented in a contradictory fashion. They are, however, key to the
nature of the VSU, and {\bf will}, barring profound retcon, eventually
come be expressed as described in anything that can be considered an
accurate portrayal of the VSU.

\begin{itemize}
\item Law and order

With all apologies to such noted SF authors as Ben Bova and Poul Anderson, space (at least in the VSU) is not
reserved for libertarians. In the VSU, statist influences will
persist, and, by the UtCS period, will have pulled much, though not
nearly all, of the frontier back under various degrees of control.

\item Large Scale industrialization and infrastructure deployment

Interstellar wars are not fought by economies built around goat
herding. Heavily developed planets should be common, and feature
similarly heavily developed orbital infrastructure. Docking stations,
planetary mass drivers, dedicated shuttle fleets, other key components
connecting planetary populations to space, and even various flavors of
space elevators should be widespread.

\end{itemize}


\section{Physics and Technology in the VS Universe}
\label{sec:VSphysics}
\subsection{On Weaponry, Defense, and Damage in the VS universe}

So, assuming one has the ability to diddle with the surrounding space
(leaving discussion of whether this, or any other stated principle,
was/could be a good choice for a fundamental assumption to another
time) how might one construct a shield?  Well, I thought perhaps one
could set up something based around gravitic shear forces (locally
violent, but, with opposing forces mostly canceling each other out at
greater distance due to super-linear falloff).

I then figured it would probably be worthwhile to augment such a setup
with an EM component, so as to assist against charged particles, as
charged particles are easy to accelerate, and therefore a likely
choice in assorted weapons systems. So, when descriptions (minimal as
they were) were written for shields, they were referred to as
providing a combination of gravitic and electro-magnetic protection.

Now, where did this lead me (at least as far as I saw it) - almost
everything except for something that looks like a shield should
penetrate a shield in some manner to some degree.

(a brief aside: ship collisions are somewhat outside the scope of this
post - suffice it to say that they should be much more catastrophic
than they are, but the reason is not related to shields - it's that
our damage model only works on energy right now, and doesn't look at
time related components, so if a ship smacks into something at 300m/s
and bounces off at 100m/s in the opposite direction we apply damage
due to the loss of kinetic energy, but don't currently address the
problem that, if this collision took 1/10 of a second, the ship
experienced an acceleration of 400g's, the pilot should be paste (even
assuming some (limited) means of inertial compensation as a cheap way
to warp space may be deemed to provide), and the ship should be
assorted bits of fine debris - this is a bug, a feature failure in
need of fixing. We don't have a model for acceleration tolerance,
clearly, we need one.)

Shield effects, by category:
\begin{itemize}
\item LASERS and other coherent EM radiation 

hard to get a beam of light to interact strongly with this setup at
all (unless one assumes that photons passing through the distorted
topology can be convinced to dump energy and shift down the frequency
spectrum in return for degrading the desired topology - but the more
that I've thought about that, the less it appeals to me, so let's not
spend much time there) but it might interact weakly, de-focusing the
beam. For low frequency radiation, de-focusing is going to be quite
detrimental (in terms of the likelihood of armor being capable of
dealing with incoming beam) but one imagines that xasers and grasers
are still going to be quite damaging even if the incoming beam is
distorted and defocused. Hence, at best, fair protection against low
end laser weapons, to negligible protection against high-end laser
weapons. This translucency (not transparency) has the benefit of making
it easier to explain how EM spectrum sensor data gets in, but causes
some problems with pilot-line-of-sight (upon further reflection, I've
come to the opinion that chuck raised an excellent point with respect
to his comment about the insistence of early astronauts on capsule
windows - there are only two major human groups in the VS universe
with pilots that likely wouldn't demand the same, if not windows per
se, then some semi-direct optical access (I also briefly, and not in
particular seriousness, pondered the notion of an "optical fuse" )-
but this delves into a whole other train of thought, so I'll stop it
here for now.)

\item Solid objects

should interact fairly strongly with the shear forces. Complex objects
could end up giving up non-negligible amounts of energy in undergoing
deformation or otherwise smacking into bits of themselves. However,
given high initial velocity, sizable portions of the incoming
remnants of the object will not be sufficiently diverted and will
still intercept the target. This is still a preferable scenario, as a
defocused impact of something more resembling dust and shrapnel should
be a lot easier for armor to handle than an intact shell. (Unless of
course, one doesn't have armor, in which case one may have just traded
one set of holes for many sets of holes.)

\item Particle beams

A) Charged - high velocity makes them hard to divert with the
gravitics, again just gaining a defocusing, but that's what the EM
systems are there for helping out with. Still, in the end it's just a
very good defocusing and diverting, and can't be expected to stop all
the incoming particles completely.

B) Neutralized - EM field doesn't help in any meaningful way,
defocused, more-so than a laser, but protection is pretty poor, and
it's mostly up to the armor.

\item Plasma 

A)Net-neutral, or B) net-charged clouds of high temperature ionized
particles that are likely to be fairly effectively diverted by an EM
field unless the plasma density was quite high at the time of
interaction (still efficiently diverted in such a case, but perhaps
not effectively).

\item Shields and shield-like weapons

Directly act upon the topology created by the shields, significantly
degrading them. However the directness of their interaction also means
that their effects do not penetrate the shields.

How I saw this playing out in terms of game mechanics: 

Firstly, as shields degraded (topology becoming unstructured, shear
forces going away), anything that penetrates a shield already would
penetrate more. The EM field wouldn't degrade in the same manner as
the space-warping component, but it was only useful in mitigating
charged particles anyway.

\end{itemize}

Weapons, by category:
\begin{itemize}
\item Lasers

would seem to be quite nasty beasts in that they mostly ignore
shields, especially at higher frequencies, except that lasers have
lousy energy efficiency, especially at higher frequencies, and
especially given that laser inefficiency tends to materialize as waste
heat. Thus I saw lasers as weapons with extreme cooling problems,
either resorting to open cycle cooling (venting coolant = limited
ammo, limited re-fire rate) or {\em very} slow re-fire rates (also a source
of perhaps interesting complexity if/when any form of heat modeling
gets implemented). Likewise, the higher frequency lasers would be
prohibitively expensive and potentially bulky beasts, probably not
found in small craft. Additionally, as they don't interact strongly
with shields, they wouldn't be good weapons for degrading them
rapidly. Range would be good though,

(lasers don't degrade as the inverse square, but diffract according to something along the lines of 

RT = 0.61 * D * L / RL 
where: 
RT = beam radius at target (m) 
D = distance from laser emitter to target (m) 
L = wavelength of laser beam (m) 
RL = radius of laser lens or reflector (m) 
) 

\item Solid objects 

Lower energy requirements (could also have internal energy sources, as
per rockets), easier cooling solutions, good rates of fire, degraded
by shields but degrade shields, and become increasingly effective as
the shield degrades. Limited ammunition. Can be augmented (at
increased size/cost) by addition of shielding, and/or nuclear or
anti-matter warheads. At the (expected relative) velocity these would
be impacting at, conventional explosives would not be useful
additions. Damage does not decrease with range (although for reasons
of limited processing power, a "max range" still needs to be specified
engine level).

\item Particle beams

A)Charged - low yield electron beams can already be made with very
high efficiency - but cranking up the power will drop the efficiency a
lot. More importantly, any charged particle beam suffers from severe
thermal and electro-static bloom. The constant on the super-linear (I
believe it's actually an inverse-square) decay in beam density can be
helped by using more massive particles, or accelerating to
relativistic velocities for the sake of time dilation, but at the
expense of efficiency (significant relativistic velocities are a
{\em huge} energy investment, neutrons are dead weight to an EM
accelerator, and only so many electrons can be conveniently added to
or removed from an atom). To make matters worse, one's ship will
accumulate net charge if repeatedly firing a charged beam, unless the
excess charge is bled off somehow (I've seen indications that
alternating between positive and negatively charged firings is a "bad
idea (tm)" due to creating a current loop involving the vessel). So,
to sum up, the range is pretty bad, the efficiency is questionable,
there's probably a hell of a re-fire delay as one cleans up the charge
accumulation problem, and EM fields can do a lot to defocus the
incoming beam. However, if you are close enough, and your particle
density is high enough, then what does get through would do nasty
things to armor, surface mounted electronics, and throw off lots of
secondary radiation.  

B) Neutralized - (and by neutralized I don't mean "neutron beams",
because I haven't the foggiest idea how to generate or accelerate them
effectively in anything resembling a coherent beam unless we start
talking about space-warping that is probably powerful enough that'd
we'd have to go back and revisit the whole "can't do to much to
photons" issue which I'd rather not, and besides, that would probably
mean that shields were impervious to just about anything... which is
rather much not the goal either) more specifically, a beam of
particles that has been rendered charge neutral; one in which
oppositely charged particles (likely electrons) are added back in
after acceleration (both must have been accelerated) to neutralize the
beam. This will almost certainly defocus the beam, and again almost
certainly drop efficiency even lower. However, it avoids the local
charge accumulation problem, this removes electrostatic bloom, leaving
only thermal bloom, increasing range, and it also negates the
effectiveness of EM fields to disperse the beam at the
target. However, it also negates the current and charge accumulation
effects on the target that might damage electronics. Still, plenty
unpleasant on impact, only mildly affected by shields, but range isn't
as good compared to lasers, and efficiency is only questionably
better, and could easily raise similar cooling/re-fire issues.

So, as for beams - mediocre range due to bloom effects, efficiency
questionable, neutralized beams achieve good penetration against
shields at cost of even lower efficiency, charged beams have lousy
penetration against shields, but can probably be used in efforts to
disable the target's electronics (at the least, those present on the
surface, or accessible by necessity (engine/reactor) - the core
protected elements are going to have to be in some Faraday cages with
optical links to the externals (optical links don't like shear forces
though, so they could break with some probability upon impact or
impact resembling damage). Ammunition (the particles in question)
necessary, but in sufficiently small quantities per firing that it can
either be ignored or modeled as extremely cheap, small, and
plentiful. Some noticeable degradation of shields due to some
interaction.


\item Plasma 

Last I investigated, unless there's some way to make plasma somehow
generate its own magnetic fields of exceptionally interesting (read:
somewhat absurd) strength, or one wants to accelerate the plasma to
very high velocity (which would start to look something more like a
shorter pulsed version of the the beams above), it's not going to be
an effective weapon at anything beyond the shortest of ranges, because
it expands like no one's business (our dear friend the inverse square
law, but with indications of unforgiving constants, the prevalence of
plasma weapons in many sci-fi works notwithstanding) and in every
direction. High-tech flamethrowers with interesting electrical
properties are cool, but not very effective unless one is close enough
to read the serial numbers on the target's fuzzy dice, never minding
the effects of EM fields on ions, which further limits effectiveness.

In short, one could build the bolt (short pulse) rather than beam
version of a particle beam, and it would be rather similar to the
particle beams, and not what one traditionally calls a plasma
weapon. Or, one could build a reasonably efficient plasma weapon, but
be limited by rapid falloff to the shortest of ranges. Ammo for plasma
weapons should be in the dirt cheap, small, and exceptionally
plentiful category. If you're actually close enough to get any
reasonable number of particles past the EM fields, you'll do nasty
things to the electronics, and you can probably afford to keep firing
for a while. Shield degradation can be somewhat more pronounced than
particle beams if more matter is being thrown at the target.

\item Shields-and shield based weapons

Ammo, none. Shield penetration, none. Efficiency, mediocre-poor, hence
re-fire, fair-slow. Target shield degradation better than any other
damage source. Transmitted damage after shield collapse (topology
unstructured) worse than any other damage source, but non-zero. Damage
vs. unshielded objects significant.

\item Missiles

Mostly depends on warhead type. Shielded kinetic is one option, single
shot weapons of various types also options, as are bomb pumped lasers
or simple nukes. Ultra-low-yield (0.5 - 1 ton range) fusion warheads
are presumed commonly available (preferable to chemical explosives due
to the manner of transmission of the energy, namely, high frequency
radiation and neutrons).
\end{itemize}

% LocalWords:  VSU Kardashev psionics FTL Stargate gravitics Gravitics
% LocalWords:  technobabble retcon statist UtCS gravitic
