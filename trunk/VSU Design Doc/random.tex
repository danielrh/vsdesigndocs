\section{Aesthetics advice and other conversations}
\subsection{Rlaan Aesthetics}
Conversation (excerpts) on Rlaan aesthetics
t: we'll start with the "how much is alive" part
not too much, although there are a lot of organics used, especially internally.
parts of the life support system are made from arguably living material, and the automated repair systems rely upon living organisms
ToO: hmmm.. is it something that can be seen?
T: excepting one case - not really, no. Not unless, sometime far down the line we do interiors. The only visible exception is the effect that having living secretion sites would have on what we might want damage maps to look like.
for visual range, the Rlaan are heavily blue shifted relative to our visual spectrum
ToO: near uv?
T: yes, they can see in near UV, but their red-end range is inferior to ours, and their peak response point is also shifted up in frequency, although their frequency responses are more even than ours. They use 5 pigments vs. our 3.
ToO: interesting.. very much so
the reason I had previously brought up the visual range, was for colours.. so one could get an idea of what they would look like
or rather one of the reasons
T: their color choices may look a bit dark (we don't percieve blue tones as intensely) and some of them bland in contrast (different pigment response curves)
ToO: ok.. so no seeing through the hulls and that whatnot.. what was the next one?
T: (although there'll be images on the hull we can't see at all, and some of the red-end of our hulls will just look black to them - their sun isn't a yellow star like ours - no superman for them ;-))
ToO: *nods* I thought that maybe some insignias and markings might look broken or even fairly faded, as parts of it, may go out of our visual range
TribeOOne (11:44:19 PM): which would give it an interesting look
T: indeed.
the 5 models we have for the Rlaan are pretty good indicators of style for their military craft
 no visible engines (the Rlaan just do gravitics)
 Big dual purpose radiator/shield manipulator fins
ToO: how about for the larger craft.. battleships, and bases?
T: so, we've got 3 Rlaan small craft, and 2 Rlaan capital vessels right now
 a destroyer and a cruiser
ToO: so.. do you want the stationary rlaan craft to follow the same general gist? squatish insect puapa (sp) 
T: no, they should have a lot more radial symmetry rather than bilateral
in 4s preferably
so, for the stations
My first thought is to go for a cored, squashed, pruned, and resurfaced sea urchin look :-)
less vaguely:
an ovoid (as opposed to a spheroid, hence squashed)
with an empty center at its rotational axis (cored)
with intermittent long, thin spires and smaller fins (pruned)
ToO: more towards crystalline, or with the metallic chiton/membrane look?
T: the latter, as with the underside spires on the (destroyer?) with a surface similar to the topology and aesthetics of the cap-ships, excepting that civilian installations have a {\em much} more whitewashed color scheme
(resurfaced)
ToO: ok.. sounds a lot more simplified then I think had been coming out of some folks looking to do rlaan
T: well, that's first approximation
the surface isn't smooth though, and each of the four sections isn't perfectly ovoid
so there's plenty of detailing to be done
ToO: *nods* irregularities and so forth
T: but everything is rounded
use the 2 capships as your guide to surfacing
ToO: ok.. I'd been coping the uv maps for a little bit of a guide on some of the ships
T: Just as long as it doesn't get mistaken for Mon Calimari design, you're probably somewhere on the right track :-)
ToO: heh
T: so, the key is this
design a very interesting quarter of a station
for any Rlaan station, really
and then make 3 more of them
the only exception for this would be at the very small end (too specialized for arbitrary symmetry)
or on something like a shipyard
where it's not a practical design for enclosing things
ToO: *nods* hmm.. so you can have fairly expansive and maybe even elongaded forms.. that still follow the general outlines of this.. or are elongated forms out?
T: in what sense elongated?
ToO: well.. I wrote that thinking long.. but if you're going with general crab like shapes.. then elongated would actually be from the center out , rather than along the length of the core
say the core is the y access.. you could have the center along the y and z axis fairly wide out, but not along the y axis in comparison to the z and x
T: so radially symmetric around y axis hence z=x but span(y)!=span(z)
well, yes
they're radially symmetric, not spherical
but the bases won't look like them
they just borrow the radial concept
here are some important things to remember about the Rlaan
ToO: *nods* radial and rounded.. I'm trying to get the feel for the general way in which forms are done
T: Consider their music as indicitave in some sense of their attitude towards construction: they like to take lots of simple themes and plaster, superimpose, and alternate among them
The result, to human ears, is often hopelessly noisy, or monotonously simple, or both
two: The Rlaan are, to quote the "Tough guide to the known galaxy" "Really Alien"
they aren't aliens with forehead ridges
ToO: *nods* truly different
T: They are foreign to us, and as such are not beings that should lend themselves to comfort in our perception of their being and their edifice
thus, in creating things for them
there is a line to walk between creating questions of "but why would they do that?" and "No rational being could have done anything remotely like that"
ToO:hmm.. sounds a lot like a form of brainstorming that is drawing based.. 
T: So, having a large, prominent object of non-discernable purpose that something is built around, or a placement policy for certain necessary things that seems... odd is good, but having recognizeable objects in clearly wrong places just makes them look like idiots or lunatics
e.g. placing all of the bathrooms for human visitors inside trees - strange, possibly alien misconception. Placing all of the bathrooms outside the hull.... um.... yeah
ToO: you're looking for something that has the taste of something that was done naturally, not something that was randomly placed together
T: The best approach is to pick some fairly arbitrary goals
but ones that can be consistently applied
the key to the particular mindset of the Rlaan is that these should likely be fairly simple, but there should be many of them
and they should interact
ToO: oh.. *nods* had to think about that a moment
when I say natural.. I mean that it flows as one piece. and not like it was a pirate stealing from various races to build something 
T: The huldra and Lodur, I think do a better job of making one wonder what the designer was thinking than do the smaller craft
ToO: *nods* the smaller craft look little like different pieces slapped onto a craft to make it look different
T: they also look more familiar
they're more identifiable
ToO: there was something I did want to ask you about.. about technology in vs.. 
T: sure
ToO: is it ok to base technology off those portions of vs that use the magic theme.. so your antigrav and stuff like that
T: Some of them are based off of that, yes
ToO: I know that you are doing that with rlaan.. or it would seem so.. but in general I get the idea that folks should shy away from technology that relies on the magic theme
based off of existing.. that is
T: gravitics-spacetime warping is pretty heavily in the VS scheme of things right now
there's always a danger when playing around with *magic* that you'll paint yourself into a corner
ToO: so.. should greebles for rlaan look more like veins and spines?
aside from things jutting out.. 
also, should the fins you get with ships, be present to any degree on stationary craft
T: mmm. some spines, not too many veins. More blisters, and opened blisters with internal protrusions, and overlayed regions of different construction.
The fins should still be there
T the Rlaan like to have lots of radiator surface
ToO: ok
T: and, as mentioned, there's lots of shield control circuitry embedded in them

\subsection{Textual ship descriptions}

Some very, very brief descriptions of vessels

Manufacturer: Mechanist Defense Contracting, under contract from Confederation Navy
Class: Battle Cruiser
Designation: Battle Cruiser Mk. 32

the armament on the Mechanist battlecruiser is fairly simple
48 heavy beams, and a bunch of PD turrets
and designed such that all 48 beams can face dead ahead
12 can face dead stern
and 12- 16 can face to each side
so, a limited, if potent, vessel
a tapered appearance, starting at a somewhat ovoid part, and narrowing as it goes towards the front - then stopping suddenly in what amounts to a wide, narrow mouth with 3 4x1 restricted movement LR beam turrets as jutting tusks
on the top, behind the mouth and up the slope somewhat, a less restricted 4x1 turret. Up the slope more and to each side, more 4x1 turrets
the bottom and top are symmetric
around the rim of the ovoid part
12 cocoon shaped turrets jut out, positioned such that the gun can be pointed forwards and backwards
slanting down on both sides to angle in if need be
but heavily restricted in the other degree of motion
to the rear of the turrets, a brief ring at 45 degree angle slope with PD turrets
then a sudden extreme slope that curves around and gives rise to a bumpy engine and whatnot rear
4 docking bays, each located in a cutaway area in the side


Manufacturer: Andolian Military and Andolian Protectorate Fleet, for crewing by Andolian and Klk'k Forces.
Class: Destroyer
Designation: Nietzsche

Visual description:

The basic shape of the main hull is roughly cylindrical. The thickness of the body does not vary overmuch, though it does vary, except at the front and rear. At the rear, it tapers abruptly for a very short time, then curves in on itself, creating a caldera, or bowl-like depression into the rear of the vessel.  The main engine exhausts are in this inset area.  On the tapered part of the ring around the engine bowl are four point-defense turrets, one top, bottom, and to each side. Coming out from the main hull are four projections, one pair on top and bottom, not too much after the engine area, and one pair on the sides, forward of the top and bottom pair, but overlapping somewhat in that they begin before the other pair ends. These projections are quite thick, and shaped something like a compromise between a circle and an equilateral triangle.  Imagining them as triangles one would say that they were aligned such that the top and bottom pair pointed forwards, and the two side projections pointed rearwards.  The projections extend out about the radius of the main body, whereupon they are capped by a thick, slightly wider plate, much as one would imagine a toadstool, shitake, or portabello would appear if the stalk were almost as thick as the head, and the head flat on the top and rounded on the bottom and edge.  Just below the cap, embedded in the projections, are large engine thrusters. The engine thrusters in the top/bottom pair are larger than those in the side pair.  On top of the cap, at each of the "corners" is a largish turret, though not a massive one, somewhat tilted down from the flat plane of the top of the cap such that its gun can depress below level. Each such turret contains a single gun.  Forward of the side projections somewhat, the cylindrical main body differentiates.  The top portion becomes a bundle of six tubes (for launching anti-capital missiles) around a larger central projection, two above, two below, and one to each side in slightly svertically squashed hexagonal fashion.  The tubes are not actually touching each other, and the area in between them is filled in solid as is the area between the tubes and the central projection. Beneath this bundle of tubes is a hangar/docking bay. The combination of the docking bay and the collection of tubes are slightly slimmer than the main body of the ship. The tubes extend slightly beyond the end of the hangar.  The central tube area terminates in an inset sensor array and two small turrets, one to each side.  The missile launcher tubes terminate as one would expect them to, flush with the base of the turrets on the central projection. There is a tractor beam turret on both sides of the hangar, and a disabling turret on the bottom lip of the hangar.

There are various PD and anti-fighter turrets, but their position isn't as important at this level of description, save to say that I see 3 anti-fighter turrets on the sides of each projection to discourage loitering beneath the guns, and that PD coverage must be excellent :)

To describe a bit more...

The open face of the docking bay is in the forward direction, and though, from the outside, it is clearly much deeper, the initial open area is somewhat shallow as it terminates in a large armored door that protects the inner docking bay.

Looking at the Nietzsche from dead ahead of the vessel, one might imagine the missile launcher tubes as the hideously deformed descendants of magazine from a six-shooter revolver. Relative to the crisp radial symmetry of the six-shooter, the tubes are stretched further apart horizontally, maintaining vertical and horizontal symmetry, but not radial. Likewise, the aperatures where the missiles exit are vastly smaller by comparison to the size of the tubes themselves than in a revolver. Most notably of course, these tubes do not at all revolve, nor move at all. So they really do not look all that much like a revolver magazine; perhaps the image of a revolver is only the quick shadow of a thought that comes from knowing that one is staring at six tubes, each of which holds death, and each of which has opened its dark mouth in your direction....


Materials/texture/greebles/etc. appearance: A purely military vessel, there are no vulnerable areas exposed unnecessarily.  However, there are various sensor arrays, shield emitters, escape pod launchers, and maneuvering thrusters located on the hull, so it is not just a giant smooth armored mass. That being said, it IS a military vessel, and the dominant feature of its hull will still be the nearly seamless overlapping plates of multiple layers of armor.

Approximate Sizes: length of main body - 1500 meters, including engines, hangar bay and launch tubes.  radius of main body, 150 meters.  length of projection caps ~ 450 meters

Hopefully this gives you something to work with without completely smothering your creative license :)

Anaxander:
(Design work previously ascribed to the Anaxander will be shuffled onto one of the other Aeran ships - as that thread has been silent for a year, I think it's probably not a big deal - and I think I like what I'm coming up with better)

Description 1: (a quick verbal sketch of a quick freehand sketch - I'll write more later)

A tube with length:diameter ~ 6:1 very slightly squashed in the vertical dimension. Narrowed vertically somewhat at front to attach to hammerhead-style frontal region. Hammerhead has teardrop shaped cross section viewed from the side. Point of teardrop faces backwards. At the rear of the ship there are two similar projections on the sides of the 'engine'. There is no narrowing, rather the projections expand rapidly to merge into hull. Height of front teardrop ~1/2 diameter length (base to point) ~1 diameter (more or less) and rear teardrops are somewhat larger. Width of hammerhead - artist's discretion + field of fire for mounted PD + anti-small craft turrets. At bottom rear of vessel, forward slightly of exhaust region is an underslung docking bay, with an opening at the rear, facing back and slightly downwards.

There are six major projections extending radially from the ship, three rear, and three front. The three rear are in a top, bottom-left, bottom-right radial symmetry, and the front three are shifted 60 degrees to a bottom, top-right, top-left radial symmetry. The rear projections begin just after the engine region and are narrow at the back, wider at the front, have a gradual slope on the rear side, a flat top, and a very steep returning slope (something vaguely like a rose thorn that has been pruned of the pointy part). They do not return all the way to their original height level relative to the hull and instead merge into spinal gun emplacements that sit on top of the hull, each gun emplacement housing three long, large weapons.  Attached to the flat top of each projection is a triple-mouthed torpedo turret. The three fore projections are longer, less stout, and have maneuvering engines through their middles. They are faced oppositely (flatter region back-facing, pointy region forward-facing) and are topped with gun turrets.

Additional gun emplacements are present on top of the engine section and on the bottom of the docking bay. The longer projections are (in height) ~30% of the ship's length, the shorter rear ones ~20%. On the exterior facing flattened sides of the hammerheads (both front and rear) are "sensor stuff". Overall ship's length estimate for scaling perspective ~ 2.1km. 

Ultra-brief ship's history:
The Anaxander is old, and even with retrofits is aging significantly. It was the main Aeran cruiser at the beginning of the Rlaan-Aera conflict, having already been produced in some number. As newer designs (some already in the pipeline) were phased in, the Anaxander moved to less prestigious roles. While still a vessel of noteworthy offensive potential, it now almost universally finds itself attached to task forces rather than leading them.



Purist Star Car (Name to be decided later):
LIHW Star Bus (Name to be decided later):

First, some context:

Ownership of personal spacecraft is most akin to ownership of such boats as one would have to dock at the local marina today. Thus, while not of unreachable expense (only the luxury yachts are luxury yachts, even in space) private craft are not ubiquitous and mass transit, both public and private, is much relied upon.

The bulk of civilian passenger transport vessels (that is, excluding mercantile transport) would fill one of the following niches (with modern analogue in parentheses):

Interplanetary Personal Transit ("Car")
Interplanetary Charter and Rental ("Limo/Rental/Taxi/Charter Bus/Charter plane")
Interplanetary Mass transit ("Bus/Light-rail")
Interstellar Mass transit ("Commercial airliner")
Interstellar Recreational ("Cruise ship")
Interstellar Jump Ferry (no direct analog - taking one's car/small boat across an ocean isn't common enough)
Interstellar Personal Transit("Yacht/Private Plane - not necessarily of luxury variety")


\subsection{The Aesthetics of Deucalion with tangents on the Klk'k}
A conversation on appearances

T: so, first off, a general comment (not directly about the drawing) about Deucalion's ancestry - namely, he has designer genes that apper to draw from several groups as opposed to being a standard Shaper variant of anything, so inferences directly from or to the Shapers are fuzzy.
A: how old do you picture him?
T: early thirties
say 32 for a round number
haven't entirely nailed that down yet, but about that -
A: cool, not enough to be 'old' but old enough to have a past
T: but this is also in the context of a longer expected life-span, gene-smithing, etc
A: so he looks 20-something?
T: the detrimental effects of aging won't have kicked in yet
A: roger
T: it's less that he looks 20-something or that he is particularly youthful than just the above 
A: ah, gotcha--should look young yet mature
at once
am I right?
T: something like that ;-)
an extended "prime of life" if you will
rather than the very brief peaking of our modern phsyiology
if, in 1200 years we can't age much more gracefully, it's time to shoot the geneticists
A: all right. what else? given any thoughts to the scar suggestion?
T: he heals quite well, so it'd be more apt to due some subtle discolorations rather than overt scars
we'll see how it turns out
if it looks good, we'll work with it, if not, we can re-examine it
also might work better to limit the scarring on the face, as faces are delicate in their perception to begin with
A: I suggested decoloring--to ditch the overused stitch-mark clich
a line of lighter skin, just enough to be noticeable, probably
T: arms and such are much easier to get away with if the face appears over marred
but yeah, give it a try, and we'll see how it looks
A: ok.
T: speaking of markings
he'll need a high bandwidth I/O port
A: what does that look like?
T: discrete, not overt, near flush with the skin (self covering when not active), but obviously not of organic origin. Small region towards the classic back of the head/base of the skull. Might as well make it a utilitarian looking, if aestheticly not displeasing color.
basically, a small panel for interfacing the spinal taps to an external high-bandwidth source, for when the ubiquitous wireless data connection is too slow
A: a Pluralis trait, or is this a standard of sorts for the VS universe?
or rather Klk'k habit?
T: It's mostly Andolian hardware, but connections of various degree and make are common in other groups
A: yeah, andolian--sorry, I may be liking the homo sapiens something signature a bit too much :-)
T: The Protectorate citizens have the highest and earliest implantation rates
With the Andolians and Purth having a 100% implantation rate (the Purth are all heavily cybernetically augmented - but then they have to be to be sapient in the first place) and the Klk'k have a very high implantation rate
well, The vast majority of the Andolian population is distinguishably Pluralis :-)
so it wasn't an inappropriate phrasing
Deucalion actually has a number of augmentations, but most aren't externally visible
A: you'll have to tell me about those some time or other, since I do think it's my duty to know the characters I draw
T: assuredly :-)
hmm yeah, we'll have to work on hairstyles at some point, but anyone who's seen the various incarnations of my mop would know I'm not an authority on {\em good} looking hair
some other thoughts that came to mind
A: I have to cut his hair shorter, according to your description. does he wear, I dunno, braids? some sort of mohawk (please tell me he doesn't)? shaved patterns? there's lots of options to choose from, or combine.
T: his main concern would be that his hair doesn't interfere with his helmet, while still showing that he {\em has} hair
(the Shapers having a lack thereof as a distinguishing characteristic, thus his hair as a distancing feature)
I'm seeing hair that's short enough that he doesn't style it much
a close cut, somewhat unruly mess
A: ah,, that's simple enough. a bit of a manga mess over short hair.
no greek style curls
T: well, perhaps the hint of such on the top
waves that would become curls if they were to grow long enough
A: I'm taking note.
this is a very productive meeting!
T: basically, on the sides, the hair shouldn't jut out any more than the ears do
and the top would be somewhat more amenable to tweaking based on aesthetic issues
it'll be a different version of "helmet hair" than is normally implied :-)
A: so he's a bit vain, he he--is he a ladies' man, or a man's man, or both?
T: well, self-maintenance relative to some societal norms isn't something I'd label quite so harshly as vain
but clearly conscious of his appearance
he'd have to be, if not in the traditional sense, given that he grew up among aliens
A: I'm looking forward to know about that, too
T: He's wired fairly hetero, but growing up among Klk'k is it's own deal with respect to sexual mores - it's certainly not judgemental in the current western schema
so, definitely not a Man's man. I'd avoid the label of ladies man on the principle that, while he's sexually active, and not currently involved in any long term relationships, neither does he prioritize activities towards garnering female companionship
But, not averse to female company, no
The more interesting thought than gender orientation is species specificity :-)
He's specific to humans (of all subspecies) and Klk'k.
A: and so was his sister, from the looks of it
T: well, his "sister" {\em is} Klk'k
he's adopted
A: I knew he's adopted from the monologue, didn't stop much to think about the sister
T: Interspecies relations are still outside the mainstream. Moreso for certain pairings than for others, moreso for certain groups than for others.
Clearly, any such pairing is non-reproductive, and the physiological issues presented that govern any physical relating (let alone alien psychological issues) are non-trivial
A: and when you least expect it, you see the most gorgeous shaper dating the ugliest rlaan.
T: well, any pairing with a Rlaan would be very hard
A: just kidding--I know about their two-strata society
T: actually it's because they aren't Oxy breathers, and require entirely incompatible temperature and pressure ranges
A: methane breathers
T: and who wants to dance with a Rlaan-Briin in an encounter suit, when he has to keep making stops to recharge it ;-)
anywho
clothing
A: here's were you start defining klk'k (at least their apparel ;-) )
T: actually, not yet (at least until I get to the more formal-wearish :-) )
I'm seeing the default Deucalion casual-wear leaning towards under-clothes for his flight-suit
non-loose tank-top-esque top
non-loose shorts
flight-suit boots
A: socks?
T: no, the flight-suit boots would cover that. More likely just a couple of soft thin bands to pad above-ankle region from the rim of the boots (low boots, boots rather than shoes solely for sealing purposes)
depending on climate and such
he could be wearing the bottom half of the flight-suit
the flight-suit isn't very thick, and is form-fitting and stretchy, almost clingy
in contrast to the exceptionaly loose and baggy nasa space-suits of today
A: ok, for form-fitting flight suits we're talking entirely different technologies
yup
T: yes, different materials science
A: It's clear that I should have talked to you before doing those fixers :p
T: well, a couple of them :-) But clothing will vary greatly amongst factions. Also, it's a flight suit - it's not intended to be worn for hours on end in space on a regular basis - it's designed to keep you alive while flying, and safe in the event of hull breach and such
it's not designed for extensive repair-work or prolonged EVA
that, combined with materials advances, makes it feasable to have a suit that does not encumber the wearer nearly so much
although it's still cumbersome
more like a stretchier wetsuit/drysuit in hinderance factor than a space-suit
so basically, for vacuum wear, Hard-Suit >> Long-Suit >> Flight-Suit >> Skin-Suit >> anything else
well, Skin-Suit/Environment-Suit
err sorry scratch that
Flight-Suit/Environment-Suit
The Flight-Suit is optimized for Vacuum, whereas an Environment-Suit is optimized for one or a range of inhospitable environments due to biologicals, radiologicals, atmospheric chemistry, low pressure, or temperature range
(high-pressure needs a Hard-Suit)
Long-Suits being similar to the Flight/Environment models excepting being designed for extended periods in said environments
A: so how many in total? I'm confused. Just two, the skinsuit and the env suit?
T: Skin-Suits are vacuum rated, but offer little physical or radiological protection, and are valued more for the hermetical seal than anything else
no, several classes
A: ok, how about the hard suit?
T: Hard Suits >> Long Suits >> Flight/Environment Suits >> Skin suits
Hard Suits are completely self-contained
they have external manipulators (fingers being subject to pressure crush issues)
and are designed for exceptionally hostile environments
crushing pressure, massive radiation, highly corrosive atmospheres
A: ah, EVA pods of sorts
T: yeah
http://www.nuytco.com/exosuit.html
+1200 years
well, that's actually more like a Long-Suit, but you get the point
EVA Pod
Whereas someone wearing a Flightsuit is going to look more like much more like they are wearing thick neoprene or some such
but, as to his clothing, for casual wear, it's things he doesn't have to take off to put his flight-suit on
now, for more formal-wear
A: go on
T: firstly, the Klk'k are somewhat anthropomorphoid insofar as they have 2 arms, 2 legs, are bipedal, and have heads.

Klk'k style uni-gender semi-formalwear tends to be variants on a loose, sleeveless, single-piece garment that runs completely flat across the front, tapers slightly out from waist down and is slit in the back somewhat below the waist 
ignoring the lack of arms, the closest thing that comes to mind in human clothes is an Indian garment I saw in a fashion show once
it continues down to slightly above the ground
A: no provisions for neck or collarbone?
I take that, despite its simplicity, it's ornate
T: not entirely sure what you mean?
the neck/collarbone part
it's sleeveless, not strapless :-)
A: ah
T: there is fabric straight from shoulder to neck. but no collar
the rear opening is convenient for the Klk'k as their legs bend opposite ours
The ornateness is highly restrained, with patterns being limited to the edges and waist area, running in thin seams around the garment
the prime regions of Ktah are quite humid, and ostentatious layering would not have been comfortable
informal Klk'k clothing tends towards a short, loose skirt, with tops varying by region
A: but our guy doesn't walk around showing his buttocks through this slitted garment or does he?
T: no, the slit starts lower than that
and he wears undies
when such is called for
nudity not such a big deal in Klk'k culture. The coverings are as much practical and protective as shielding from the public eye.
pockets, belts, places to put or hang things, footwear, clothing to prevent scrapes and scratches to various areas
A: they're a practical bunch, these klk'k.
T: well, there's also clothing to show respect
show allegiance
traditional body-paintings
tattoos
A: that's a chapter on itself, I take
T: sure
they aren't anti-decoration, or even anti-clothing
they just aren't obsessed with covering themselves up with the local cloth equivalents
A: does Deucalion wear any tatoos, ritual/cultural marks?
T: yes
He has a Tk'latl tattoo on his upper left topforearm/shoulder area, with annotations denoting his rank and record in Amakakt (the martial art heavily featuring said Tk'latl)
http://www.ocf.berkeley.edu/~jackass/sket ches/klk'k-emblem.png
the Double-bladed staff (seen sheathed and cradled in the horn in the sketch) is the Tk'latl
A: written characters? I can come up with an alphabet/syllabary/pictogram set.
or just make up the marks in the forearm :-)
T: Not written characters
A: the shape of the staff itself?
T: the ranking is above the Tk'latl, and consists of a series of vertical bars of different color, denoting increasing ranks attained, from left to right
A: and in Deucalion's case...
T: the history of the matches is recorded in colors corresponding to the rank of the opponents, and lies below the Tk'latl, with vetical bars for victories and horizontal bars for losses
Deucalion has achieved the 10th of 12 possible ranks
A: no failures?
T: none against lesser or equally ranked opponents. A couple tournament losses against mid-ranked(4,5) early on, a couple challenge tournament losses against masters (11th rank) later on.
A: no horizontals then.
or tiny ones?
hehe
just joking
I can picture him now far more clearly
How do we go about the story for the comic?
T: well, horizontals still. Defeats still count as defeats :-)
A: (not changing subject, just bringing up the issue for when it's appropriate)
T: his other shoulder has calligraphy for each of the family names of his adopted parents, and the chosen name of their bond set
I don't have a clear idea about exactly what the Klk'k written languages will look like, and it's supposed to be calligraphy and they're all given names, so feel free to make up something
A:family names as in jewish tradition? "jehuda son of moses son of abimael son of samuel son of ..."?
roger
T: not quite
the dominant culture for some time uses a strict two-name scheme
when a bond-set is formed, they choose a name for themselves
A: ok
T: and that is the bond-set name for those children raised by that set
if the set changes, through addition, the name may be changed or it may be kept, but the children will retain the old name, unless very young
if the set changes through attrition, the name does not change
in order to honor what was, even if it is no more
so, you are known by your bond-set name (the name of those who raised you) and your personal name, usually given in that order
so it's not lineage tracing beyond one generation
A: are bond-sets two-partite?
T: no
A: I imagined not :-)
T: 4 is normal
2 is somewhat odd, 3,5,6 aren't considered odd, but are less common
A: sounds... complex
T: more than 6 is generally considered somewhat odd
well, the easiest way to think of it is as if a small commune all married each other
A: BTW, I've done reverse-jointed legged aliens before--see if you like the way these look-- 
\begin{verbatim} http://www.haeggalaxy.com/haeggalax y/modules.php?name=Content&pa=sh owpage&pid=3&page=4
\end{verbatim}
still, two sexes?
T: yes
seen that comic actually, I read it when you posted a link to previous work :-)
the Klk'k are built a bit differently
A: then saga of ryzom came and stole my idea of a tree-world :p
lol
T: different proportions, more hip splay, longer, flatter feet
good jumping legs
not good things to be kicked with
disproportionately large leg muscles on an otherwise slender frame... but I digress slightly
A: heh, it becomes VS lore the moment you type it, and I need all of the data I can have on those guys
well, that should be all for today--g2g
T: other attire for Deucalion, right armband signifying his rank in the Protectorate Fleet and non-duty status
(depending on where he is, this may or may not be appropriate attire, but he has one)

(Further conversation excerpts (edited) on similar topics)

T: looks too young now, or more accurately, too inexperienced. Especially the eyes, especially in the profile sketch.
Alexbetzone (2:21:07 PM): more knowing eyes... check
I found no guidelines for eye color. What color are they?
I'll fix the hair, no prob
T: Dark green eyes
hair color should be darker
more navy-blue that looks black except next to black and less cornflower/royal blue
A: all right, navy blue dark--shiny, I guess.
T: doesn't have to be shiny
A: matte hair?
I know what you mean, just kidding
T: well, highlights make it perceivable as hair and not a spilled mass of indigo ink
:-)
A: yup--no superman spit-curl
any other corrections or additions? note I haven't tried the scar thing yet. I was going to place it near the jaw.
T: trying to come up with a good rgb approx for skin tone
A: I used the tone in your mockup, but for the sketch I've been using "bleached" colors so the line art is easily seen
I won't next time
T: yeah, the color at present is a bit light, and seems short somewhat on green
but, we'll see how it goes as things progress :-)
A: ah, too purple? would you prefer a more neutral tone?
T: just a sec
http://www.ocf.berkeley.edu/~jackass/sket ches/headproto2.jpg
A: I'm ... beginning to suspect that your name isn't Jack
T: quick photoshop paint-bucketing
actually it is
sort of
A): I mean from the url :-)
T: I figured as much
A: once again jokingly
T: :-)
I'm not averse to humor
sorry to butcher your sketch with the paint bucket
it's not a precision tool
A: eh, you just gave me a color scheme I needed. that you grok photoshop has made it easier for me.
T: glad to be of service :-)
A: same here--I hope the comic, posters, intro fmvs and whatever else help spread the word about VS
deserves a lot more attention than it has
 btw, there's talk about a logo every now and then but nothing concrete
T: My thoughts on a new tag line-esque phrase go as follows:
A: (ah, you must've read my post on the subject already.) ?
T: "Before Success must come Survival"
Vegastrike.
"The void beckons"
A: somehow I suspected "tales of the void" not to be final
T: Legends of the Void sounds like a name for an expansion pack :-)
"Play all the NPC characters you could only dream of being in the original!"
hehe
I can see the advert 
with the above tag line
or rather, above above
Word on black "Before Success..."
Images of wealth, splendor, glory
word on black "Must come survival"
images and pounding sounds of labored breathing and heartbeats, screams of pain and battle cries, quick cuts of interspersed combat
A: don't mind me if I take this verbatim and screenplay an intro around it!
T: Sound cuts, spoken "And you are very small" Show EVA suit working on repairing battlecruiser in space dock, slow zoom out
Cue logo 
A: which we are in dire need of
T: not sure what sounds to put here, something ambient, or maybe some voices in the background through static, or maybe.. not sure
pause for a couple seconds.
text and voice-over "The Void Beckons" (slight pause) between Void and Beckons in the speech, but not the text
A: background music: classical (such as Holszt's Planets), cosmic new age a-la Jarre, rock, VS theme...?
T: different for the different emphasis points
something classical would fit well for the success part
A: Hmmm, something baroque?
T: but the music the survival part should be percussion driven chase music, agressive, raging, scared all at once
brash, violent and jagged
no music at all for the pan out in space
conspicuous silence after the previous section
not sure what to do for the end music-wise
A: I suggest not music, but an undefined sound of "unknown aliens out there", sort of a low harmonic with just a bit of a shrill halo, or something along those lines
as if there was something mystical about this "call of the void"
T: or a low harmonic with a muted pulsing beat
if it wasn't "the void" I'd go for softly whistling wind :-)
something to play around with, for sure
A: you know, a quantized wind effect could work there too
aka "electronic wind", just manipulated for a more subtle effect
T: I can talk about the Klk'k some, if you're willing to deal with some ideas not yet full rendered from mind to either page or word
A: of course, and if you wish, my pencil is in service of helping ideas take shape--I could try to approximate something from your description
so you can refine, I try again, and so on
T: that was indeed my hope :-)
A: great!
T: so, here was my first attempt to draw a Klk'k skull http://www.ocf.berkeley.edu/~jackass/sket ches/klk'k-skullbones.png
and here's a poorly done profile (head is still skeletal here) http://www.ocf.berkeley.edu/~jackass/sket ches/klk'ksketch.png
and the ankle isn't right
but.. it's a start
average height range 4.5 to 5.5 feet
long, largish feet, reverse jointed knees, large easily splayed hips, excellent jumpers, thick, disproportionally long, muscular legs in proportion to generally much skinnier tops
wide, short, fair lengthed, back-bottom flanging head, with back facing nostrils at the rear base between neck and jawbone
jaw is wider than head on each side. because of angle, more teeth on bottom jaw than on top, bottom jaw teeth face slightly in, top jaw teeth face slightly out
mouth contains 2 tongues. has no connection to air passageway. has resonant "click" cavity in front top of mouth, behind and below the rear of the bone ridges forming the eye sockets
eyes are large and wide set, and the sockets prononouced in their bonyness
ears, such as they are, are 2 long curved ovoids, extending forward from just behind and below the top of the jaw joint to somewhat even with the top of the jaw joint and in front of the joint. Mostly flush with the skull, each ear is a series of cartilige-analog ridges protruding slightly out from the side of the head to focus sound into a central shallow curved trough, the bottom of which has numerous tiny folds of hair-lined skin atop a drum
smooth, hairless, thick, leathery skin
running from green-brown to brown-green
two thumbs on each hand, four fingers between
hands hang at about knee level
 as noted, nasal passages are at the rear of the head, connecting to a common tube inside the neck and providing the sole breathing route
the real trick with these guys is to not make them look like little green men :-\
A: ah, leave that to me
:-)
T: their anthropomorphic features got them in serious trouble with the Lightbearers
the Lightbearers saw it as a mockery of the human form and took religious affront
A: I'll give them enough personality to upset a KKK
T: you can see my inability to use perspective well here: http://www.ocf.berkeley.edu/~jackass/sket ches/klk'k-emblem.png
the center object is supposed to be a Klk'k horn
with dual nasal inputs that must curve around to the back
A: ah, you mentioned that this also goes on Deucalion's arm over his fighting training marks
T: but the forward tube and bell got reial wierded out
well, just the weapon
not the horn
the weapon through the horn is the Klk'k emblem
 the (notably sheathed) weapon :-)
A: bladed, shooter, spear, multiuse?
T: it's an old traditional weapon
bladed on each end
A: makes me imagine a trident
T: no, it's a single blade on each side
looks more like a demented double-oar
the blade is wide and can cut on all three edges
but is not blindingly sharp
except on the outermost edge
A: beveled near the edges?
T: yes
the sides are for cleaving, as with an axe
the juts are for piercing, because of angle still somewhere between an axe and a spear, and the outer edge is for slashing, or thrusting, but not against a heavily armored target
or rather, doing so may dull or chip the outer edge somewhat, as it is more heavily beveled
hitting with the side works for bludgeoning, but it's not aerodynamic that way
the "wooden" ones are semi-functional oars
although metal versions are far too over-heavy for that task
the truly traditional incarnations, carved from the (human termed) Obelisk tree
are better suited for being multitasked as oars and spades
ultra-modern incarnations are much lighter, and forsake the traditional cleaving approach for having all three sides terminate in monomolecular thick blades
tranforming cleaving into cutting
A: quite deadly
T: no one uses that in sport matches :-)
they use the Obelisk "wood" ones
monomolecular spade works well as an entrenching tool too ;-)
well, not that well really
cuts well, but doesn't carry dirt all that well
strength is all running in the wrong direction
anyway... ummm
that's a start on the Klk'k
any obvious questions right off the bat?
A: quite enough actually
yes, although on a different subject
T: oh, one more thing, lips extend back along jaws, for very wide grins :-)
have to really, at the angles the jaw will be moving at

<conversation wanders>


A: you've seen the new fixers. what's you opinion? I know that those are only partially similar to neoprene suits :-)
T: I like them
the externality of the small features is certainly appropriate for civilian suits, and most military ones too
the only thing I have any reservations about is the head/helmet for the proposed confed pilot
A: that's not the helmet, but go ahead
T: ok, that's what I was checking :-)
good
A: I haven't drawn any helmets now that I think about it
T: makes sense really
they're on the station
A: yup
T: most people would take them off
A: I'll use that as my excuse :-)
T: so, what then exactly {\em is} the pilot wearing on his head :-)
(the confed one)
A: combination head-gear (additional data feeds for the goggles), audio equipment, an additional level of cushioning, whatnot
T: sounds good
makes sense for most of the human groups :-)
but we can fairly safely assume that most of the pilots working directly for the confederation rather than in forces assigned to the confed are Purists or LIHW of fairly equivalent stock
A: well, could the headgear also be an option for those not wanting a socket in their head?
T: it's the only option for those not wanting a socket in their head ;-)
ell, excepting getting your head removed :-)
(but we'll ignore mechanist military pilots for the moment)
but yeah, you need to get the data one way or the other
A: the exoskeletal bits I imagined as meant to preserve the body from being bent grossly out of shape in high Gs or collisions--self-activating like airbags, maybe with some intertial dampening in them? of course the life-support layer is still below
some sort of support frame. is that still valid within VS techs?
T: no inertial dampening in the suit. it's generated externally to the pilot.
too big to carry around
A: so I need another explanation for the exo bits
T: so - two things
1. conduits
2. old-fashioned structural support for high-Gs (not enough to save you in an accident, but enough to make the flight more comfortable)
i.e inertial dampening is not a perfect thing
A: that'd explain the pilot having so much of it
T: especially the conduit issue,
you need a stable structure for things to flow through
A: how about multifunctional? both 1 and 2, and maybe some additional functions.
T: well, that's what I was implying :-)
A: ah, ok
has anyone come up with an explanation of VS's spatial distortion technology?
T: the details on how it's remarkably efficient to be able to nearly arbitrarily alter the geometry of local space?
 or "how do shields stop X"
A: the former
T: not really, and I've intentionally not pursued it, because I believe that, once you've decided that something is *magic* then it's best to not try to explain it until you really have to
now, for questions like "how do shields work" or "how does SPEC work" that seem related, those I've pondered
the underlying ability to be able to manipulate space-time that fuels both of those (among other things) I haven't gone into details for
that answer the question?
A: sure, amply so
oh, before I leave: should pirates have melee weapons showing/hidden in their apparel?
I was planning on a suit for them made for the show: barbaric spikes, armor plates, the works
T: might work for the more eccentric special chars, but in general, one would imagine a pirate would keep a lower profile in most of civilized space
A: so concealment it is
T: the primary non-ornamental melee weapons would likely be a shock-stick and the old-faithful sharp-pointy-thing
the sharp pointy thing purely for stabbing people in the back / cutting throats with
after all, bringing anything heavy enough to breach a hull is not going to be appreciated by security
but if you want to gut each other in an out of the way corridor... so be it
A: clothes to conceal small to medium size weapons, a hint of seclusion
T: yup.
because the bounty hunters aren't there just for looks :-)
and they're likely licensed to carry light weapons :-)
(of the ranged variety)
A: ah, so the brutish pirate thoughs that didn't have a knack for subtlety are extinct now
T: or have fled to more forgiving regions of space
Forsaken and Uln territories should have some ... interesting characters
A: ah, the stories they could tell :-) well, I'm off

A: there's not much data on the klk'k, and I'd rather not use too much creative freedom (most likely my picture of the klk'k wouldn't be very consistent)
I have your data on combat ranks, some on biology (sketches), mating habits and clothing
but nothing to really 'picture them' physically in my mind. how do you 'see' them?
 do they move slowly and heavily, or quickly like small reptilians and birds? do their skins resemble a lizard's, or a mammal's?
T: not quite either
more leathery, but not scaled
smooth
Moving freely, but with the constant potential for a quick, jerking, jump
Imagine a dignified dancer, but with too many cups of coffee and convinced he's being stalked
A: he he, that's very graphic :-)
T: the leg angles mean that they have to strut a bit to walk
A: so the legs extend backwards and sideways
T much more back than sideways, but far more splayed than ours are
I often think of them as creatures that came down from the trees, not like our ancestors, onto a savannah, but into a bayou
A: a bayou planet? I like these guys
T: planet, no
original habitat, yes
A: ok
T a common one as well
but not ... ubiquitous
that would be boring :-)
well, or dagobah :-)
only with more sunlight
I guess dagobah was more swamp than bayou
A: if we ever go into production design for this kind of thing, noz and I are gonna have a field day
T: :-)
A: I see a sort of 'lip' in the klk'k skull. what's it for?
T: which pic?
A: side view
heh, actually that seems to be the ocular cavity
T: yeah, somewhat pronounced ocular ridges
A: any special ability/feature associated?
T: they have no nose, so the first thing to hit them in the face hits them in the eyes otherwise
A: so how do they breathe? similar to aera?
T: no
A: thank you
T: rear facing dual nostrils
note the splayed jaw
A: swimmers?
T: they come up under and behind
capable swimmers, not ambphibious, but hydrophilic
A: very long feet, too. clawed, membraned?
T: the foot length is mostly jumping leverage, but it does make a somewhat decent paddle.
A: fast runners?
T: not incredibly - at least for distance
short stubby claw/nails on the feet, degenerated tree grippers.
A: of course. if the length of the feet was arched, then I would believe them to be sprinters.
A: I'm thinking the arms should be a bit bigger in girth than I drew them, but length I'm ok with
not too much though.
A: the double opposable thumb did make me curious
T dextrous
good strong grip
A: more than us I take?
T: A bit, the second thumb doesn't buy that much without additional fingers
but they'd have an easier time opening child-proofed medicine containers with one hand
basically, you can get grip+manipulate with one hand, but that doesn't make the manipulate part better
now, when just holding something, they'd have better stability
A: gotcha. just one eye though? no chance for independent maneuvering on each arm?
T: no, two eyes
look at the top view of the skull
A: anyway. independent sockets, of course? they can be inset in the ridge
T: yah
two distinct ridges actually
A: ah, I see now
must have confused yours with other stuff
T: jack.art-skill.clarity--
;-)
A: no, no, they're clearly distinct sockets, you're right
T: I think a key thing to keep in mind is the Klk'k have to be sufficiently anthromorphic that the Light-Bearers were infuriated by them, but never anything close to pity :-)

\subsection{A conversation on baselines, models, and playability}

P: I am pretty interested in making sure the player has some sensible ships to buy, though.
The way it is now, all the ships to buy are hardcore fighters, shitty, or giant freighters.
T: sure, but that's as much an issue of not having all the models we want as what each model we have should do
P: there's no middle ground, and no progression.
I suppose...
but making a good game is important too.
T: I agree
P: It doesn't really make sense to keep all the top quality models locked up.
I pretty much said "shit, we don't have an Llama +1 s"
T: (I've often been tempted to dump a few hundred place-holder lines into the units.csv and just assign a box as the model... but I thought better of it ;-) )
P: "what model looks good, and could fit some cargo in?"
"ahh, the admonisher. cool"
T: so you think the plowshare is too much of a jump from the llama?
P: I do. Well, I mean it is definitely close to the llama! but look-wise it is more towards the cargo-cargo side of things.
what if the player wants to keep on the fighter-cargo route?
a balanced one.
We really don't have enough good models to go around... that's the root of the problem.
So until that happens, I think we should try and find a creative way to satisfy the needs of the game, while keeping the integrity of the universe.
Maybe by double using models, massaging some things around a bit...
T: I'm having a flashback to "red slime, green slime, blue slime!"
P: a necessary evil...
some of the best games have done that.
T: watch out for them palatte shifted ones ;-)
P: Those are always more dangerous, yes.
The new upgrade thing lets players pretty much decide for themselves what they want the ship to be. Aside from mass, cargo, and flying characteristics.
So just massaging some of the admonisher's (or any ship's) core stats around so it could help fill a different role wouldn't really kill anyone.
you know what i'm saying?
T: well, it isn't really a civilian ship to begin with though, so I'd prioritize making sure that the milspec versions made sense.
I think I see what you're saying - give them lots of .blanks, and they'll build whatever they want
P: yeah. a lot of .blanks with a lot of variation in quality, shape, and color. and it should all fall into place.
I'll fidget the admonisher back in the original direction
T:so, as I said, I'm personally tempted to go the placeholder route myself
rather than the co-opting route
but either way the root problem is stil the same - insufficient model capital
P: I honestly don't see what that would add to the game.
yeah
T: it's hard to make the universe piecemeal
so, the ship I think that most suits what you were looking for would be the Forsaken's Scarab
but we don't have one
The Kafka is unarmed, the Reindeer is even more towards the plowshare side, and the only other things that come even remotely close are intended for roles as orbital cargo landers to provide interesting traffic for stations
so... 
yeah.. not enough models
there are other things that we've been "making do with" for some time as well.
the Schroedinger is a scout, not an interceptor, and so on 
The Goddard model will eventually be moved to something else when I manage to commission something that looks more... Andolian.
bleh.
So, the question from my perspective is, is it better to take an existing model+ship pair, and pull it away from it's intended and eventual role, and then rebalance again when we get more models and can shift around, or to steal/copy a model from an existing model+ship pair and use it to incarnate a ship whose role we desire but whose model we lack
P: number two.
duplicating art is lame...
T: so you mean number 1 then
P: but!
aha. but!
i mind farted for a second there.
anyways
but! it's an accepted practice to do it. 
even in professional, commercial computer games a bit.
or... i don't know.
i'm out of brain juice
T: well, what seems to make some sense to me is to do the balancing based on ships that fill the role you're actually looking for, and then, if nothing else, we can always harangue artists for those most key unmodelled ships before each release and not spawn the rest, duplicate art, or decomission underused models for temporary assignment to more important ships (even if the models don't fit so well :-( )
but, whatever is chosen, I think balancing based on the ships we'll eventually have reduces duplication of work and the "don't have a model issue" can always be pushed out until we actually have a release
P: that's reasonable.
T: besides, maybe having a bunch of placeholders in CVS will motivate artists to help make some more art for the things they keep having to see ;-) (or not)
ok, so with that in mind - how about we try to perhaps make sure we have the set of ships we really want to be working with :-)
P: that seems like a step in the right direction.
T: so I updated the brief descriptions to reflect the roles of the 8 selected ships. Only a couple of minor changes in emphasis (excluding the major change to the admonisher)
P: yeah, i saw the message.
T: so
llama: Venerable LIHW built armed light cargo shuttle
redeemer: Aging Luddite insystem fighter
admonisher: Purist light assault craft
plowshare: A top of the line Purist cargo shuttle
pacifier: Aging Purist heavy fighter/bomber
gawain: Main line High-Born interceptor
lancelot: High-Born heavy superiority fighter
dostoevsky: Andolian Protectorate mass production superiority fighter

P: those are better descriptions.
T: Thanks :-)
So one thing I want to make sure we have the same semantics about is what "interceptor" means
P: what's your take on the word?
T: designed to take out incoming assault craft
generally sacrificing durability, and enduring firepower etc. for accel and heavy first strike capability in order to be sure to engage and destroy hostiles outside of the range at which they can effectively damage friendly targets of importance
ot, however, a synonym for "agile, fragile light-superiority fighter"
et tu?
P: yeah, pretty much. interceptors intercept. that's what they do.
T: the VS ai priorities currently reflect this :-)
okiedoke. Just something I wanted to check because I've seen games where that really wasn't the case :-P

