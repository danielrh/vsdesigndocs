\label{chapt:portfolios}
The following sections give a holistic overview of various key groups
as the exist or existed circa 3276 CE (i.e. the UtCS period, see
~\ref{timeline:UtCS} and~\ref{exptimeline:UtCS}), combining factional, species, and aesthetic
information in a single place, from an omniscient viewpoint. These
portfolios are primarily designed to give insight and direction to
artists and other content creators. Much of the information here may
be duplicated elsewhere. There may be cases where the omniscient
viewpoint and in-game viewpoint disagree. Such disagreements are
likely intentional, reflecting misconceptions or ignorance on the
part of the extant groups. In cases where the disagreements seem more
ambiguous or unlikely in intent, please feel free to direct questions
as necessary.

\section{Cross-Portfolio Information}
\subsection{Vessel Overview}
\begin{itemize}
\item Engines

Retro thrusters, retro thrusters, retro thrusters, retro thrusters!
(to the tune of ``Developers, developers, developers!'')

If your craft is: A) not Rlaan manufactured and B) more than 5 meters
long it probably needs retro thrusters.

Now, the retro thrusters and maneuvering thrusters don't need to be
necessarily as pronounced as the rear engines (giving you some
artistic license in defining a visual front/back) but they NEED TO
HAVE VISUAL INDICATORS OF THEIR EXISTENCE. For some ships this is
mainly a texturing question. For larger vessels, this will necessarily
influence your model design.  

\item Radiators

One of the biggest problems in space is heat dissipation. VS
spacecraft have a lot of heat to get rid of, so they're going to need
radiators. While, as with retro and maneuvering engines, artistic
license is granted in how much of the ship's surface area is going to
be radiator dominated, the same need for visual indicators of
radiators exists. If you have fins, you should probably put a radiator
texture on them, because they're almost certainly not for aerodynamic
flight.  

\item Internals

VS ships aren't, in general, intended to be very dense. This is more
true for larger craft than strike craft. In general, the idea for VS
ships is that the surface-area to internals ratio should be pretty
high. The bigger ships in VS can be seen as big shells of armor and
radiators around many, smaller, internal components. Moreover, a lot
of the internal components would actually be plumbing, of the coolant
(I always had a soft spot for gallium alloys, but no decisions are
finalized on what exactly the coolant is) circulating variety, running
between the reactors/engines (the engines being fusion reactors
themselves) and other heat-generating components (e.g. weapons). The
rest of the space being taken up by insulation (vacuum is pretty good)
and physical shielding to keep the crewed and other sensitive parts of
vessel safe from the engines/reactors, and other hazardous portions of
the ship.  

\item Atmospheric flight

Most spaceships in the VS universe, even if capable of accelerating to
escape velocity, are not designed for atmospheric flight. Some are. If
its role description mentions ``Aerospace'' or ``lander'', ``puddle
jumper'', ``dropship'', ``ground support'', or ``orbital'' then it's a good
bet it has an atmosphere-friendly design. If not - build a spaceship,
not an airplane.

That said, some smaller craft that are not atmosphere-friendly will
still be atmosphere capable. The more developed the infrastructure of
the group using/designing the ship, the more likely they don't intend
for it to ever see atmospheric use on an inhabited planet - this is
what docking stations, space-elevators, orbital shuttles, and other
infrastructure exists for. In fact, (once we get planetary structures
squared away a bit better), the really industrialized planets will
probably start firing on you if you attempt to land in anything larger
than a pinnace/dispatch craft, and even then only with appropriate
permission. On the flip side, if it's a Forsaken ship, or a Luddite
craft, then it probably spends a lot more time visiting places without
well-developed orbital infrastructure and may need to land on
occasion. Also take into account the age of the design you're
considering. Much of the VS universe as of 3276 is post-frontier, but
a lot of places don't have to think back too far to remember when they
weren't. Use your better judgment.

If the ship is very large, even if it is atmosphere-capable, it is
almost certainly {\em not atmosphere friendly}, unless perhaps it's
something very esoteric (atmosphere skimmer for terraforming project
or some such). In general, if the ship is very large, it's best to
assume it's not even atmosphere capable unless there's a very good
reason for it. Large vessels will make use of shuttles and dispatch
craft to take care of business on the ground.

\end{itemize}


\section{Portfolio: The Aerans}

\section{Portfolio: The Andolians}

\section{Portfolio: The Andolians}

\section{Portfolio: The Andolian Protectorate}
Ships overview for all groups (Work In Progress): \href{http://vegastrike.sourceforge.net/wiki/Artstyle\_guide:Overview\_Guide}{Ship Overview} \\
Art-style Guide (Work In Progress): \href{http://vegastrike.sourceforge.net/wiki/Artstyle\_guide:Andolian\_Protectorate}{Artstyle Guide} \\
Species overviews: \href{http://vegastrike.sourceforge.net/wiki/Species:Humanity}{Species:Humanity} \\
 \href{http://vegastrike.sourceforge.net/wiki/Species:Klk\'k}{Species:Klk'k} \\
 \href{http://vegastrike.sourceforge.net/wiki/Species:Purth}{Species:Klk'k} \\

\subsection{Origin}
\begin{itemize}
\item Gravity: 

\item Atmosphere: 

\item Primary liquid bodies: 

\item Average temperature of homeworld (pre-industrialization):

\item Sun: 

\item Primary challenges (pre-industrialization): 
\end{itemize}

%ORIGIN COMMENTS GO HERE

\subsubsection{Habitat}

\subsection{Physical}
\begin{itemize}
\item Dimensions: 

\item Mass: 

\item Skeletal system: 

\item Major divisions: 

\item Senses: 

\item Visual acuity: 

\item Chemosense: 

\item Locomotion: 

\item Manipulators: 

\item Textural appearance: 
\end{itemize}

%PHYSICAL COMMENTS GO HERE

\subsection{Mental}


\subsection{Technological}
\begin{itemize}
\item Tech: 


\item Weapons:

\item Tactics:
\begin{itemize}
\item    Small groups: 
\item    Large groups/Fleets: 
\end{itemize}


\item Installations:


\end{itemize}

\subsection{Culture}

\subsubsection{Factions and Organizational Groups}
Listed below are noteworthy Aeran sub-factions and organizational groups: 
\begin{itemize}
\item FACTIONS GO HERE
\end{itemize}

\subsubsection{Religion}

\subsubsection{Cultural Aesthetics}

\subsection{Writing, numbers, and insignia}

\subsection{Faction: PRIMARY FACTION}
%Faction data 
%Aera 
%Species 	Aera 
%Homeworld (Origin) 	Aeneth 
%Capital 	Aeneth 


\subsubsection{A Brief History of the PRIMARY FACTION}


\subsubsection{Development}

\subsubsection{Culture}

\subsubsection{Organization}

\subsection{Faction: OTHER FACTIONS}

%Faction data 
%Merchant Marines 
%Species 	Aera 
%Homeworld (Origin) 	Aeneth 
%Capital 	Aeneth 


\subsection{Vessels}

\subsubsection{Style Overview}
\begin{itemize}

\item Primary distinguishing color ranges: 

\item Common accent colors:

\item Primary lighting color:

\item Frequently visible: 

\item Rarely visible:

\item Seen inside, but not out: 

\item Moving parts(non-turret): 

\item Capital vs. light craft: 

\end{itemize}

\subsubsection{Surface features of large vessels}

\subsubsection{Small things found on the hull of a large  vessel}
\begin{itemize}
\item Service/Maintenance hatches
\end{itemize}
{\it Somewhat larger things found on the hull of a large Aeran vessel:}
\begin{itemize}
\item Escape pod launcher ports
\end{itemize}
{\it Yet larger things ... :}
\begin{itemize}
\item Pinnace/lander launch bay (non-carrier vessels)
\end{itemize}

\subsubsection{Listing of vessels}

\begin{itemize}
\item \href{http://vegastrike.sourceforge.net/wiki/Vessel:FOO}{FOO:} 

Existing concept art is not particularly canonical. Please redesign.

\end{itemize}

% LocalWords:  Aerans Aeran Artstyle Aera Pinnace Acrotatus Agasicles Agesilaus
% LocalWords:  Agesipolis Agis Alcmenes Anaxander Wiki Anaxandridas Rlaan Areus
% LocalWords:  Anaxidamus Ariston Charillus Cleombrotus Cleomenes Demaratus Uln
% LocalWords:  Theopompus Dorissus Echestratus Eurycratides Eurypon Leons Bzbr
% LocalWords:  Nicander Pausanias Pleistarchus UnAeranned Pleistoanax Andolian
% LocalWords:  Spaceborn's MacGyver Polydectes Polydorus EVAs Procles Prytanis
% LocalWords:  Soos Teleclus terraforming Aenethforming Zeuxidamus homeworld
% LocalWords:  coreward chemoreception Klk'k Aeneth ecologies Shmrn

\section{Portfolio: The Bzbr}
Ships overview for all groups (Work In Progress): \href{http://vegastrike.sourceforge.net/wiki/Artstyle\_guide:Overview\_Guide}{Ship Overview} \\
Art-style Guide (Work In Progress): \href{http://vegastrike.sourceforge.net/wiki/Artstyle\_guide:Bzbr}{Artstyle Guide} \\
Species overview: \href{http://vegastrike.sourceforge.net/wiki/Species:Humanity}{Species:Bzbr} \\

\subsection{Origin}
\begin{itemize}
\item Gravity: 

\item Atmosphere: 

\item Primary liquid bodies: 

\item Average temperature of homeworld (pre-industrialization):

\item Sun: 

\item Primary challenges (pre-industrialization): 
\end{itemize}

%ORIGIN COMMENTS GO HERE

\subsubsection{Habitat}

\subsection{Physical}
\begin{itemize}
\item Dimensions: 

\item Mass: 

\item Skeletal system: 

\item Major divisions: 

\item Senses: 

\item Visual acuity: 

\item Chemosense: 

\item Locomotion: 

\item Manipulators: 

\item Textural appearance: 
\end{itemize}

%PHYSICAL COMMENTS GO HERE

\subsection{Mental}


\subsection{Technological}
\begin{itemize}
\item Tech: 


\item Weapons:

\item Tactics:
\begin{itemize}
\item    Small groups: 
\item    Large groups/Fleets: 
\end{itemize}


\item Installations:


\end{itemize}

\subsection{Culture}

\subsubsection{Factions and Organizational Groups}
Listed below are noteworthy Aeran sub-factions and organizational groups: 
\begin{itemize}
\item FACTIONS GO HERE
\end{itemize}

\subsubsection{Religion}

\subsubsection{Cultural Aesthetics}

\subsection{Writing, numbers, and insignia}

\subsection{Faction: PRIMARY FACTION}
%Faction data 
%Aera 
%Species 	Aera 
%Homeworld (Origin) 	Aeneth 
%Capital 	Aeneth 


\subsubsection{A Brief History of the PRIMARY FACTION}


\subsubsection{Development}

\subsubsection{Culture}

\subsubsection{Organization}

\subsection{Faction: OTHER FACTIONS}

%Faction data 
%Merchant Marines 
%Species 	Aera 
%Homeworld (Origin) 	Aeneth 
%Capital 	Aeneth 


\subsection{Vessels}

\subsubsection{Style Overview}
\begin{itemize}

\item Primary distinguishing color ranges: 

\item Common accent colors:

\item Primary lighting color:

\item Frequently visible: 

\item Rarely visible:

\item Seen inside, but not out: 

\item Moving parts(non-turret): 

\item Capital vs. light craft: 

\end{itemize}

\subsubsection{Surface features of large vessels}

\subsubsection{Small things found on the hull of a large  vessel}
\begin{itemize}
\item Service/Maintenance hatches
\end{itemize}
{\it Somewhat larger things found on the hull of a large Aeran vessel:}
\begin{itemize}
\item Escape pod launcher ports
\end{itemize}
{\it Yet larger things ... :}
\begin{itemize}
\item Pinnace/lander launch bay (non-carrier vessels)
\end{itemize}

\subsubsection{Listing of vessels}

\begin{itemize}
\item \href{http://vegastrike.sourceforge.net/wiki/Vessel:FOO}{FOO:} 

Existing concept art is not particularly canonical. Please redesign.

\end{itemize}

% LocalWords:  Aerans Aeran Artstyle Aera Pinnace Acrotatus Agasicles Agesilaus
% LocalWords:  Agesipolis Agis Alcmenes Anaxander Wiki Anaxandridas Rlaan Areus
% LocalWords:  Anaxidamus Ariston Charillus Cleombrotus Cleomenes Demaratus Uln
% LocalWords:  Theopompus Dorissus Echestratus Eurycratides Eurypon Leons Bzbr
% LocalWords:  Nicander Pausanias Pleistarchus UnAeranned Pleistoanax Andolian
% LocalWords:  Spaceborn's MacGyver Polydectes Polydorus EVAs Procles Prytanis
% LocalWords:  Soos Teleclus terraforming Aenethforming Zeuxidamus homeworld
% LocalWords:  coreward chemoreception Klk'k Aeneth ecologies Shmrn

\section{Portfolio: The Confederation of Inhabited Worlds}
Ships overview for all groups (Work In Progress): \href{http://vegastrike.sourceforge.net/wiki/Artstyle\_guide:Overview\_Guide}{Ship Overview} \\
Art-style Guide (Work In Progress): \href{http://vegastrike.sourceforge.net/wiki/Artstyle\_guide:Confederation}{Artstyle Guide} \\
Species overviews: \href{http://vegastrike.sourceforge.net/wiki/Species:Humanity}{Species:Humanity} \\
\href{http://vegastrike.sourceforge.net/wiki/Species:Klk\'k}{Species:Klk\'k} \\
\href{http://vegastrike.sourceforge.net/wiki/Species:Purth}{Species:Purth} \\
\href{http://vegastrike.sourceforge.net/wiki/Species:Mishtali}{Species:Mishtali} \\
\href{http://vegastrike.sourceforge.net/wiki/Species:Dgn}{Species:Dgn} \\

\subsection{Origin}
\begin{itemize}
\item Gravity: 

\item Atmosphere: 

\item Primary liquid bodies: 

\item Average temperature of homeworld (pre-industrialization):

\item Sun: 

\item Primary challenges (pre-industrialization): 
\end{itemize}

%ORIGIN COMMENTS GO HERE

\subsubsection{Habitat}

\subsection{Physical}
\begin{itemize}
\item Dimensions: 

\item Mass: 

\item Skeletal system: 

\item Major divisions: 

\item Senses: 

\item Visual acuity: 

\item Chemosense: 

\item Locomotion: 

\item Manipulators: 

\item Textural appearance: 
\end{itemize}

%PHYSICAL COMMENTS GO HERE

\subsection{Mental}


\subsection{Technological}
\begin{itemize}
\item Tech: 


\item Weapons:

\item Tactics:
\begin{itemize}
\item    Small groups: 
\item    Large groups/Fleets: 
\end{itemize}


\item Installations:


\end{itemize}

\subsection{Culture}

\subsubsection{Factions and Organizational Groups}
Listed below are noteworthy Aeran sub-factions and organizational groups: 
\begin{itemize}
\item FACTIONS GO HERE
\end{itemize}

\subsubsection{Religion}

\subsubsection{Cultural Aesthetics}

\subsection{Writing, numbers, and insignia}

\subsection{Faction: PRIMARY FACTION}
%Faction data 
%Aera 
%Species 	Aera 
%Homeworld (Origin) 	Aeneth 
%Capital 	Aeneth 


\subsubsection{A Brief History of the PRIMARY FACTION}


\subsubsection{Development}

\subsubsection{Culture}

\subsubsection{Organization}

\subsection{Faction: OTHER FACTIONS}

%Faction data 
%Merchant Marines 
%Species 	Aera 
%Homeworld (Origin) 	Aeneth 
%Capital 	Aeneth 


\subsection{Vessels}

\subsubsection{Style Overview}
\begin{itemize}

\item Primary distinguishing color ranges: 

\item Common accent colors:

\item Primary lighting color:

\item Frequently visible: 

\item Rarely visible:

\item Seen inside, but not out: 

\item Moving parts(non-turret): 

\item Capital vs. light craft: 

\end{itemize}

\subsubsection{Surface features of large vessels}

\subsubsection{Small things found on the hull of a large  vessel}
\begin{itemize}
\item Service/Maintenance hatches
\end{itemize}
{\it Somewhat larger things found on the hull of a large Aeran vessel:}
\begin{itemize}
\item Escape pod launcher ports
\end{itemize}
{\it Yet larger things ... :}
\begin{itemize}
\item Pinnace/lander launch bay (non-carrier vessels)
\end{itemize}

\subsubsection{Listing of vessels}

\begin{itemize}
\item \href{http://vegastrike.sourceforge.net/wiki/Vessel:FOO}{FOO:} 

Existing concept art is not particularly canonical. Please redesign.

\end{itemize}

% LocalWords:  Aerans Aeran Artstyle Aera Pinnace Acrotatus Agasicles Agesilaus
% LocalWords:  Agesipolis Agis Alcmenes Anaxander Wiki Anaxandridas Rlaan Areus
% LocalWords:  Anaxidamus Ariston Charillus Cleombrotus Cleomenes Demaratus Uln
% LocalWords:  Theopompus Dorissus Echestratus Eurycratides Eurypon Leons Bzbr
% LocalWords:  Nicander Pausanias Pleistarchus UnAeranned Pleistoanax Andolian
% LocalWords:  Spaceborn's MacGyver Polydectes Polydorus EVAs Procles Prytanis
% LocalWords:  Soos Teleclus terraforming Aenethforming Zeuxidamus homeworld
% LocalWords:  coreward chemoreception Klk'k Aeneth ecologies Shmrn
     
\section{Portfolio: Criminal Elements}
Ships overview for all groups (Work In Progress): \href{http://vegastrike.sourceforge.net/wiki/Artstyle\_guide:Overview\_Guide}{Ship Overview} \\
Art-style Guide (Work In Progress): \href{http://vegastrike.sourceforge.net/wiki/Artstyle\_guide:Pirates}{Artstyle Guide} \\


\subsection{Origin}
\begin{itemize}
\item Gravity: 

\item Atmosphere: 

\item Primary liquid bodies: 

\item Average temperature of homeworld (pre-industrialization):

\item Sun: 

\item Primary challenges (pre-industrialization): 
\end{itemize}

%ORIGIN COMMENTS GO HERE

\subsubsection{Habitat}

\subsection{Physical}
\begin{itemize}
\item Dimensions: 

\item Mass: 

\item Skeletal system: 

\item Major divisions: 

\item Senses: 

\item Visual acuity: 

\item Chemosense: 

\item Locomotion: 

\item Manipulators: 

\item Textural appearance: 
\end{itemize}

%PHYSICAL COMMENTS GO HERE

\subsection{Mental}


\subsection{Technological}
\begin{itemize}
\item Tech: 


\item Weapons:

\item Tactics:
\begin{itemize}
\item    Small groups: 
\item    Large groups/Fleets: 
\end{itemize}


\item Installations:


\end{itemize}

\subsection{Culture}

\subsubsection{Factions and Organizational Groups}
Listed below are noteworthy Aeran sub-factions and organizational groups: 
\begin{itemize}
\item FACTIONS GO HERE
\end{itemize}

\subsubsection{Religion}

\subsubsection{Cultural Aesthetics}

\subsection{Writing, numbers, and insignia}

\subsection{Faction: PRIMARY FACTION}
%Faction data 
%Aera 
%Species 	Aera 
%Homeworld (Origin) 	Aeneth 
%Capital 	Aeneth 


\subsubsection{A Brief History of the PRIMARY FACTION}


\subsubsection{Development}

\subsubsection{Culture}

\subsubsection{Organization}

\subsection{Faction: OTHER FACTIONS}

%Faction data 
%Merchant Marines 
%Species 	Aera 
%Homeworld (Origin) 	Aeneth 
%Capital 	Aeneth 


\subsection{Vessels}

\subsubsection{Style Overview}
\begin{itemize}

\item Primary distinguishing color ranges: 

\item Common accent colors:

\item Primary lighting color:

\item Frequently visible: 

\item Rarely visible:

\item Seen inside, but not out: 

\item Moving parts(non-turret): 

\item Capital vs. light craft: 

\end{itemize}

\subsubsection{Surface features of large vessels}

\subsubsection{Small things found on the hull of a large  vessel}
\begin{itemize}
\item Service/Maintenance hatches
\end{itemize}
{\it Somewhat larger things found on the hull of a large Aeran vessel:}
\begin{itemize}
\item Escape pod launcher ports
\end{itemize}
{\it Yet larger things ... :}
\begin{itemize}
\item Pinnace/lander launch bay (non-carrier vessels)
\end{itemize}

\subsubsection{Listing of vessels}

\begin{itemize}
\item \href{http://vegastrike.sourceforge.net/wiki/Vessel:FOO}{FOO:} 

Existing concept art is not particularly canonical. Please redesign.

\end{itemize}

% LocalWords:  Aerans Aeran Artstyle Aera Pinnace Acrotatus Agasicles Agesilaus
% LocalWords:  Agesipolis Agis Alcmenes Anaxander Wiki Anaxandridas Rlaan Areus
% LocalWords:  Anaxidamus Ariston Charillus Cleombrotus Cleomenes Demaratus Uln
% LocalWords:  Theopompus Dorissus Echestratus Eurycratides Eurypon Leons Bzbr
% LocalWords:  Nicander Pausanias Pleistarchus UnAeranned Pleistoanax Andolian
% LocalWords:  Spaceborn's MacGyver Polydectes Polydorus EVAs Procles Prytanis
% LocalWords:  Soos Teleclus terraforming Aenethforming Zeuxidamus homeworld
% LocalWords:  coreward chemoreception Klk'k Aeneth ecologies Shmrn

\section{Portfolio: The Dgn}
Ships overview for all groups (Work In Progress): \href{http://vegastrike.sourceforge.net/wiki/Artstyle\_guide:Overview\_Guide}{Ship Overview} \\
Art-style Guide (Work In Progress): \href{http://vegastrike.sourceforge.net/wiki/Artstyle\_guide:Dgn}{Artstyle Guide} \\
Species overview: \href{http://vegastrike.sourceforge.net/wiki/Species:Dgn}{Species:Dgn} \\

\subsection{Origin}
\begin{itemize}
\item Gravity: 

\item Atmosphere: 

\item Primary liquid bodies: 

\item Average temperature of homeworld (pre-industrialization):

\item Sun: 

\item Primary challenges (pre-industrialization): 
\end{itemize}

%ORIGIN COMMENTS GO HERE

\subsubsection{Habitat}

\subsection{Physical}
\begin{itemize}
\item Dimensions: 

\item Mass: 

\item Skeletal system: 

\item Major divisions: 

\item Senses: 

\item Visual acuity: 

\item Chemosense: 

\item Locomotion: 

\item Manipulators: 

\item Textural appearance: 
\end{itemize}

%PHYSICAL COMMENTS GO HERE

\subsection{Mental}


\subsection{Technological}
\begin{itemize}
\item Tech: 


\item Weapons:

\item Tactics:
\begin{itemize}
\item    Small groups: 
\item    Large groups/Fleets: 
\end{itemize}


\item Installations:


\end{itemize}

\subsection{Culture}

\subsubsection{Factions and Organizational Groups}
Listed below are noteworthy Aeran sub-factions and organizational groups: 
\begin{itemize}
\item FACTIONS GO HERE
\end{itemize}

\subsubsection{Religion}

\subsubsection{Cultural Aesthetics}

\subsection{Writing, numbers, and insignia}

\subsection{Faction: PRIMARY FACTION}
%Faction data 
%Aera 
%Species 	Aera 
%Homeworld (Origin) 	Aeneth 
%Capital 	Aeneth 


\subsubsection{A Brief History of the PRIMARY FACTION}


\subsubsection{Development}

\subsubsection{Culture}

\subsubsection{Organization}

\subsection{Faction: OTHER FACTIONS}

%Faction data 
%Merchant Marines 
%Species 	Aera 
%Homeworld (Origin) 	Aeneth 
%Capital 	Aeneth 


\subsection{Vessels}

\subsubsection{Style Overview}
\begin{itemize}

\item Primary distinguishing color ranges: 

\item Common accent colors:

\item Primary lighting color:

\item Frequently visible: 

\item Rarely visible:

\item Seen inside, but not out: 

\item Moving parts(non-turret): 

\item Capital vs. light craft: 

\end{itemize}

\subsubsection{Surface features of large vessels}

\subsubsection{Small things found on the hull of a large  vessel}
\begin{itemize}
\item Service/Maintenance hatches
\end{itemize}
{\it Somewhat larger things found on the hull of a large Aeran vessel:}
\begin{itemize}
\item Escape pod launcher ports
\end{itemize}
{\it Yet larger things ... :}
\begin{itemize}
\item Pinnace/lander launch bay (non-carrier vessels)
\end{itemize}

\subsubsection{Listing of vessels}

\begin{itemize}
\item \href{http://vegastrike.sourceforge.net/wiki/Vessel:FOO}{FOO:} 

Existing concept art is not particularly canonical. Please redesign.

\end{itemize}

% LocalWords:  Aerans Aeran Artstyle Aera Pinnace Acrotatus Agasicles Agesilaus
% LocalWords:  Agesipolis Agis Alcmenes Anaxander Wiki Anaxandridas Rlaan Areus
% LocalWords:  Anaxidamus Ariston Charillus Cleombrotus Cleomenes Demaratus Uln
% LocalWords:  Theopompus Dorissus Echestratus Eurycratides Eurypon Leons Bzbr
% LocalWords:  Nicander Pausanias Pleistarchus UnAeranned Pleistoanax Andolian
% LocalWords:  Spaceborn's MacGyver Polydectes Polydorus EVAs Procles Prytanis
% LocalWords:  Soos Teleclus terraforming Aenethforming Zeuxidamus homeworld
% LocalWords:  coreward chemoreception Klk'k Aeneth ecologies Shmrn
             
\section{Portfolio: The Forsaken (Union of Dispossessed Settlers)}
Ships overview for all groups (Work In Progress): \href{http://vegastrike.sourceforge.net/wiki/Artstyle\_guide:Overview\_Guide}{Ship Overview} \\
Art-style Guide (Work In Progress): \href{http://vegastrike.sourceforge.net/wiki/Artstyle\_guide:Forsaken}{Artstyle Guide} \\
Species overview: \href{http://vegastrike.sourceforge.net/wiki/Species:Humanity}{Species:Humanity} \\

\subsection{Origin}
\begin{itemize}
\item Gravity: 

\item Atmosphere: 

\item Primary liquid bodies: 

\item Average temperature of homeworld (pre-industrialization):

\item Sun: 

\item Primary challenges (pre-industrialization): 
\end{itemize}

%ORIGIN COMMENTS GO HERE

\subsubsection{Habitat}

\subsection{Physical}
\begin{itemize}
\item Dimensions: 

\item Mass: 

\item Skeletal system: 

\item Major divisions: 

\item Senses: 

\item Visual acuity: 

\item Chemosense: 

\item Locomotion: 

\item Manipulators: 

\item Textural appearance: 
\end{itemize}

%PHYSICAL COMMENTS GO HERE

\subsection{Mental}


\subsection{Technological}
\begin{itemize}
\item Tech: 


\item Weapons:

\item Tactics:
\begin{itemize}
\item    Small groups: 
\item    Large groups/Fleets: 
\end{itemize}


\item Installations:


\end{itemize}

\subsection{Culture}

\subsubsection{Factions and Organizational Groups}
Listed below are noteworthy Aeran sub-factions and organizational groups: 
\begin{itemize}
\item FACTIONS GO HERE
\end{itemize}

\subsubsection{Religion}

\subsubsection{Cultural Aesthetics}

\subsection{Writing, numbers, and insignia}

\subsection{Faction: PRIMARY FACTION}
%Faction data 
%Aera 
%Species 	Aera 
%Homeworld (Origin) 	Aeneth 
%Capital 	Aeneth 


\subsubsection{A Brief History of the PRIMARY FACTION}


\subsubsection{Development}

\subsubsection{Culture}

\subsubsection{Organization}

\subsection{Faction: OTHER FACTIONS}

%Faction data 
%Merchant Marines 
%Species 	Aera 
%Homeworld (Origin) 	Aeneth 
%Capital 	Aeneth 


\subsection{Vessels}

\subsubsection{Style Overview}
\begin{itemize}

\item Primary distinguishing color ranges: 

\item Common accent colors:

\item Primary lighting color:

\item Frequently visible: 

\item Rarely visible:

\item Seen inside, but not out: 

\item Moving parts(non-turret): 

\item Capital vs. light craft: 

\end{itemize}

\subsubsection{Surface features of large vessels}

\subsubsection{Small things found on the hull of a large  vessel}
\begin{itemize}
\item Service/Maintenance hatches
\end{itemize}
{\it Somewhat larger things found on the hull of a large Aeran vessel:}
\begin{itemize}
\item Escape pod launcher ports
\end{itemize}
{\it Yet larger things ... :}
\begin{itemize}
\item Pinnace/lander launch bay (non-carrier vessels)
\end{itemize}

\subsubsection{Listing of vessels}

\begin{itemize}
\item \href{http://vegastrike.sourceforge.net/wiki/Vessel:FOO}{FOO:} 

Existing concept art is not particularly canonical. Please redesign.

\end{itemize}

% LocalWords:  Aerans Aeran Artstyle Aera Pinnace Acrotatus Agasicles Agesilaus
% LocalWords:  Agesipolis Agis Alcmenes Anaxander Wiki Anaxandridas Rlaan Areus
% LocalWords:  Anaxidamus Ariston Charillus Cleombrotus Cleomenes Demaratus Uln
% LocalWords:  Theopompus Dorissus Echestratus Eurycratides Eurypon Leons Bzbr
% LocalWords:  Nicander Pausanias Pleistarchus UnAeranned Pleistoanax Andolian
% LocalWords:  Spaceborn's MacGyver Polydectes Polydorus EVAs Procles Prytanis
% LocalWords:  Soos Teleclus terraforming Aenethforming Zeuxidamus homeworld
% LocalWords:  coreward chemoreception Klk'k Aeneth ecologies Shmrn
             
\section{Portfolio: The Highborn}
Ships overview for all groups (Work In Progress): \href{http://vegastrike.sourceforge.net/wiki/Artstyle\_guide:Overview\_Guide}{Ship Overview} \\
Art-style Guide (Work In Progress): \href{http://vegastrike.sourceforge.net/wiki/Artstyle\_guide:Highborn}{Artstyle Guide} \\
Species overview: \href{http://vegastrike.sourceforge.net/wiki/Species:Humanity}{Species:Humanity} \\

\subsection{Origin}
\begin{itemize}
\item Gravity: 

\item Atmosphere: 

\item Primary liquid bodies: 

\item Average temperature of homeworld (pre-industrialization):

\item Sun: 

\item Primary challenges (pre-industrialization): 
\end{itemize}

%ORIGIN COMMENTS GO HERE

\subsubsection{Habitat}

\subsection{Physical}
\begin{itemize}
\item Dimensions: 

\item Mass: 

\item Skeletal system: 

\item Major divisions: 

\item Senses: 

\item Visual acuity: 

\item Chemosense: 

\item Locomotion: 

\item Manipulators: 

\item Textural appearance: 
\end{itemize}

%PHYSICAL COMMENTS GO HERE

\subsection{Mental}


\subsection{Technological}
\begin{itemize}
\item Tech: 


\item Weapons:

\item Tactics:
\begin{itemize}
\item    Small groups: 
\item    Large groups/Fleets: 
\end{itemize}


\item Installations:


\end{itemize}

\subsection{Culture}

\subsubsection{Factions and Organizational Groups}
Listed below are noteworthy Aeran sub-factions and organizational groups: 
\begin{itemize}
\item FACTIONS GO HERE
\end{itemize}

\subsubsection{Religion}

\subsubsection{Cultural Aesthetics}

\subsection{Writing, numbers, and insignia}

\subsection{Faction: PRIMARY FACTION}
%Faction data 
%Aera 
%Species 	Aera 
%Homeworld (Origin) 	Aeneth 
%Capital 	Aeneth 


\subsubsection{A Brief History of the PRIMARY FACTION}


\subsubsection{Development}

\subsubsection{Culture}

\subsubsection{Organization}

\subsection{Faction: OTHER FACTIONS}

%Faction data 
%Merchant Marines 
%Species 	Aera 
%Homeworld (Origin) 	Aeneth 
%Capital 	Aeneth 


\subsection{Vessels}

\subsubsection{Style Overview}
\begin{itemize}

\item Primary distinguishing color ranges: 

\item Common accent colors:

\item Primary lighting color:

\item Frequently visible: 

\item Rarely visible:

\item Seen inside, but not out: 

\item Moving parts(non-turret): 

\item Capital vs. light craft: 

\end{itemize}

\subsubsection{Surface features of large vessels}

\subsubsection{Small things found on the hull of a large  vessel}
\begin{itemize}
\item Service/Maintenance hatches
\end{itemize}
{\it Somewhat larger things found on the hull of a large Aeran vessel:}
\begin{itemize}
\item Escape pod launcher ports
\end{itemize}
{\it Yet larger things ... :}
\begin{itemize}
\item Pinnace/lander launch bay (non-carrier vessels)
\end{itemize}

\subsubsection{Listing of vessels}

\begin{itemize}
\item \href{http://vegastrike.sourceforge.net/wiki/Vessel:FOO}{FOO:} 

Existing concept art is not particularly canonical. Please redesign.

\end{itemize}

% LocalWords:  Aerans Aeran Artstyle Aera Pinnace Acrotatus Agasicles Agesilaus
% LocalWords:  Agesipolis Agis Alcmenes Anaxander Wiki Anaxandridas Rlaan Areus
% LocalWords:  Anaxidamus Ariston Charillus Cleombrotus Cleomenes Demaratus Uln
% LocalWords:  Theopompus Dorissus Echestratus Eurycratides Eurypon Leons Bzbr
% LocalWords:  Nicander Pausanias Pleistarchus UnAeranned Pleistoanax Andolian
% LocalWords:  Spaceborn's MacGyver Polydectes Polydorus EVAs Procles Prytanis
% LocalWords:  Soos Teleclus terraforming Aenethforming Zeuxidamus homeworld
% LocalWords:  coreward chemoreception Klk'k Aeneth ecologies Shmrn
     
\section{Portfolio: Homeland Security}
Ships overview for all groups (Work In Progress): \href{http://vegastrike.sourceforge.net/wiki/Artstyle\_guide:Overview\_Guide}{Ship Overview} \\
Art-style Guide (Work In Progress): \href{http://vegastrike.sourceforge.net/wiki/Artstyle\_guide:Homeland\_Security}{Artstyle Guide} \\
Species overview: \href{http://vegastrike.sourceforge.net/wiki/Species:Humanity}{Species:Humanity} \\

\subsection{Origin}
\begin{itemize}
\item Gravity: 

\item Atmosphere: 

\item Primary liquid bodies: 

\item Average temperature of homeworld (pre-industrialization):

\item Sun: 

\item Primary challenges (pre-industrialization): 
\end{itemize}

%ORIGIN COMMENTS GO HERE

\subsubsection{Habitat}

\subsection{Physical}
\begin{itemize}
\item Dimensions: 

\item Mass: 

\item Skeletal system: 

\item Major divisions: 

\item Senses: 

\item Visual acuity: 

\item Chemosense: 

\item Locomotion: 

\item Manipulators: 

\item Textural appearance: 
\end{itemize}

%PHYSICAL COMMENTS GO HERE

\subsection{Mental}


\subsection{Technological}
\begin{itemize}
\item Tech: 


\item Weapons:

\item Tactics:
\begin{itemize}
\item    Small groups: 
\item    Large groups/Fleets: 
\end{itemize}


\item Installations:


\end{itemize}

\subsection{Culture}

\subsubsection{Factions and Organizational Groups}
Listed below are noteworthy Aeran sub-factions and organizational groups: 
\begin{itemize}
\item FACTIONS GO HERE
\end{itemize}

\subsubsection{Religion}

\subsubsection{Cultural Aesthetics}

\subsection{Writing, numbers, and insignia}

\subsection{Faction: PRIMARY FACTION}
%Faction data 
%Aera 
%Species 	Aera 
%Homeworld (Origin) 	Aeneth 
%Capital 	Aeneth 


\subsubsection{A Brief History of the PRIMARY FACTION}


\subsubsection{Development}

\subsubsection{Culture}

\subsubsection{Organization}

\subsection{Faction: OTHER FACTIONS}

%Faction data 
%Merchant Marines 
%Species 	Aera 
%Homeworld (Origin) 	Aeneth 
%Capital 	Aeneth 


\subsection{Vessels}

\subsubsection{Style Overview}
\begin{itemize}

\item Primary distinguishing color ranges: 

\item Common accent colors:

\item Primary lighting color:

\item Frequently visible: 

\item Rarely visible:

\item Seen inside, but not out: 

\item Moving parts(non-turret): 

\item Capital vs. light craft: 

\end{itemize}

\subsubsection{Surface features of large vessels}

\subsubsection{Small things found on the hull of a large  vessel}
\begin{itemize}
\item Service/Maintenance hatches
\end{itemize}
{\it Somewhat larger things found on the hull of a large Aeran vessel:}
\begin{itemize}
\item Escape pod launcher ports
\end{itemize}
{\it Yet larger things ... :}
\begin{itemize}
\item Pinnace/lander launch bay (non-carrier vessels)
\end{itemize}

\subsubsection{Listing of vessels}

\begin{itemize}
\item \href{http://vegastrike.sourceforge.net/wiki/Vessel:FOO}{FOO:} 

Existing concept art is not particularly canonical. Please redesign.

\end{itemize}

% LocalWords:  Aerans Aeran Artstyle Aera Pinnace Acrotatus Agasicles Agesilaus
% LocalWords:  Agesipolis Agis Alcmenes Anaxander Wiki Anaxandridas Rlaan Areus
% LocalWords:  Anaxidamus Ariston Charillus Cleombrotus Cleomenes Demaratus Uln
% LocalWords:  Theopompus Dorissus Echestratus Eurycratides Eurypon Leons Bzbr
% LocalWords:  Nicander Pausanias Pleistarchus UnAeranned Pleistoanax Andolian
% LocalWords:  Spaceborn's MacGyver Polydectes Polydorus EVAs Procles Prytanis
% LocalWords:  Soos Teleclus terraforming Aenethforming Zeuxidamus homeworld
% LocalWords:  coreward chemoreception Klk'k Aeneth ecologies Shmrn

\section{Portfolio: The Bounty Hunter's Guild}
Ships overview for all groups (Work In Progress): \href{http://vegastrike.sourceforge.net/wiki/Artstyle\_guide:Overview\_Guide}{Ship Overview} \\
Art-style Guide (Work In Progress): \href{http://vegastrike.sourceforge.net/wiki/Artstyle\_guide:Hunters}{Artstyle Guide} \\
Species overview: \href{http://vegastrike.sourceforge.net/wiki/Species:Humanity}{Species:Humanity} \\

\subsection{Origin}
\begin{itemize}
\item Gravity: 

\item Atmosphere: 

\item Primary liquid bodies: 

\item Average temperature of homeworld (pre-industrialization):

\item Sun: 

\item Primary challenges (pre-industrialization): 
\end{itemize}

%ORIGIN COMMENTS GO HERE

\subsubsection{Habitat}

\subsection{Physical}
\begin{itemize}
\item Dimensions: 

\item Mass: 

\item Skeletal system: 

\item Major divisions: 

\item Senses: 

\item Visual acuity: 

\item Chemosense: 

\item Locomotion: 

\item Manipulators: 

\item Textural appearance: 
\end{itemize}

%PHYSICAL COMMENTS GO HERE

\subsection{Mental}


\subsection{Technological}
\begin{itemize}
\item Tech: 


\item Weapons:

\item Tactics:
\begin{itemize}
\item    Small groups: 
\item    Large groups/Fleets: 
\end{itemize}


\item Installations:


\end{itemize}

\subsection{Culture}

\subsubsection{Factions and Organizational Groups}
Listed below are noteworthy Aeran sub-factions and organizational groups: 
\begin{itemize}
\item FACTIONS GO HERE
\end{itemize}

\subsubsection{Religion}

\subsubsection{Cultural Aesthetics}

\subsection{Writing, numbers, and insignia}

\subsection{Faction: PRIMARY FACTION}
%Faction data 
%Aera 
%Species 	Aera 
%Homeworld (Origin) 	Aeneth 
%Capital 	Aeneth 


\subsubsection{A Brief History of the PRIMARY FACTION}


\subsubsection{Development}

\subsubsection{Culture}

\subsubsection{Organization}

\subsection{Faction: OTHER FACTIONS}

%Faction data 
%Merchant Marines 
%Species 	Aera 
%Homeworld (Origin) 	Aeneth 
%Capital 	Aeneth 


\subsection{Vessels}

\subsubsection{Style Overview}
\begin{itemize}

\item Primary distinguishing color ranges: 

\item Common accent colors:

\item Primary lighting color:

\item Frequently visible: 

\item Rarely visible:

\item Seen inside, but not out: 

\item Moving parts(non-turret): 

\item Capital vs. light craft: 

\end{itemize}

\subsubsection{Surface features of large vessels}

\subsubsection{Small things found on the hull of a large  vessel}
\begin{itemize}
\item Service/Maintenance hatches
\end{itemize}
{\it Somewhat larger things found on the hull of a large Aeran vessel:}
\begin{itemize}
\item Escape pod launcher ports
\end{itemize}
{\it Yet larger things ... :}
\begin{itemize}
\item Pinnace/lander launch bay (non-carrier vessels)
\end{itemize}

\subsubsection{Listing of vessels}

\begin{itemize}
\item \href{http://vegastrike.sourceforge.net/wiki/Vessel:FOO}{FOO:} 

Existing concept art is not particularly canonical. Please redesign.

\end{itemize}

% LocalWords:  Aerans Aeran Artstyle Aera Pinnace Acrotatus Agasicles Agesilaus
% LocalWords:  Agesipolis Agis Alcmenes Anaxander Wiki Anaxandridas Rlaan Areus
% LocalWords:  Anaxidamus Ariston Charillus Cleombrotus Cleomenes Demaratus Uln
% LocalWords:  Theopompus Dorissus Echestratus Eurycratides Eurypon Leons Bzbr
% LocalWords:  Nicander Pausanias Pleistarchus UnAeranned Pleistoanax Andolian
% LocalWords:  Spaceborn's MacGyver Polydectes Polydorus EVAs Procles Prytanis
% LocalWords:  Soos Teleclus terraforming Aenethforming Zeuxidamus homeworld
% LocalWords:  coreward chemoreception Klk'k Aeneth ecologies Shmrn

\section{Portfolio: The Interstellar Shipping and Mercantile Guild}
Ships overview for all groups (Work In Progress): \href{http://vegastrike.sourceforge.net/wiki/Artstyle\_guide:Overview\_Guide}{Ship Overview} \\
Art-style Guide (Work In Progress): \href{http://vegastrike.sourceforge.net/wiki/Artstyle\_guide:ISMG}{Artstyle Guide} \\
Species overview: \href{http://vegastrike.sourceforge.net/wiki/Species:Humanity}{Species:Humanity} \\

\subsection{Origin}
\begin{itemize}
\item Gravity: 

\item Atmosphere: 

\item Primary liquid bodies: 

\item Average temperature of homeworld (pre-industrialization):

\item Sun: 

\item Primary challenges (pre-industrialization): 
\end{itemize}

%ORIGIN COMMENTS GO HERE

\subsubsection{Habitat}

\subsection{Physical}
\begin{itemize}
\item Dimensions: 

\item Mass: 

\item Skeletal system: 

\item Major divisions: 

\item Senses: 

\item Visual acuity: 

\item Chemosense: 

\item Locomotion: 

\item Manipulators: 

\item Textural appearance: 
\end{itemize}

%PHYSICAL COMMENTS GO HERE

\subsection{Mental}


\subsection{Technological}
\begin{itemize}
\item Tech: 


\item Weapons:

\item Tactics:
\begin{itemize}
\item    Small groups: 
\item    Large groups/Fleets: 
\end{itemize}


\item Installations:


\end{itemize}

\subsection{Culture}

\subsubsection{Factions and Organizational Groups}
Listed below are noteworthy Aeran sub-factions and organizational groups: 
\begin{itemize}
\item FACTIONS GO HERE
\end{itemize}

\subsubsection{Religion}

\subsubsection{Cultural Aesthetics}

\subsection{Writing, numbers, and insignia}

\subsection{Faction: PRIMARY FACTION}
%Faction data 
%Aera 
%Species 	Aera 
%Homeworld (Origin) 	Aeneth 
%Capital 	Aeneth 


\subsubsection{A Brief History of the PRIMARY FACTION}


\subsubsection{Development}

\subsubsection{Culture}

\subsubsection{Organization}

\subsection{Faction: OTHER FACTIONS}

%Faction data 
%Merchant Marines 
%Species 	Aera 
%Homeworld (Origin) 	Aeneth 
%Capital 	Aeneth 


\subsection{Vessels}

\subsubsection{Style Overview}
\begin{itemize}

\item Primary distinguishing color ranges: 

\item Common accent colors:

\item Primary lighting color:

\item Frequently visible: 

\item Rarely visible:

\item Seen inside, but not out: 

\item Moving parts(non-turret): 

\item Capital vs. light craft: 

\end{itemize}

\subsubsection{Surface features of large vessels}

\subsubsection{Small things found on the hull of a large  vessel}
\begin{itemize}
\item Service/Maintenance hatches
\end{itemize}
{\it Somewhat larger things found on the hull of a large Aeran vessel:}
\begin{itemize}
\item Escape pod launcher ports
\end{itemize}
{\it Yet larger things ... :}
\begin{itemize}
\item Pinnace/lander launch bay (non-carrier vessels)
\end{itemize}

\subsubsection{Listing of vessels}

\begin{itemize}
\item \href{http://vegastrike.sourceforge.net/wiki/Vessel:FOO}{FOO:} 

Existing concept art is not particularly canonical. Please redesign.

\end{itemize}

% LocalWords:  Aerans Aeran Artstyle Aera Pinnace Acrotatus Agasicles Agesilaus
% LocalWords:  Agesipolis Agis Alcmenes Anaxander Wiki Anaxandridas Rlaan Areus
% LocalWords:  Anaxidamus Ariston Charillus Cleombrotus Cleomenes Demaratus Uln
% LocalWords:  Theopompus Dorissus Echestratus Eurycratides Eurypon Leons Bzbr
% LocalWords:  Nicander Pausanias Pleistarchus UnAeranned Pleistoanax Andolian
% LocalWords:  Spaceborn's MacGyver Polydectes Polydorus EVAs Procles Prytanis
% LocalWords:  Soos Teleclus terraforming Aenethforming Zeuxidamus homeworld
% LocalWords:  coreward chemoreception Klk'k Aeneth ecologies Shmrn
                  
\section{Portfolio: The Interstellar Socialist Organization}
Ships overview for all groups (Work In Progress): \href{http://vegastrike.sourceforge.net/wiki/Artstyle\_guide:Overview\_Guide}{Ship Overview} \\
Art-style Guide (Work In Progress): \href{http://vegastrike.sourceforge.net/wiki/Artstyle\_guide:ISO}{Artstyle Guide} \\
Species overview: \href{http://vegastrike.sourceforge.net/wiki/Species:Humanity}{Species:Humanity} \\

\subsection{Origin}
\begin{itemize}
\item Gravity: 

\item Atmosphere: 

\item Primary liquid bodies: 

\item Average temperature of homeworld (pre-industrialization):

\item Sun: 

\item Primary challenges (pre-industrialization): 
\end{itemize}

%ORIGIN COMMENTS GO HERE

\subsubsection{Habitat}

\subsection{Physical}
\begin{itemize}
\item Dimensions: 

\item Mass: 

\item Skeletal system: 

\item Major divisions: 

\item Senses: 

\item Visual acuity: 

\item Chemosense: 

\item Locomotion: 

\item Manipulators: 

\item Textural appearance: 
\end{itemize}

%PHYSICAL COMMENTS GO HERE

\subsection{Mental}


\subsection{Technological}
\begin{itemize}
\item Tech: 


\item Weapons:

\item Tactics:
\begin{itemize}
\item    Small groups: 
\item    Large groups/Fleets: 
\end{itemize}


\item Installations:


\end{itemize}

\subsection{Culture}

\subsubsection{Factions and Organizational Groups}
Listed below are noteworthy Aeran sub-factions and organizational groups: 
\begin{itemize}
\item FACTIONS GO HERE
\end{itemize}

\subsubsection{Religion}

\subsubsection{Cultural Aesthetics}

\subsection{Writing, numbers, and insignia}

\subsection{Faction: PRIMARY FACTION}
%Faction data 
%Aera 
%Species 	Aera 
%Homeworld (Origin) 	Aeneth 
%Capital 	Aeneth 


\subsubsection{A Brief History of the PRIMARY FACTION}


\subsubsection{Development}

\subsubsection{Culture}

\subsubsection{Organization}

\subsection{Faction: OTHER FACTIONS}

%Faction data 
%Merchant Marines 
%Species 	Aera 
%Homeworld (Origin) 	Aeneth 
%Capital 	Aeneth 


\subsection{Vessels}

\subsubsection{Style Overview}
\begin{itemize}

\item Primary distinguishing color ranges: 

\item Common accent colors:

\item Primary lighting color:

\item Frequently visible: 

\item Rarely visible:

\item Seen inside, but not out: 

\item Moving parts(non-turret): 

\item Capital vs. light craft: 

\end{itemize}

\subsubsection{Surface features of large vessels}

\subsubsection{Small things found on the hull of a large  vessel}
\begin{itemize}
\item Service/Maintenance hatches
\end{itemize}
{\it Somewhat larger things found on the hull of a large Aeran vessel:}
\begin{itemize}
\item Escape pod launcher ports
\end{itemize}
{\it Yet larger things ... :}
\begin{itemize}
\item Pinnace/lander launch bay (non-carrier vessels)
\end{itemize}

\subsubsection{Listing of vessels}

\begin{itemize}
\item \href{http://vegastrike.sourceforge.net/wiki/Vessel:FOO}{FOO:} 

Existing concept art is not particularly canonical. Please redesign.

\end{itemize}

% LocalWords:  Aerans Aeran Artstyle Aera Pinnace Acrotatus Agasicles Agesilaus
% LocalWords:  Agesipolis Agis Alcmenes Anaxander Wiki Anaxandridas Rlaan Areus
% LocalWords:  Anaxidamus Ariston Charillus Cleombrotus Cleomenes Demaratus Uln
% LocalWords:  Theopompus Dorissus Echestratus Eurycratides Eurypon Leons Bzbr
% LocalWords:  Nicander Pausanias Pleistarchus UnAeranned Pleistoanax Andolian
% LocalWords:  Spaceborn's MacGyver Polydectes Polydorus EVAs Procles Prytanis
% LocalWords:  Soos Teleclus terraforming Aenethforming Zeuxidamus homeworld
% LocalWords:  coreward chemoreception Klk'k Aeneth ecologies Shmrn
               
\section{Portfolio: The Klk\'k}
Ships overview for all groups (Work In Progress): \href{http://vegastrike.sourceforge.net/wiki/Artstyle\_guide:Overview\_Guide}{Ship Overview} \\
Art-style Guide (Work In Progress): \href{http://vegastrike.sourceforge.net/wiki/Artstyle\_guide:Klkk}{Artstyle Guide} \\
Species overview: \href{http://vegastrike.sourceforge.net/wiki/Species:Klkk}{Species:Klkk} \\

\subsection{Origin}
\begin{itemize}
\item Gravity: 

\item Atmosphere: 

\item Primary liquid bodies: 

\item Average temperature of homeworld (pre-industrialization):

\item Sun: 

\item Primary challenges (pre-industrialization): 
\end{itemize}

%ORIGIN COMMENTS GO HERE

\subsubsection{Habitat}

\subsection{Physical}
\begin{itemize}
\item Dimensions: 

\item Mass: 

\item Skeletal system: 

\item Major divisions: 

\item Senses: 

\item Visual acuity: 

\item Chemosense: 

\item Locomotion: 

\item Manipulators: 

\item Textural appearance: 
\end{itemize}

%PHYSICAL COMMENTS GO HERE

\subsection{Mental}


\subsection{Technological}
\begin{itemize}
\item Tech: 


\item Weapons:

\item Tactics:
\begin{itemize}
\item    Small groups: 
\item    Large groups/Fleets: 
\end{itemize}


\item Installations:


\end{itemize}

\subsection{Culture}

\subsubsection{Factions and Organizational Groups}
Listed below are noteworthy Aeran sub-factions and organizational groups: 
\begin{itemize}
\item FACTIONS GO HERE
\end{itemize}

\subsubsection{Religion}

\subsubsection{Cultural Aesthetics}

\subsection{Writing, numbers, and insignia}

\subsection{Faction: PRIMARY FACTION}
%Faction data 
%Aera 
%Species 	Aera 
%Homeworld (Origin) 	Aeneth 
%Capital 	Aeneth 


\subsubsection{A Brief History of the PRIMARY FACTION}


\subsubsection{Development}

\subsubsection{Culture}

\subsubsection{Organization}

\subsection{Faction: OTHER FACTIONS}

%Faction data 
%Merchant Marines 
%Species 	Aera 
%Homeworld (Origin) 	Aeneth 
%Capital 	Aeneth 


\subsection{Vessels}

\subsubsection{Style Overview}
\begin{itemize}

\item Primary distinguishing color ranges: 

\item Common accent colors:

\item Primary lighting color:

\item Frequently visible: 

\item Rarely visible:

\item Seen inside, but not out: 

\item Moving parts(non-turret): 

\item Capital vs. light craft: 

\end{itemize}

\subsubsection{Surface features of large vessels}

\subsubsection{Small things found on the hull of a large  vessel}
\begin{itemize}
\item Service/Maintenance hatches
\end{itemize}
{\it Somewhat larger things found on the hull of a large Aeran vessel:}
\begin{itemize}
\item Escape pod launcher ports
\end{itemize}
{\it Yet larger things ... :}
\begin{itemize}
\item Pinnace/lander launch bay (non-carrier vessels)
\end{itemize}

\subsubsection{Listing of vessels}

\begin{itemize}
\item \href{http://vegastrike.sourceforge.net/wiki/Vessel:FOO}{FOO:} 

Existing concept art is not particularly canonical. Please redesign.

\end{itemize}

% LocalWords:  Aerans Aeran Artstyle Aera Pinnace Acrotatus Agasicles Agesilaus
% LocalWords:  Agesipolis Agis Alcmenes Anaxander Wiki Anaxandridas Rlaan Areus
% LocalWords:  Anaxidamus Ariston Charillus Cleombrotus Cleomenes Demaratus Uln
% LocalWords:  Theopompus Dorissus Echestratus Eurycratides Eurypon Leons Bzbr
% LocalWords:  Nicander Pausanias Pleistarchus UnAeranned Pleistoanax Andolian
% LocalWords:  Spaceborn's MacGyver Polydectes Polydorus EVAs Procles Prytanis
% LocalWords:  Soos Teleclus terraforming Aenethforming Zeuxidamus homeworld
% LocalWords:  coreward chemoreception Klk'k Aeneth ecologies Shmrn
                 
\section{Portfolio: The League of Independent Human Worlds}
Ships overview for all groups (Work In Progress): \href{http://vegastrike.sourceforge.net/wiki/Artstyle\_guide:Overview\_Guide}{Ship Overview} \\
Art-style Guide (Work In Progress): \href{http://vegastrike.sourceforge.net/wiki/Artstyle\_guide:LIHW}{Artstyle Guide} \\
Species overview: \href{http://vegastrike.sourceforge.net/wiki/Species:Humanity}{Species:Humanity} \\

\subsection{Origin}
\begin{itemize}
\item Gravity: 

\item Atmosphere: 

\item Primary liquid bodies: 

\item Average temperature of homeworld (pre-industrialization):

\item Sun: 

\item Primary challenges (pre-industrialization): 
\end{itemize}

%ORIGIN COMMENTS GO HERE

\subsubsection{Habitat}

\subsection{Physical}
\begin{itemize}
\item Dimensions: 

\item Mass: 

\item Skeletal system: 

\item Major divisions: 

\item Senses: 

\item Visual acuity: 

\item Chemosense: 

\item Locomotion: 

\item Manipulators: 

\item Textural appearance: 
\end{itemize}

%PHYSICAL COMMENTS GO HERE

\subsection{Mental}


\subsection{Technological}
\begin{itemize}
\item Tech: 


\item Weapons:

\item Tactics:
\begin{itemize}
\item    Small groups: 
\item    Large groups/Fleets: 
\end{itemize}


\item Installations:


\end{itemize}

\subsection{Culture}

\subsubsection{Factions and Organizational Groups}
Listed below are noteworthy Aeran sub-factions and organizational groups: 
\begin{itemize}
\item FACTIONS GO HERE
\end{itemize}

\subsubsection{Religion}

\subsubsection{Cultural Aesthetics}

\subsection{Writing, numbers, and insignia}

\subsection{Faction: PRIMARY FACTION}
%Faction data 
%Aera 
%Species 	Aera 
%Homeworld (Origin) 	Aeneth 
%Capital 	Aeneth 


\subsubsection{A Brief History of the PRIMARY FACTION}


\subsubsection{Development}

\subsubsection{Culture}

\subsubsection{Organization}

\subsection{Faction: OTHER FACTIONS}

%Faction data 
%Merchant Marines 
%Species 	Aera 
%Homeworld (Origin) 	Aeneth 
%Capital 	Aeneth 


\subsection{Vessels}

\subsubsection{Style Overview}
\begin{itemize}

\item Primary distinguishing color ranges: 

\item Common accent colors:

\item Primary lighting color:

\item Frequently visible: 

\item Rarely visible:

\item Seen inside, but not out: 

\item Moving parts(non-turret): 

\item Capital vs. light craft: 

\end{itemize}

\subsubsection{Surface features of large vessels}

\subsubsection{Small things found on the hull of a large  vessel}
\begin{itemize}
\item Service/Maintenance hatches
\end{itemize}
{\it Somewhat larger things found on the hull of a large Aeran vessel:}
\begin{itemize}
\item Escape pod launcher ports
\end{itemize}
{\it Yet larger things ... :}
\begin{itemize}
\item Pinnace/lander launch bay (non-carrier vessels)
\end{itemize}

\subsubsection{Listing of vessels}

\begin{itemize}
\item \href{http://vegastrike.sourceforge.net/wiki/Vessel:FOO}{FOO:} 

Existing concept art is not particularly canonical. Please redesign.

\end{itemize}

% LocalWords:  Aerans Aeran Artstyle Aera Pinnace Acrotatus Agasicles Agesilaus
% LocalWords:  Agesipolis Agis Alcmenes Anaxander Wiki Anaxandridas Rlaan Areus
% LocalWords:  Anaxidamus Ariston Charillus Cleombrotus Cleomenes Demaratus Uln
% LocalWords:  Theopompus Dorissus Echestratus Eurycratides Eurypon Leons Bzbr
% LocalWords:  Nicander Pausanias Pleistarchus UnAeranned Pleistoanax Andolian
% LocalWords:  Spaceborn's MacGyver Polydectes Polydorus EVAs Procles Prytanis
% LocalWords:  Soos Teleclus terraforming Aenethforming Zeuxidamus homeworld
% LocalWords:  coreward chemoreception Klk'k Aeneth ecologies Shmrn
                 
\section{Portfolio: The Lmpl}
Ships overview for all groups (Work In Progress): \href{http://vegastrike.sourceforge.net/wiki/Artstyle\_guide:Overview\_Guide}{Ship Overview} \\
Art-style Guide (Work In Progress): \href{http://vegastrike.sourceforge.net/wiki/Artstyle\_guide:Lmpl}{Artstyle Guide} \\
Species overview: \href{http://vegastrike.sourceforge.net/wiki/Species:Lmpl}{Species:Lmpl} \\

\subsection{Origin}
\begin{itemize}
\item Gravity: 

\item Atmosphere: 

\item Primary liquid bodies: 

\item Average temperature of homeworld (pre-industrialization):

\item Sun: 

\item Primary challenges (pre-industrialization): 
\end{itemize}

%ORIGIN COMMENTS GO HERE

\subsubsection{Habitat}

\subsection{Physical}
\begin{itemize}
\item Dimensions: 

\item Mass: 

\item Skeletal system: 

\item Major divisions: 

\item Senses: 

\item Visual acuity: 

\item Chemosense: 

\item Locomotion: 

\item Manipulators: 

\item Textural appearance: 
\end{itemize}

%PHYSICAL COMMENTS GO HERE

\subsection{Mental}


\subsection{Technological}
\begin{itemize}
\item Tech: 


\item Weapons:

\item Tactics:
\begin{itemize}
\item    Small groups: 
\item    Large groups/Fleets: 
\end{itemize}


\item Installations:


\end{itemize}

\subsection{Culture}

\subsubsection{Factions and Organizational Groups}
Listed below are noteworthy Aeran sub-factions and organizational groups: 
\begin{itemize}
\item FACTIONS GO HERE
\end{itemize}

\subsubsection{Religion}

\subsubsection{Cultural Aesthetics}

\subsection{Writing, numbers, and insignia}

\subsection{Faction: PRIMARY FACTION}
%Faction data 
%Aera 
%Species 	Aera 
%Homeworld (Origin) 	Aeneth 
%Capital 	Aeneth 


\subsubsection{A Brief History of the PRIMARY FACTION}


\subsubsection{Development}

\subsubsection{Culture}

\subsubsection{Organization}

\subsection{Faction: OTHER FACTIONS}

%Faction data 
%Merchant Marines 
%Species 	Aera 
%Homeworld (Origin) 	Aeneth 
%Capital 	Aeneth 


\subsection{Vessels}

\subsubsection{Style Overview}
\begin{itemize}

\item Primary distinguishing color ranges: 

\item Common accent colors:

\item Primary lighting color:

\item Frequently visible: 

\item Rarely visible:

\item Seen inside, but not out: 

\item Moving parts(non-turret): 

\item Capital vs. light craft: 

\end{itemize}

\subsubsection{Surface features of large vessels}

\subsubsection{Small things found on the hull of a large  vessel}
\begin{itemize}
\item Service/Maintenance hatches
\end{itemize}
{\it Somewhat larger things found on the hull of a large Aeran vessel:}
\begin{itemize}
\item Escape pod launcher ports
\end{itemize}
{\it Yet larger things ... :}
\begin{itemize}
\item Pinnace/lander launch bay (non-carrier vessels)
\end{itemize}

\subsubsection{Listing of vessels}

\begin{itemize}
\item \href{http://vegastrike.sourceforge.net/wiki/Vessel:FOO}{FOO:} 

Existing concept art is not particularly canonical. Please redesign.

\end{itemize}

% LocalWords:  Aerans Aeran Artstyle Aera Pinnace Acrotatus Agasicles Agesilaus
% LocalWords:  Agesipolis Agis Alcmenes Anaxander Wiki Anaxandridas Rlaan Areus
% LocalWords:  Anaxidamus Ariston Charillus Cleombrotus Cleomenes Demaratus Uln
% LocalWords:  Theopompus Dorissus Echestratus Eurycratides Eurypon Leons Bzbr
% LocalWords:  Nicander Pausanias Pleistarchus UnAeranned Pleistoanax Andolian
% LocalWords:  Spaceborn's MacGyver Polydectes Polydorus EVAs Procles Prytanis
% LocalWords:  Soos Teleclus terraforming Aenethforming Zeuxidamus homeworld
% LocalWords:  coreward chemoreception Klk'k Aeneth ecologies Shmrn

\section{Portfolio: The Interstellar Church of True Form's Return}
Ships overview for all groups (Work In Progress): \href{http://vegastrike.sourceforge.net/wiki/Artstyle\_guide:Overview\_Guide}{Ship Overview} \\
Art-style Guide (Work In Progress): \href{http://vegastrike.sourceforge.net/wiki/Artstyle\_guide:Luddites}{Artstyle Guide} \\
Species overview: \href{http://vegastrike.sourceforge.net/wiki/Species:Humanity}{Species:Humanity} \\

\subsection{Origin}
\begin{itemize}
\item Gravity: 

\item Atmosphere: 

\item Primary liquid bodies: 

\item Average temperature of homeworld (pre-industrialization):

\item Sun: 

\item Primary challenges (pre-industrialization): 
\end{itemize}

%ORIGIN COMMENTS GO HERE

\subsubsection{Habitat}

\subsection{Physical}
\begin{itemize}
\item Dimensions: 

\item Mass: 

\item Skeletal system: 

\item Major divisions: 

\item Senses: 

\item Visual acuity: 

\item Chemosense: 

\item Locomotion: 

\item Manipulators: 

\item Textural appearance: 
\end{itemize}

%PHYSICAL COMMENTS GO HERE

\subsection{Mental}


\subsection{Technological}
\begin{itemize}
\item Tech: 


\item Weapons:

\item Tactics:
\begin{itemize}
\item    Small groups: 
\item    Large groups/Fleets: 
\end{itemize}


\item Installations:


\end{itemize}

\subsection{Culture}

\subsubsection{Factions and Organizational Groups}
Listed below are noteworthy Aeran sub-factions and organizational groups: 
\begin{itemize}
\item FACTIONS GO HERE
\end{itemize}

\subsubsection{Religion}

\subsubsection{Cultural Aesthetics}

\subsection{Writing, numbers, and insignia}

\subsection{Faction: PRIMARY FACTION}
%Faction data 
%Aera 
%Species 	Aera 
%Homeworld (Origin) 	Aeneth 
%Capital 	Aeneth 


\subsubsection{A Brief History of the PRIMARY FACTION}


\subsubsection{Development}

\subsubsection{Culture}

\subsubsection{Organization}

\subsection{Faction: OTHER FACTIONS}

%Faction data 
%Merchant Marines 
%Species 	Aera 
%Homeworld (Origin) 	Aeneth 
%Capital 	Aeneth 


\subsection{Vessels}

\subsubsection{Style Overview}
\begin{itemize}

\item Primary distinguishing color ranges: 

\item Common accent colors:

\item Primary lighting color:

\item Frequently visible: 

\item Rarely visible:

\item Seen inside, but not out: 

\item Moving parts(non-turret): 

\item Capital vs. light craft: 

\end{itemize}

\subsubsection{Surface features of large vessels}

\subsubsection{Small things found on the hull of a large  vessel}
\begin{itemize}
\item Service/Maintenance hatches
\end{itemize}
{\it Somewhat larger things found on the hull of a large Aeran vessel:}
\begin{itemize}
\item Escape pod launcher ports
\end{itemize}
{\it Yet larger things ... :}
\begin{itemize}
\item Pinnace/lander launch bay (non-carrier vessels)
\end{itemize}

\subsubsection{Listing of vessels}

\begin{itemize}
\item \href{http://vegastrike.sourceforge.net/wiki/Vessel:FOO}{FOO:} 

Existing concept art is not particularly canonical. Please redesign.

\end{itemize}

% LocalWords:  Aerans Aeran Artstyle Aera Pinnace Acrotatus Agasicles Agesilaus
% LocalWords:  Agesipolis Agis Alcmenes Anaxander Wiki Anaxandridas Rlaan Areus
% LocalWords:  Anaxidamus Ariston Charillus Cleombrotus Cleomenes Demaratus Uln
% LocalWords:  Theopompus Dorissus Echestratus Eurycratides Eurypon Leons Bzbr
% LocalWords:  Nicander Pausanias Pleistarchus UnAeranned Pleistoanax Andolian
% LocalWords:  Spaceborn's MacGyver Polydectes Polydorus EVAs Procles Prytanis
% LocalWords:  Soos Teleclus terraforming Aenethforming Zeuxidamus homeworld
% LocalWords:  coreward chemoreception Klk'k Aeneth ecologies Shmrn

\section{Portfolio: The Mechanists (Mandate for Corporeal Perfection via the Abandonment of Flesh)}
Ships overview for all groups (Work In Progress): \href{http://vegastrike.sourceforge.net/wiki/Artstyle\_guide:Overview\_Guide}{Ship Overview} \\
Art-style Guide (Work In Progress): \href{http://vegastrike.sourceforge.net/wiki/Artstyle\_guide:Mechanist}{Artstyle Guide} \\
Species overview: \href{http://vegastrike.sourceforge.net/wiki/Species:Humanity}{Species:Humanity} \\

\subsection{Origin}
\begin{itemize}
\item Gravity: 

\item Atmosphere: 

\item Primary liquid bodies: 

\item Average temperature of homeworld (pre-industrialization):

\item Sun: 

\item Primary challenges (pre-industrialization): 
\end{itemize}

%ORIGIN COMMENTS GO HERE

\subsubsection{Habitat}

\subsection{Physical}
\begin{itemize}
\item Dimensions: 

\item Mass: 

\item Skeletal system: 

\item Major divisions: 

\item Senses: 

\item Visual acuity: 

\item Chemosense: 

\item Locomotion: 

\item Manipulators: 

\item Textural appearance: 
\end{itemize}

%PHYSICAL COMMENTS GO HERE

\subsection{Mental}


\subsection{Technological}
\begin{itemize}
\item Tech: 


\item Weapons:

\item Tactics:
\begin{itemize}
\item    Small groups: 
\item    Large groups/Fleets: 
\end{itemize}


\item Installations:


\end{itemize}

\subsection{Culture}

\subsubsection{Factions and Organizational Groups}
Listed below are noteworthy Aeran sub-factions and organizational groups: 
\begin{itemize}
\item FACTIONS GO HERE
\end{itemize}

\subsubsection{Religion}

\subsubsection{Cultural Aesthetics}

\subsection{Writing, numbers, and insignia}

\subsection{Faction: PRIMARY FACTION}
%Faction data 
%Aera 
%Species 	Aera 
%Homeworld (Origin) 	Aeneth 
%Capital 	Aeneth 


\subsubsection{A Brief History of the PRIMARY FACTION}


\subsubsection{Development}

\subsubsection{Culture}

\subsubsection{Organization}

\subsection{Faction: OTHER FACTIONS}

%Faction data 
%Merchant Marines 
%Species 	Aera 
%Homeworld (Origin) 	Aeneth 
%Capital 	Aeneth 


\subsection{Vessels}

\subsubsection{Style Overview}
\begin{itemize}

\item Primary distinguishing color ranges: 

\item Common accent colors:

\item Primary lighting color:

\item Frequently visible: 

\item Rarely visible:

\item Seen inside, but not out: 

\item Moving parts(non-turret): 

\item Capital vs. light craft: 

\end{itemize}

\subsubsection{Surface features of large vessels}

\subsubsection{Small things found on the hull of a large  vessel}
\begin{itemize}
\item Service/Maintenance hatches
\end{itemize}
{\it Somewhat larger things found on the hull of a large Aeran vessel:}
\begin{itemize}
\item Escape pod launcher ports
\end{itemize}
{\it Yet larger things ... :}
\begin{itemize}
\item Pinnace/lander launch bay (non-carrier vessels)
\end{itemize}

\subsubsection{Listing of vessels}

\begin{itemize}
\item \href{http://vegastrike.sourceforge.net/wiki/Vessel:FOO}{FOO:} 

Existing concept art is not particularly canonical. Please redesign.

\end{itemize}

% LocalWords:  Aerans Aeran Artstyle Aera Pinnace Acrotatus Agasicles Agesilaus
% LocalWords:  Agesipolis Agis Alcmenes Anaxander Wiki Anaxandridas Rlaan Areus
% LocalWords:  Anaxidamus Ariston Charillus Cleombrotus Cleomenes Demaratus Uln
% LocalWords:  Theopompus Dorissus Echestratus Eurycratides Eurypon Leons Bzbr
% LocalWords:  Nicander Pausanias Pleistarchus UnAeranned Pleistoanax Andolian
% LocalWords:  Spaceborn's MacGyver Polydectes Polydorus EVAs Procles Prytanis
% LocalWords:  Soos Teleclus terraforming Aenethforming Zeuxidamus homeworld
% LocalWords:  coreward chemoreception Klk'k Aeneth ecologies Shmrn
 
\section{Portfolio: The Mishtali}
Ships overview for all groups (Work In Progress): \href{http://vegastrike.sourceforge.net/wiki/Artstyle\_guide:Overview\_Guide}{Ship Overview} \\
Art-style Guide (Work In Progress): \href{http://vegastrike.sourceforge.net/wiki/Artstyle\_guide:Mishtali}{Artstyle Guide} \\
Species overview: \href{http://vegastrike.sourceforge.net/wiki/Species:Mishtali}{Species:Mishtali} \\

\subsection{Origin}
\begin{itemize}
\item Gravity: 

\item Atmosphere: 

\item Primary liquid bodies: 

\item Average temperature of homeworld (pre-industrialization):

\item Sun: 

\item Primary challenges (pre-industrialization): 
\end{itemize}

%ORIGIN COMMENTS GO HERE

\subsubsection{Habitat}

\subsection{Physical}
\begin{itemize}
\item Dimensions: 

\item Mass: 

\item Skeletal system: 

\item Major divisions: 

\item Senses: 

\item Visual acuity: 

\item Chemosense: 

\item Locomotion: 

\item Manipulators: 

\item Textural appearance: 
\end{itemize}

%PHYSICAL COMMENTS GO HERE

\subsection{Mental}


\subsection{Technological}
\begin{itemize}
\item Tech: 


\item Weapons:

\item Tactics:
\begin{itemize}
\item    Small groups: 
\item    Large groups/Fleets: 
\end{itemize}


\item Installations:


\end{itemize}

\subsection{Culture}

\subsubsection{Factions and Organizational Groups}
Listed below are noteworthy Aeran sub-factions and organizational groups: 
\begin{itemize}
\item FACTIONS GO HERE
\end{itemize}

\subsubsection{Religion}

\subsubsection{Cultural Aesthetics}

\subsection{Writing, numbers, and insignia}

\subsection{Faction: PRIMARY FACTION}
%Faction data 
%Aera 
%Species 	Aera 
%Homeworld (Origin) 	Aeneth 
%Capital 	Aeneth 


\subsubsection{A Brief History of the PRIMARY FACTION}


\subsubsection{Development}

\subsubsection{Culture}

\subsubsection{Organization}

\subsection{Faction: OTHER FACTIONS}

%Faction data 
%Merchant Marines 
%Species 	Aera 
%Homeworld (Origin) 	Aeneth 
%Capital 	Aeneth 


\subsection{Vessels}

\subsubsection{Style Overview}
\begin{itemize}

\item Primary distinguishing color ranges: 

\item Common accent colors:

\item Primary lighting color:

\item Frequently visible: 

\item Rarely visible:

\item Seen inside, but not out: 

\item Moving parts(non-turret): 

\item Capital vs. light craft: 

\end{itemize}

\subsubsection{Surface features of large vessels}

\subsubsection{Small things found on the hull of a large  vessel}
\begin{itemize}
\item Service/Maintenance hatches
\end{itemize}
{\it Somewhat larger things found on the hull of a large Aeran vessel:}
\begin{itemize}
\item Escape pod launcher ports
\end{itemize}
{\it Yet larger things ... :}
\begin{itemize}
\item Pinnace/lander launch bay (non-carrier vessels)
\end{itemize}

\subsubsection{Listing of vessels}

\begin{itemize}
\item \href{http://vegastrike.sourceforge.net/wiki/Vessel:FOO}{FOO:} 

Existing concept art is not particularly canonical. Please redesign.

\end{itemize}

% LocalWords:  Aerans Aeran Artstyle Aera Pinnace Acrotatus Agasicles Agesilaus
% LocalWords:  Agesipolis Agis Alcmenes Anaxander Wiki Anaxandridas Rlaan Areus
% LocalWords:  Anaxidamus Ariston Charillus Cleombrotus Cleomenes Demaratus Uln
% LocalWords:  Theopompus Dorissus Echestratus Eurycratides Eurypon Leons Bzbr
% LocalWords:  Nicander Pausanias Pleistarchus UnAeranned Pleistoanax Andolian
% LocalWords:  Spaceborn's MacGyver Polydectes Polydorus EVAs Procles Prytanis
% LocalWords:  Soos Teleclus terraforming Aenethforming Zeuxidamus homeworld
% LocalWords:  coreward chemoreception Klk'k Aeneth ecologies Shmrn

\section{Portfolio: The Nuhln}
Ships overview for all groups (Work In Progress): \href{http://vegastrike.sourceforge.net/wiki/Artstyle\_guide:Overview\_Guide}{Ship Overview} \\
Art-style Guide (Work In Progress): \href{http://vegastrike.sourceforge.net/wiki/Artstyle\_guide:Nuhln}{Artstyle Guide} \\
Species overview: \href{http://vegastrike.sourceforge.net/wiki/Species:Nuhln}{Species:Nuhln} \\

\subsection{Origin}
\begin{itemize}
\item Gravity: 

\item Atmosphere: 

\item Primary liquid bodies: 

\item Average temperature of homeworld (pre-industrialization):

\item Sun: 

\item Primary challenges (pre-industrialization): 
\end{itemize}

%ORIGIN COMMENTS GO HERE

\subsubsection{Habitat}

\subsection{Physical}
\begin{itemize}
\item Dimensions: 

\item Mass: 

\item Skeletal system: 

\item Major divisions: 

\item Senses: 

\item Visual acuity: 

\item Chemosense: 

\item Locomotion: 

\item Manipulators: 

\item Textural appearance: 
\end{itemize}

%PHYSICAL COMMENTS GO HERE

\subsection{Mental}


\subsection{Technological}
\begin{itemize}
\item Tech: 


\item Weapons:

\item Tactics:
\begin{itemize}
\item    Small groups: 
\item    Large groups/Fleets: 
\end{itemize}


\item Installations:


\end{itemize}

\subsection{Culture}

\subsubsection{Factions and Organizational Groups}
Listed below are noteworthy Aeran sub-factions and organizational groups: 
\begin{itemize}
\item FACTIONS GO HERE
\end{itemize}

\subsubsection{Religion}

\subsubsection{Cultural Aesthetics}

\subsection{Writing, numbers, and insignia}

\subsection{Faction: PRIMARY FACTION}
%Faction data 
%Aera 
%Species 	Aera 
%Homeworld (Origin) 	Aeneth 
%Capital 	Aeneth 


\subsubsection{A Brief History of the PRIMARY FACTION}


\subsubsection{Development}

\subsubsection{Culture}

\subsubsection{Organization}

\subsection{Faction: OTHER FACTIONS}

%Faction data 
%Merchant Marines 
%Species 	Aera 
%Homeworld (Origin) 	Aeneth 
%Capital 	Aeneth 


\subsection{Vessels}

\subsubsection{Style Overview}
\begin{itemize}

\item Primary distinguishing color ranges: 

\item Common accent colors:

\item Primary lighting color:

\item Frequently visible: 

\item Rarely visible:

\item Seen inside, but not out: 

\item Moving parts(non-turret): 

\item Capital vs. light craft: 

\end{itemize}

\subsubsection{Surface features of large vessels}

\subsubsection{Small things found on the hull of a large  vessel}
\begin{itemize}
\item Service/Maintenance hatches
\end{itemize}
{\it Somewhat larger things found on the hull of a large Aeran vessel:}
\begin{itemize}
\item Escape pod launcher ports
\end{itemize}
{\it Yet larger things ... :}
\begin{itemize}
\item Pinnace/lander launch bay (non-carrier vessels)
\end{itemize}

\subsubsection{Listing of vessels}

\begin{itemize}
\item \href{http://vegastrike.sourceforge.net/wiki/Vessel:FOO}{FOO:} 

Existing concept art is not particularly canonical. Please redesign.

\end{itemize}

% LocalWords:  Aerans Aeran Artstyle Aera Pinnace Acrotatus Agasicles Agesilaus
% LocalWords:  Agesipolis Agis Alcmenes Anaxander Wiki Anaxandridas Rlaan Areus
% LocalWords:  Anaxidamus Ariston Charillus Cleombrotus Cleomenes Demaratus Uln
% LocalWords:  Theopompus Dorissus Echestratus Eurycratides Eurypon Leons Bzbr
% LocalWords:  Nicander Pausanias Pleistarchus UnAeranned Pleistoanax Andolian
% LocalWords:  Spaceborn's MacGyver Polydectes Polydorus EVAs Procles Prytanis
% LocalWords:  Soos Teleclus terraforming Aenethforming Zeuxidamus homeworld
% LocalWords:  coreward chemoreception Klk'k Aeneth ecologies Shmrn

\section{Portfolio: The Purists}
Ships overview for all groups (Work In Progress): \href{http://vegastrike.sourceforge.net/wiki/Artstyle\_guide:Overview\_Guide}{Ship Overview} \\
Art-style Guide (Work In Progress): \href{http://vegastrike.sourceforge.net/wiki/Artstyle\_guide:Purist}{Artstyle Guide} \\
Species overview: \href{http://vegastrike.sourceforge.net/wiki/Species:Humanity}{Species:Humanity} \\

\subsection{Origin}
\begin{itemize}
\item Gravity: 

\item Atmosphere: 

\item Primary liquid bodies: 

\item Average temperature of homeworld (pre-industrialization):

\item Sun: 

\item Primary challenges (pre-industrialization): 
\end{itemize}

%ORIGIN COMMENTS GO HERE

\subsubsection{Habitat}

\subsection{Physical}
\begin{itemize}
\item Dimensions: 

\item Mass: 

\item Skeletal system: 

\item Major divisions: 

\item Senses: 

\item Visual acuity: 

\item Chemosense: 

\item Locomotion: 

\item Manipulators: 

\item Textural appearance: 
\end{itemize}

%PHYSICAL COMMENTS GO HERE

\subsection{Mental}


\subsection{Technological}
\begin{itemize}
\item Tech: 


\item Weapons:

\item Tactics:
\begin{itemize}
\item    Small groups: 
\item    Large groups/Fleets: 
\end{itemize}


\item Installations:


\end{itemize}

\subsection{Culture}

\subsubsection{Factions and Organizational Groups}
Listed below are noteworthy Aeran sub-factions and organizational groups: 
\begin{itemize}
\item FACTIONS GO HERE
\end{itemize}

\subsubsection{Religion}

\subsubsection{Cultural Aesthetics}

\subsection{Writing, numbers, and insignia}

\subsection{Faction: PRIMARY FACTION}
%Faction data 
%Aera 
%Species 	Aera 
%Homeworld (Origin) 	Aeneth 
%Capital 	Aeneth 


\subsubsection{A Brief History of the PRIMARY FACTION}


\subsubsection{Development}

\subsubsection{Culture}

\subsubsection{Organization}

\subsection{Faction: OTHER FACTIONS}

%Faction data 
%Merchant Marines 
%Species 	Aera 
%Homeworld (Origin) 	Aeneth 
%Capital 	Aeneth 


\subsection{Vessels}

\subsubsection{Style Overview}
\begin{itemize}

\item Primary distinguishing color ranges: 

\item Common accent colors:

\item Primary lighting color:

\item Frequently visible: 

\item Rarely visible:

\item Seen inside, but not out: 

\item Moving parts(non-turret): 

\item Capital vs. light craft: 

\end{itemize}

\subsubsection{Surface features of large vessels}

\subsubsection{Small things found on the hull of a large  vessel}
\begin{itemize}
\item Service/Maintenance hatches
\end{itemize}
{\it Somewhat larger things found on the hull of a large Aeran vessel:}
\begin{itemize}
\item Escape pod launcher ports
\end{itemize}
{\it Yet larger things ... :}
\begin{itemize}
\item Pinnace/lander launch bay (non-carrier vessels)
\end{itemize}

\subsubsection{Listing of vessels}

\begin{itemize}
\item \href{http://vegastrike.sourceforge.net/wiki/Vessel:FOO}{FOO:} 

Existing concept art is not particularly canonical. Please redesign.

\end{itemize}

% LocalWords:  Aerans Aeran Artstyle Aera Pinnace Acrotatus Agasicles Agesilaus
% LocalWords:  Agesipolis Agis Alcmenes Anaxander Wiki Anaxandridas Rlaan Areus
% LocalWords:  Anaxidamus Ariston Charillus Cleombrotus Cleomenes Demaratus Uln
% LocalWords:  Theopompus Dorissus Echestratus Eurycratides Eurypon Leons Bzbr
% LocalWords:  Nicander Pausanias Pleistarchus UnAeranned Pleistoanax Andolian
% LocalWords:  Spaceborn's MacGyver Polydectes Polydorus EVAs Procles Prytanis
% LocalWords:  Soos Teleclus terraforming Aenethforming Zeuxidamus homeworld
% LocalWords:  coreward chemoreception Klk'k Aeneth ecologies Shmrn
 
\section{Portfolio: The Purth}
Ships overview for all groups (Work In Progress): \href{http://vegastrike.sourceforge.net/wiki/Artstyle\_guide:Overview\_Guide}{Ship Overview} \\
Art-style Guide (Work In Progress): \href{http://vegastrike.sourceforge.net/wiki/Artstyle\_guide:Purth}{Artstyle Guide} \\
Species overview: \href{http://vegastrike.sourceforge.net/wiki/Species:Humanity}{Species:Purth} \\

\subsection{Origin}
\begin{itemize}
\item Gravity: 

\item Atmosphere: 

\item Primary liquid bodies: 

\item Average temperature of homeworld (pre-industrialization):

\item Sun: 

\item Primary challenges (pre-industrialization): 
\end{itemize}

%ORIGIN COMMENTS GO HERE

\subsubsection{Habitat}

\subsection{Physical}
\begin{itemize}
\item Dimensions: 

\item Mass: 

\item Skeletal system: 

\item Major divisions: 

\item Senses: 

\item Visual acuity: 

\item Chemosense: 

\item Locomotion: 

\item Manipulators: 

\item Textural appearance: 
\end{itemize}

%PHYSICAL COMMENTS GO HERE

\subsection{Mental}


\subsection{Technological}
\begin{itemize}
\item Tech: 


\item Weapons:

\item Tactics:
\begin{itemize}
\item    Small groups: 
\item    Large groups/Fleets: 
\end{itemize}


\item Installations:


\end{itemize}

\subsection{Culture}

\subsubsection{Factions and Organizational Groups}
Listed below are noteworthy Aeran sub-factions and organizational groups: 
\begin{itemize}
\item FACTIONS GO HERE
\end{itemize}

\subsubsection{Religion}

\subsubsection{Cultural Aesthetics}

\subsection{Writing, numbers, and insignia}

\subsection{Faction: PRIMARY FACTION}
%Faction data 
%Aera 
%Species 	Aera 
%Homeworld (Origin) 	Aeneth 
%Capital 	Aeneth 


\subsubsection{A Brief History of the PRIMARY FACTION}


\subsubsection{Development}

\subsubsection{Culture}

\subsubsection{Organization}

\subsection{Faction: OTHER FACTIONS}

%Faction data 
%Merchant Marines 
%Species 	Aera 
%Homeworld (Origin) 	Aeneth 
%Capital 	Aeneth 


\subsection{Vessels}

\subsubsection{Style Overview}
\begin{itemize}

\item Primary distinguishing color ranges: 

\item Common accent colors:

\item Primary lighting color:

\item Frequently visible: 

\item Rarely visible:

\item Seen inside, but not out: 

\item Moving parts(non-turret): 

\item Capital vs. light craft: 

\end{itemize}

\subsubsection{Surface features of large vessels}

\subsubsection{Small things found on the hull of a large  vessel}
\begin{itemize}
\item Service/Maintenance hatches
\end{itemize}
{\it Somewhat larger things found on the hull of a large Aeran vessel:}
\begin{itemize}
\item Escape pod launcher ports
\end{itemize}
{\it Yet larger things ... :}
\begin{itemize}
\item Pinnace/lander launch bay (non-carrier vessels)
\end{itemize}

\subsubsection{Listing of vessels}

\begin{itemize}
\item \href{http://vegastrike.sourceforge.net/wiki/Vessel:FOO}{FOO:} 

Existing concept art is not particularly canonical. Please redesign.

\end{itemize}

% LocalWords:  Aerans Aeran Artstyle Aera Pinnace Acrotatus Agasicles Agesilaus
% LocalWords:  Agesipolis Agis Alcmenes Anaxander Wiki Anaxandridas Rlaan Areus
% LocalWords:  Anaxidamus Ariston Charillus Cleombrotus Cleomenes Demaratus Uln
% LocalWords:  Theopompus Dorissus Echestratus Eurycratides Eurypon Leons Bzbr
% LocalWords:  Nicander Pausanias Pleistarchus UnAeranned Pleistoanax Andolian
% LocalWords:  Spaceborn's MacGyver Polydectes Polydorus EVAs Procles Prytanis
% LocalWords:  Soos Teleclus terraforming Aenethforming Zeuxidamus homeworld
% LocalWords:  coreward chemoreception Klk'k Aeneth ecologies Shmrn

\section{Portfolio: The Rlaan}
Ships overview for all groups (Work In Progress): \href{http://vegastrike.sourceforge.net/wiki/Artstyle\_guide:Overview\_Guide}{Ship Overview} \\
Art-style Guide (Work In Progress): \href{http://vegastrike.sourceforge.net/wiki/Artstyle\_guide:Rlaan}{Artstyle Guide} \\
Species overview: \href{http://vegastrike.sourceforge.net/wiki/Species:Rlaan}{Species:Rlaan} \\

\subsection{Origin}
\begin{itemize}
\item Gravity: 

\item Atmosphere: 

\item Primary liquid bodies: 

\item Average temperature of homeworld (pre-industrialization):

\item Sun: 

\item Primary challenges (pre-industrialization): 
\end{itemize}

%ORIGIN COMMENTS GO HERE

\subsubsection{Habitat}

\subsection{Physical}
\begin{itemize}
\item Dimensions: 

\item Mass: 

\item Skeletal system: 

\item Major divisions: 

\item Senses: 

\item Visual acuity: 

\item Chemosense: 

\item Locomotion: 

\item Manipulators: 

\item Textural appearance: 
\end{itemize}

%PHYSICAL COMMENTS GO HERE

\subsection{Mental}


\subsection{Technological}
\begin{itemize}
\item Tech: 


\item Weapons:

\item Tactics:
\begin{itemize}
\item    Small groups: 
\item    Large groups/Fleets: 
\end{itemize}


\item Installations:


\end{itemize}

\subsection{Culture}

\subsubsection{Factions and Organizational Groups}
Listed below are noteworthy Aeran sub-factions and organizational groups: 
\begin{itemize}
\item FACTIONS GO HERE
\end{itemize}

\subsubsection{Religion}

\subsubsection{Cultural Aesthetics}

\subsection{Writing, numbers, and insignia}

\subsection{Faction: PRIMARY FACTION}
%Faction data 
%Aera 
%Species 	Aera 
%Homeworld (Origin) 	Aeneth 
%Capital 	Aeneth 


\subsubsection{A Brief History of the PRIMARY FACTION}


\subsubsection{Development}

\subsubsection{Culture}

\subsubsection{Organization}

\subsection{Faction: OTHER FACTIONS}

%Faction data 
%Merchant Marines 
%Species 	Aera 
%Homeworld (Origin) 	Aeneth 
%Capital 	Aeneth 


\subsection{Vessels}

\subsubsection{Style Overview}
\begin{itemize}

\item Primary distinguishing color ranges: 

\item Common accent colors:

\item Primary lighting color:

\item Frequently visible: 

\item Rarely visible:

\item Seen inside, but not out: 

\item Moving parts(non-turret): 

\item Capital vs. light craft: 

\end{itemize}

\subsubsection{Surface features of large vessels}

\subsubsection{Small things found on the hull of a large  vessel}
\begin{itemize}
\item Service/Maintenance hatches
\end{itemize}
{\it Somewhat larger things found on the hull of a large Aeran vessel:}
\begin{itemize}
\item Escape pod launcher ports
\end{itemize}
{\it Yet larger things ... :}
\begin{itemize}
\item Pinnace/lander launch bay (non-carrier vessels)
\end{itemize}

\subsubsection{Listing of vessels}

\begin{itemize}
\item \href{http://vegastrike.sourceforge.net/wiki/Vessel:FOO}{FOO:} 

Existing concept art is not particularly canonical. Please redesign.

\end{itemize}

% LocalWords:  Aerans Aeran Artstyle Aera Pinnace Acrotatus Agasicles Agesilaus
% LocalWords:  Agesipolis Agis Alcmenes Anaxander Wiki Anaxandridas Rlaan Areus
% LocalWords:  Anaxidamus Ariston Charillus Cleombrotus Cleomenes Demaratus Uln
% LocalWords:  Theopompus Dorissus Echestratus Eurycratides Eurypon Leons Bzbr
% LocalWords:  Nicander Pausanias Pleistarchus UnAeranned Pleistoanax Andolian
% LocalWords:  Spaceborn's MacGyver Polydectes Polydorus EVAs Procles Prytanis
% LocalWords:  Soos Teleclus terraforming Aenethforming Zeuxidamus homeworld
% LocalWords:  coreward chemoreception Klk'k Aeneth ecologies Shmrn

\section{Portfolio: The Rlaan-Briin}
Ships overview for all groups (Work In Progress): \href{http://vegastrike.sourceforge.net/wiki/Artstyle\_guide:Overview\_Guide}{Ship Overview} \\
Art-style Guide (Work In Progress): \href{http://vegastrike.sourceforge.net/wiki/Artstyle\_guide:Rlaan-Briin}{Artstyle Guide} \\
Species overview: \href{http://vegastrike.sourceforge.net/wiki/Species:Rlaan}{Species:Rlaan} \\

\subsection{Origin}
\begin{itemize}
\item Gravity: 

\item Atmosphere: 

\item Primary liquid bodies: 

\item Average temperature of homeworld (pre-industrialization):

\item Sun: 

\item Primary challenges (pre-industrialization): 
\end{itemize}

%ORIGIN COMMENTS GO HERE

\subsubsection{Habitat}

\subsection{Physical}
\begin{itemize}
\item Dimensions: 

\item Mass: 

\item Skeletal system: 

\item Major divisions: 

\item Senses: 

\item Visual acuity: 

\item Chemosense: 

\item Locomotion: 

\item Manipulators: 

\item Textural appearance: 
\end{itemize}

%PHYSICAL COMMENTS GO HERE

\subsection{Mental}


\subsection{Technological}
\begin{itemize}
\item Tech: 


\item Weapons:

\item Tactics:
\begin{itemize}
\item    Small groups: 
\item    Large groups/Fleets: 
\end{itemize}


\item Installations:


\end{itemize}

\subsection{Culture}

\subsubsection{Factions and Organizational Groups}
Listed below are noteworthy Aeran sub-factions and organizational groups: 
\begin{itemize}
\item FACTIONS GO HERE
\end{itemize}

\subsubsection{Religion}

\subsubsection{Cultural Aesthetics}

\subsection{Writing, numbers, and insignia}

\subsection{Faction: PRIMARY FACTION}
%Faction data 
%Aera 
%Species 	Aera 
%Homeworld (Origin) 	Aeneth 
%Capital 	Aeneth 


\subsubsection{A Brief History of the PRIMARY FACTION}


\subsubsection{Development}

\subsubsection{Culture}

\subsubsection{Organization}

\subsection{Faction: OTHER FACTIONS}

%Faction data 
%Merchant Marines 
%Species 	Aera 
%Homeworld (Origin) 	Aeneth 
%Capital 	Aeneth 


\subsection{Vessels}

\subsubsection{Style Overview}
\begin{itemize}

\item Primary distinguishing color ranges: 

\item Common accent colors:

\item Primary lighting color:

\item Frequently visible: 

\item Rarely visible:

\item Seen inside, but not out: 

\item Moving parts(non-turret): 

\item Capital vs. light craft: 

\end{itemize}

\subsubsection{Surface features of large vessels}

\subsubsection{Small things found on the hull of a large  vessel}
\begin{itemize}
\item Service/Maintenance hatches
\end{itemize}
{\it Somewhat larger things found on the hull of a large Aeran vessel:}
\begin{itemize}
\item Escape pod launcher ports
\end{itemize}
{\it Yet larger things ... :}
\begin{itemize}
\item Pinnace/lander launch bay (non-carrier vessels)
\end{itemize}

\subsubsection{Listing of vessels}

\begin{itemize}
\item \href{http://vegastrike.sourceforge.net/wiki/Vessel:FOO}{FOO:} 

Existing concept art is not particularly canonical. Please redesign.

\end{itemize}

% LocalWords:  Aerans Aeran Artstyle Aera Pinnace Acrotatus Agasicles Agesilaus
% LocalWords:  Agesipolis Agis Alcmenes Anaxander Wiki Anaxandridas Rlaan Areus
% LocalWords:  Anaxidamus Ariston Charillus Cleombrotus Cleomenes Demaratus Uln
% LocalWords:  Theopompus Dorissus Echestratus Eurycratides Eurypon Leons Bzbr
% LocalWords:  Nicander Pausanias Pleistarchus UnAeranned Pleistoanax Andolian
% LocalWords:  Spaceborn's MacGyver Polydectes Polydorus EVAs Procles Prytanis
% LocalWords:  Soos Teleclus terraforming Aenethforming Zeuxidamus homeworld
% LocalWords:  coreward chemoreception Klk'k Aeneth ecologies Shmrn

\section{Portfolio: The Saahasayaay}
Ships overview for all groups (Work In Progress): \href{http://vegastrike.sourceforge.net/wiki/Artstyle\_guide:Overview\_Guide}{Ship Overview} \\
Art-style Guide (Work In Progress): \href{http://vegastrike.sourceforge.net/wiki/Artstyle\_guide:Saahasayaay}{Artstyle Guide} \\
Species overview: \href{http://vegastrike.sourceforge.net/wiki/Species:Saahasayaay}{Species:Saahasayaay} \\

\subsection{Origin}
\begin{itemize}
\item Gravity: 

\item Atmosphere: 

\item Primary liquid bodies: 

\item Average temperature of homeworld (pre-industrialization):

\item Sun: 

\item Primary challenges (pre-industrialization): 
\end{itemize}

%ORIGIN COMMENTS GO HERE

\subsubsection{Habitat}

\subsection{Physical}
\begin{itemize}
\item Dimensions: 

\item Mass: 

\item Skeletal system: 

\item Major divisions: 

\item Senses: 

\item Visual acuity: 

\item Chemosense: 

\item Locomotion: 

\item Manipulators: 

\item Textural appearance: 
\end{itemize}

%PHYSICAL COMMENTS GO HERE

\subsection{Mental}


\subsection{Technological}
\begin{itemize}
\item Tech: 


\item Weapons:

\item Tactics:
\begin{itemize}
\item    Small groups: 
\item    Large groups/Fleets: 
\end{itemize}


\item Installations:


\end{itemize}

\subsection{Culture}

\subsubsection{Factions and Organizational Groups}
Listed below are noteworthy Aeran sub-factions and organizational groups: 
\begin{itemize}
\item FACTIONS GO HERE
\end{itemize}

\subsubsection{Religion}

\subsubsection{Cultural Aesthetics}

\subsection{Writing, numbers, and insignia}

\subsection{Faction: PRIMARY FACTION}
%Faction data 
%Aera 
%Species 	Aera 
%Homeworld (Origin) 	Aeneth 
%Capital 	Aeneth 


\subsubsection{A Brief History of the PRIMARY FACTION}


\subsubsection{Development}

\subsubsection{Culture}

\subsubsection{Organization}

\subsection{Faction: OTHER FACTIONS}

%Faction data 
%Merchant Marines 
%Species 	Aera 
%Homeworld (Origin) 	Aeneth 
%Capital 	Aeneth 


\subsection{Vessels}

\subsubsection{Style Overview}
\begin{itemize}

\item Primary distinguishing color ranges: 

\item Common accent colors:

\item Primary lighting color:

\item Frequently visible: 

\item Rarely visible:

\item Seen inside, but not out: 

\item Moving parts(non-turret): 

\item Capital vs. light craft: 

\end{itemize}

\subsubsection{Surface features of large vessels}

\subsubsection{Small things found on the hull of a large  vessel}
\begin{itemize}
\item Service/Maintenance hatches
\end{itemize}
{\it Somewhat larger things found on the hull of a large Aeran vessel:}
\begin{itemize}
\item Escape pod launcher ports
\end{itemize}
{\it Yet larger things ... :}
\begin{itemize}
\item Pinnace/lander launch bay (non-carrier vessels)
\end{itemize}

\subsubsection{Listing of vessels}

\begin{itemize}
\item \href{http://vegastrike.sourceforge.net/wiki/Vessel:FOO}{FOO:} 

Existing concept art is not particularly canonical. Please redesign.

\end{itemize}

% LocalWords:  Aerans Aeran Artstyle Aera Pinnace Acrotatus Agasicles Agesilaus
% LocalWords:  Agesipolis Agis Alcmenes Anaxander Wiki Anaxandridas Rlaan Areus
% LocalWords:  Anaxidamus Ariston Charillus Cleombrotus Cleomenes Demaratus Uln
% LocalWords:  Theopompus Dorissus Echestratus Eurycratides Eurypon Leons Bzbr
% LocalWords:  Nicander Pausanias Pleistarchus UnAeranned Pleistoanax Andolian
% LocalWords:  Spaceborn's MacGyver Polydectes Polydorus EVAs Procles Prytanis
% LocalWords:  Soos Teleclus terraforming Aenethforming Zeuxidamus homeworld
% LocalWords:  coreward chemoreception Klk'k Aeneth ecologies Shmrn

\section{Portfolio: The Shapers}
Ships overview for all groups (Work In Progress): \href{http://vegastrike.sourceforge.net/wiki/Artstyle\_guide:Overview\_Guide}{Ship Overview} \\
Art-style Guide (Work In Progress): \href{http://vegastrike.sourceforge.net/wiki/Artstyle\_guide:Shaper}{Artstyle Guide} \\
Species overview: \href{http://vegastrike.sourceforge.net/wiki/Species:Humanity}{Species:Humanity} \\

\subsection{Origin}
\begin{itemize}
\item Gravity: 

\item Atmosphere: 

\item Primary liquid bodies: 

\item Average temperature of homeworld (pre-industrialization):

\item Sun: 

\item Primary challenges (pre-industrialization): 
\end{itemize}

%ORIGIN COMMENTS GO HERE

\subsubsection{Habitat}

\subsection{Physical}
\begin{itemize}
\item Dimensions: 

\item Mass: 

\item Skeletal system: 

\item Major divisions: 

\item Senses: 

\item Visual acuity: 

\item Chemosense: 

\item Locomotion: 

\item Manipulators: 

\item Textural appearance: 
\end{itemize}

%PHYSICAL COMMENTS GO HERE

\subsection{Mental}


\subsection{Technological}
\begin{itemize}
\item Tech: 


\item Weapons:

\item Tactics:
\begin{itemize}
\item    Small groups: 
\item    Large groups/Fleets: 
\end{itemize}


\item Installations:


\end{itemize}

\subsection{Culture}

\subsubsection{Factions and Organizational Groups}
Listed below are noteworthy Aeran sub-factions and organizational groups: 
\begin{itemize}
\item FACTIONS GO HERE
\end{itemize}

\subsubsection{Religion}

\subsubsection{Cultural Aesthetics}

\subsection{Writing, numbers, and insignia}

\subsection{Faction: PRIMARY FACTION}
%Faction data 
%Aera 
%Species 	Aera 
%Homeworld (Origin) 	Aeneth 
%Capital 	Aeneth 


\subsubsection{A Brief History of the PRIMARY FACTION}


\subsubsection{Development}

\subsubsection{Culture}

\subsubsection{Organization}

\subsection{Faction: OTHER FACTIONS}

%Faction data 
%Merchant Marines 
%Species 	Aera 
%Homeworld (Origin) 	Aeneth 
%Capital 	Aeneth 


\subsection{Vessels}

\subsubsection{Style Overview}
\begin{itemize}

\item Primary distinguishing color ranges: 

\item Common accent colors:

\item Primary lighting color:

\item Frequently visible: 

\item Rarely visible:

\item Seen inside, but not out: 

\item Moving parts(non-turret): 

\item Capital vs. light craft: 

\end{itemize}

\subsubsection{Surface features of large vessels}

\subsubsection{Small things found on the hull of a large  vessel}
\begin{itemize}
\item Service/Maintenance hatches
\end{itemize}
{\it Somewhat larger things found on the hull of a large Aeran vessel:}
\begin{itemize}
\item Escape pod launcher ports
\end{itemize}
{\it Yet larger things ... :}
\begin{itemize}
\item Pinnace/lander launch bay (non-carrier vessels)
\end{itemize}

\subsubsection{Listing of vessels}

\begin{itemize}
\item \href{http://vegastrike.sourceforge.net/wiki/Vessel:FOO}{FOO:} 

Existing concept art is not particularly canonical. Please redesign.

\end{itemize}

% LocalWords:  Aerans Aeran Artstyle Aera Pinnace Acrotatus Agasicles Agesilaus
% LocalWords:  Agesipolis Agis Alcmenes Anaxander Wiki Anaxandridas Rlaan Areus
% LocalWords:  Anaxidamus Ariston Charillus Cleombrotus Cleomenes Demaratus Uln
% LocalWords:  Theopompus Dorissus Echestratus Eurycratides Eurypon Leons Bzbr
% LocalWords:  Nicander Pausanias Pleistarchus UnAeranned Pleistoanax Andolian
% LocalWords:  Spaceborn's MacGyver Polydectes Polydorus EVAs Procles Prytanis
% LocalWords:  Soos Teleclus terraforming Aenethforming Zeuxidamus homeworld
% LocalWords:  coreward chemoreception Klk'k Aeneth ecologies Shmrn

\section{Portfolio: The Shmrn}
Ships overview for all groups (Work In Progress): \href{http://vegastrike.sourceforge.net/wiki/Artstyle\_guide:Overview\_Guide}{Ship Overview} \\
Art-style Guide (Work In Progress): \href{http://vegastrike.sourceforge.net/wiki/Artstyle\_guide:Shmrn}{Artstyle Guide} \\
Species overview: \href{http://vegastrike.sourceforge.net/wiki/Species:Shmrn}{Species:Shmrn} \\

\subsection{Origin}
\begin{itemize}
\item Gravity: 

\item Atmosphere: 

\item Primary liquid bodies: 

\item Average temperature of homeworld (pre-industrialization):

\item Sun: 

\item Primary challenges (pre-industrialization): 
\end{itemize}

%ORIGIN COMMENTS GO HERE

\subsubsection{Habitat}

\subsection{Physical}
\begin{itemize}
\item Dimensions: 

\item Mass: 

\item Skeletal system: 

\item Major divisions: 

\item Senses: 

\item Visual acuity: 

\item Chemosense: 

\item Locomotion: 

\item Manipulators: 

\item Textural appearance: 
\end{itemize}

%PHYSICAL COMMENTS GO HERE

\subsection{Mental}


\subsection{Technological}
\begin{itemize}
\item Tech: 


\item Weapons:

\item Tactics:
\begin{itemize}
\item    Small groups: 
\item    Large groups/Fleets: 
\end{itemize}


\item Installations:


\end{itemize}

\subsection{Culture}

\subsubsection{Factions and Organizational Groups}
Listed below are noteworthy Aeran sub-factions and organizational groups: 
\begin{itemize}
\item FACTIONS GO HERE
\end{itemize}

\subsubsection{Religion}

\subsubsection{Cultural Aesthetics}

\subsection{Writing, numbers, and insignia}

\subsection{Faction: PRIMARY FACTION}
%Faction data 
%Aera 
%Species 	Aera 
%Homeworld (Origin) 	Aeneth 
%Capital 	Aeneth 


\subsubsection{A Brief History of the PRIMARY FACTION}


\subsubsection{Development}

\subsubsection{Culture}

\subsubsection{Organization}

\subsection{Faction: OTHER FACTIONS}

%Faction data 
%Merchant Marines 
%Species 	Aera 
%Homeworld (Origin) 	Aeneth 
%Capital 	Aeneth 


\subsection{Vessels}

\subsubsection{Style Overview}
\begin{itemize}

\item Primary distinguishing color ranges: 

\item Common accent colors:

\item Primary lighting color:

\item Frequently visible: 

\item Rarely visible:

\item Seen inside, but not out: 

\item Moving parts(non-turret): 

\item Capital vs. light craft: 

\end{itemize}

\subsubsection{Surface features of large vessels}

\subsubsection{Small things found on the hull of a large  vessel}
\begin{itemize}
\item Service/Maintenance hatches
\end{itemize}
{\it Somewhat larger things found on the hull of a large Aeran vessel:}
\begin{itemize}
\item Escape pod launcher ports
\end{itemize}
{\it Yet larger things ... :}
\begin{itemize}
\item Pinnace/lander launch bay (non-carrier vessels)
\end{itemize}

\subsubsection{Listing of vessels}

\begin{itemize}
\item \href{http://vegastrike.sourceforge.net/wiki/Vessel:FOO}{FOO:} 

Existing concept art is not particularly canonical. Please redesign.

\end{itemize}

% LocalWords:  Aerans Aeran Artstyle Aera Pinnace Acrotatus Agasicles Agesilaus
% LocalWords:  Agesipolis Agis Alcmenes Anaxander Wiki Anaxandridas Rlaan Areus
% LocalWords:  Anaxidamus Ariston Charillus Cleombrotus Cleomenes Demaratus Uln
% LocalWords:  Theopompus Dorissus Echestratus Eurycratides Eurypon Leons Bzbr
% LocalWords:  Nicander Pausanias Pleistarchus UnAeranned Pleistoanax Andolian
% LocalWords:  Spaceborn's MacGyver Polydectes Polydorus EVAs Procles Prytanis
% LocalWords:  Soos Teleclus terraforming Aenethforming Zeuxidamus homeworld
% LocalWords:  coreward chemoreception Klk'k Aeneth ecologies Shmrn

\section{Portfolio: The Uln}
Ships overview for all groups (Work In Progress): \href{http://vegastrike.sourceforge.net/wiki/Artstyle\_guide:Overview\_Guide}{Ship Overview} \\
Art-style Guide (Work In Progress): \href{http://vegastrike.sourceforge.net/wiki/Artstyle\_guide:Uln}{Artstyle Guide} \\
Species overview: \href{http://vegastrike.sourceforge.net/wiki/Species:Uln}{Species:Uln} \\

\subsection{Origin}
\begin{itemize}
\item Gravity: 

\item Atmosphere: 

\item Primary liquid bodies: 

\item Average temperature of homeworld (pre-industrialization):

\item Sun: 

\item Primary challenges (pre-industrialization): 
\end{itemize}

%ORIGIN COMMENTS GO HERE

\subsubsection{Habitat}

\subsection{Physical}
\begin{itemize}
\item Dimensions: 

\item Mass: 

\item Skeletal system: 

\item Major divisions: 

\item Senses: 

\item Visual acuity: 

\item Chemosense: 

\item Locomotion: 

\item Manipulators: 

\item Textural appearance: 
\end{itemize}

%PHYSICAL COMMENTS GO HERE

\subsection{Mental}


\subsection{Technological}
\begin{itemize}
\item Tech: 


\item Weapons:

\item Tactics:
\begin{itemize}
\item    Small groups: 
\item    Large groups/Fleets: 
\end{itemize}


\item Installations:


\end{itemize}

\subsection{Culture}

\subsubsection{Factions and Organizational Groups}
Listed below are noteworthy Aeran sub-factions and organizational groups: 
\begin{itemize}
\item FACTIONS GO HERE
\end{itemize}

\subsubsection{Religion}

\subsubsection{Cultural Aesthetics}

\subsection{Writing, numbers, and insignia}

\subsection{Faction: PRIMARY FACTION}
%Faction data 
%Aera 
%Species 	Aera 
%Homeworld (Origin) 	Aeneth 
%Capital 	Aeneth 


\subsubsection{A Brief History of the PRIMARY FACTION}


\subsubsection{Development}

\subsubsection{Culture}

\subsubsection{Organization}

\subsection{Faction: OTHER FACTIONS}

%Faction data 
%Merchant Marines 
%Species 	Aera 
%Homeworld (Origin) 	Aeneth 
%Capital 	Aeneth 


\subsection{Vessels}

\subsubsection{Style Overview}
\begin{itemize}

\item Primary distinguishing color ranges: 

\item Common accent colors:

\item Primary lighting color:

\item Frequently visible: 

\item Rarely visible:

\item Seen inside, but not out: 

\item Moving parts(non-turret): 

\item Capital vs. light craft: 

\end{itemize}

\subsubsection{Surface features of large vessels}

\subsubsection{Small things found on the hull of a large  vessel}
\begin{itemize}
\item Service/Maintenance hatches
\end{itemize}
{\it Somewhat larger things found on the hull of a large Aeran vessel:}
\begin{itemize}
\item Escape pod launcher ports
\end{itemize}
{\it Yet larger things ... :}
\begin{itemize}
\item Pinnace/lander launch bay (non-carrier vessels)
\end{itemize}

\subsubsection{Listing of vessels}

\begin{itemize}
\item \href{http://vegastrike.sourceforge.net/wiki/Vessel:FOO}{FOO:} 

Existing concept art is not particularly canonical. Please redesign.

\end{itemize}

% LocalWords:  Aerans Aeran Artstyle Aera Pinnace Acrotatus Agasicles Agesilaus
% LocalWords:  Agesipolis Agis Alcmenes Anaxander Wiki Anaxandridas Rlaan Areus
% LocalWords:  Anaxidamus Ariston Charillus Cleombrotus Cleomenes Demaratus Uln
% LocalWords:  Theopompus Dorissus Echestratus Eurycratides Eurypon Leons Bzbr
% LocalWords:  Nicander Pausanias Pleistarchus UnAeranned Pleistoanax Andolian
% LocalWords:  Spaceborn's MacGyver Polydectes Polydorus EVAs Procles Prytanis
% LocalWords:  Soos Teleclus terraforming Aenethforming Zeuxidamus homeworld
% LocalWords:  coreward chemoreception Klk'k Aeneth ecologies Shmrn

\section{Portfolio: The Unadorned}
Ships overview for all groups (Work In Progress): \href{http://vegastrike.sourceforge.net/wiki/Artstyle\_guide:Overview\_Guide}{Ship Overview} \\
Art-style Guide (Work In Progress): \href{http://vegastrike.sourceforge.net/wiki/Artstyle\_guide:Unadorned}{Artstyle Guide} \\
Species overview: \href{http://vegastrike.sourceforge.net/wiki/Species:Humanity}{Species:Humanity} \\

\subsection{Origin}
\begin{itemize}
\item Gravity: 

\item Atmosphere: 

\item Primary liquid bodies: 

\item Average temperature of homeworld (pre-industrialization):

\item Sun: 

\item Primary challenges (pre-industrialization): 
\end{itemize}

%ORIGIN COMMENTS GO HERE

\subsubsection{Habitat}

\subsection{Physical}
\begin{itemize}
\item Dimensions: 

\item Mass: 

\item Skeletal system: 

\item Major divisions: 

\item Senses: 

\item Visual acuity: 

\item Chemosense: 

\item Locomotion: 

\item Manipulators: 

\item Textural appearance: 
\end{itemize}

%PHYSICAL COMMENTS GO HERE

\subsection{Mental}


\subsection{Technological}
\begin{itemize}
\item Tech: 


\item Weapons:

\item Tactics:
\begin{itemize}
\item    Small groups: 
\item    Large groups/Fleets: 
\end{itemize}


\item Installations:


\end{itemize}

\subsection{Culture}

\subsubsection{Factions and Organizational Groups}
Listed below are noteworthy Aeran sub-factions and organizational groups: 
\begin{itemize}
\item FACTIONS GO HERE
\end{itemize}

\subsubsection{Religion}

\subsubsection{Cultural Aesthetics}

\subsection{Writing, numbers, and insignia}

\subsection{Faction: PRIMARY FACTION}
%Faction data 
%Aera 
%Species 	Aera 
%Homeworld (Origin) 	Aeneth 
%Capital 	Aeneth 


\subsubsection{A Brief History of the PRIMARY FACTION}


\subsubsection{Development}

\subsubsection{Culture}

\subsubsection{Organization}

\subsection{Faction: OTHER FACTIONS}

%Faction data 
%Merchant Marines 
%Species 	Aera 
%Homeworld (Origin) 	Aeneth 
%Capital 	Aeneth 


\subsection{Vessels}

\subsubsection{Style Overview}
\begin{itemize}

\item Primary distinguishing color ranges: 

\item Common accent colors:

\item Primary lighting color:

\item Frequently visible: 

\item Rarely visible:

\item Seen inside, but not out: 

\item Moving parts(non-turret): 

\item Capital vs. light craft: 

\end{itemize}

\subsubsection{Surface features of large vessels}

\subsubsection{Small things found on the hull of a large  vessel}
\begin{itemize}
\item Service/Maintenance hatches
\end{itemize}
{\it Somewhat larger things found on the hull of a large Aeran vessel:}
\begin{itemize}
\item Escape pod launcher ports
\end{itemize}
{\it Yet larger things ... :}
\begin{itemize}
\item Pinnace/lander launch bay (non-carrier vessels)
\end{itemize}

\subsubsection{Listing of vessels}

\begin{itemize}
\item \href{http://vegastrike.sourceforge.net/wiki/Vessel:FOO}{FOO:} 

Existing concept art is not particularly canonical. Please redesign.

\end{itemize}

% LocalWords:  Aerans Aeran Artstyle Aera Pinnace Acrotatus Agasicles Agesilaus
% LocalWords:  Agesipolis Agis Alcmenes Anaxander Wiki Anaxandridas Rlaan Areus
% LocalWords:  Anaxidamus Ariston Charillus Cleombrotus Cleomenes Demaratus Uln
% LocalWords:  Theopompus Dorissus Echestratus Eurycratides Eurypon Leons Bzbr
% LocalWords:  Nicander Pausanias Pleistarchus UnAeranned Pleistoanax Andolian
% LocalWords:  Spaceborn's MacGyver Polydectes Polydorus EVAs Procles Prytanis
% LocalWords:  Soos Teleclus terraforming Aenethforming Zeuxidamus homeworld
% LocalWords:  coreward chemoreception Klk'k Aeneth ecologies Shmrn

%input{profile_}

% LocalWords:  UtCS
